%LaTeX2e
\documentclass[11pt]{article}
\usepackage{amsfonts,amssymb}
\textheight 9in
\textwidth 6.2in
\topmargin 0in
\headheight 0in
\headsep 0in
\oddsidemargin 0in
\newcounter{rmenum}
\newcounter{Rmenum}
\newcounter{alenum}
\newcounter{Alenum}
\newcounter{arenum}
\newcounter{arbenum}
\newcounter{arbdashenum}
\newenvironment{rmenumerate}{\begin{list}{(\roman{rmenum})}%
{\usecounter{rmenum}}}{\end{list}}
\newenvironment{Rmenumerate}{\begin{list}{(\Roman{Rmenum})}%
{\usecounter{Rmenum}}}{\end{list}}
\newenvironment{alenumerate}{\begin{list}{(\alph{alenum})}%
{\usecounter{alenum}}}{\end{list}}
\newenvironment{Alenumerate}{\begin{list}{(\Alph{Alenum})}%
{\usecounter{Alenum}}}{\end{list}}
\newenvironment{arenumerate}{\begin{list}{{\rm\arabic{arenum}}}%
{\usecounter{arenum}}}{\end{list}}
\newenvironment{arbenumerate}{\begin{list}{{\rm(\arabic{arbenum})}}%
{\usecounter{arbenum}}}{\end{list}}
\newenvironment{arbdashenumerate}{\begin{list}{{\rm(\arabic{arbdashenum})}$'$}%
{\usecounter{arbdashenum}}}{\end{list}}

\newenvironment{centre}{\begin{center}}{\end{center}}
\newcommand{\eg}{e.g.\ }
\newcommand{\ie}{i.e.\ }
\newcommand{\etc}{{\em et cetera }}
\newcommand{\sseq}{\subseteq}
\newcommand{\ra}{\rightarrow}
\newcommand{\Ra}{\Rightarrow}
\newcommand{\implies}{\;\;\Rightarrow\;\;}
\newcommand{\And}{\;\;\&\;\;}
\def\Bbb#1{\ensuremath{\mathbb{#1}}}
\def\MakeBbb#1{%
   \edef\tmp{%
      \noexpand\def\csname#1\endcsname
         {\noexpand\protect\csname p#1\endcsname}%
      \noexpand\def\csname p#1\endcsname
         {\noexpand\Bbb{#1}}%
   }%
   \tmp
}
\MakeBbb{C}
\MakeBbb{F}
\MakeBbb{N}
\MakeBbb{Q}
\MakeBbb{R}
\MakeBbb{T}
\MakeBbb{Z}

\let\leq\undefined  \let\geq\undefined
\DeclareMathSymbol{\leq}     {\mathrel}{AMSa}{"36}
\DeclareMathSymbol{\geq}     {\mathrel}{AMSa}{"3E}
\let\le\leq   \let\ge\geq
  \let\mho\undefined            \let\sqsupset\undefined
  \let\Join\undefined           \let\lhd\undefined
  \let\Box\undefined            \let\unlhd\undefined
  \let\Diamond\undefined        \let\rhd\undefined
  \let\leadsto\undefined        \let\unrhd\undefined
  \let\sqsubset\undefined

  \DeclareMathSymbol\mho     {\mathord}{AMSb}{"66}
  \DeclareMathSymbol\Box     {\mathord}{AMSa}{"03}
  \let\square\Box
  \DeclareMathSymbol\Diamond {\mathord}{AMSa}{"06}
  \DeclareMathSymbol\leadsto {\mathrel}{AMSa}{"20}
  \DeclareMathSymbol\sqsubset{\mathrel}{AMSa}{"40}
  \DeclareMathSymbol\sqsupset{\mathrel}{AMSa}{"41}
  \DeclareMathSymbol\lhd     {\mathrel}{AMSa}{"43}
  \DeclareMathSymbol\unlhd   {\mathrel}{AMSa}{"45}
  \DeclareMathSymbol\rhd     {\mathrel}{AMSa}{"42}
  \DeclareMathSymbol\unrhd   {\mathrel}{AMSa}{"44}
  \def\Join{\mathrel{{\rhd}\mkern-4mu{\lhd}}}


\def\nra{\hbox{$~\rightarrow~\kern-1.4em\hbox{/}~\kern+0.4em$}}
\def\sumprime_#1{\setbox0=\hbox{$\scriptstyle{#1}$}
	 \setbox2=\hbox{$\displaystyle{\sum}$}
	 \setbox4=\hbox{${}'\mathsurround=0pt$}
	 \dimen0=.5\wd0 \advance\dimen0 by-.5\wd2
	 \ifdim\dimen0>0pt
	 \ifdim\dimen0>\wd \kern\wd4 \else\kern\dimen0\fi\fi
  \mathop{{\sum}'}_{\kern-\wd4 #1}}
\newcommand{\nin}{\not\in}
\newcommand{\projtensor}{\hat{\otimes}}
\newcommand{\sgnbar}{\overline{\mbox{sgn}}}
\newcommand{\closure}[1]{\overline{#1}}
\renewcommand{\epsilon}{\varepsilon}
\renewcommand{\emptyset}{\mbox{\O}}
\newcommand{\dw}{dw}
\newcommand{\al}{\alpha}
\newcommand{\bt}{\beta}
\newcommand{\dl}{\delta}
\newcommand{\e}{\varepsilon}
\newcommand{\lm}{\lambda}
\newcommand{\onto}{\twoheadrightarrow}
\newcommand{\idealin}{\lhd}
\newtheorem{thm}{Theorem}[section]
\newtheorem{prop}[thm]{Proposition}
\newtheorem{cor}[thm]{Corollary}
\newtheorem{lemma}[thm]{Lemma}
\newenvironment{Proof}{{\it Proof. }}{}%no end of proof sign
\newenvironment{PoC}{{\it Proof of Corollary. }}{}
\newenvironment{PoL}{{\it Proof of Lemma. }}{}
\newenvironment{PoT}{{\it Proof of Theorem. }}{}
\newcommand{\defn}{\par\noindent{\bf Definition.}}
\newenvironment{defnun}{\par\noindent{\bf Definition. }}{\par}
\newenvironment{questionun}{\par\noindent{\bf Question. }}{\par}
\newenvironment{conjectureun}{\par\noindent{\bf Conjecture. }}{\par}
\newenvironment{thmun}{\par\noindent{\bf Theorem. }\begin{it}}{\end{it}\par}
\newenvironment{lemmaun}{\par\noindent{\bf Lemma. }\begin{it}}{\end{it}\par}
\newenvironment{propun}{\par\noindent{\bf Proposition. }\begin{it}}{\end{it}\par}
\newenvironment{corun}{\par\noindent{\bf Corollary. }\begin{it}}{\end{it}\par}
\newenvironment{defnno}[1]{\par\noindent{\bf Definition #1. }}{\par}
\newenvironment{questionno}[1]{\par\noindent{\bf Question #1. }}{\par}
\newenvironment{thmno}[1]{\par\noindent{\bf Theorem #1. }\begin{it}}{\end{it}\par}
\newenvironment{lemmano}[1]{\par\noindent{\bf Lemma #1. }\begin{it}}{\end{it}\par}
\newenvironment{propno}[1]{\par\noindent{\bf Proposition #1. }\begin{it}}{\end{it}\par}
\newenvironment{corno}[1]{\par\noindent{\bf Corollary #1. }\begin{it}}{\end{it}\par}
\newtheorem{definition}[thm]{Definition}
\newtheorem{example}[thm]{Example}
\newtheorem{examples}[thm]{Examples}
\newtheorem{exercise}[thm]{Exercise}
\newtheorem{remk}[thm]{Remark}
\newtheorem{remks}[thm]{Remarks}
\newtheorem{qu}[thm]{Question}
\newtheorem{conject}[thm]{Conjecture}
\newenvironment{Defn}{\begin{definition}\begin{rm}}{\end{rm}\end{definition}}
\newenvironment{Example}{\begin{example}\begin{rm}}{\end{rm}\end{example}}
\newenvironment{Examples}{\begin{examples}\begin{rm}}{\end{rm}\end{examples}}
\newenvironment{Exercise}{\begin{exercise}\begin{rm}}{\end{rm}\end{exercise}}
\newenvironment{Remark}{\begin{remk}\begin{rm}}{\end{rm}\end{remk}}
\newenvironment{Remarks}{\begin{remks}\begin{rm}}{\end{rm}\end{remks}}
\newenvironment{Question}{\begin{qu}\begin{rm}}{\end{rm}\end{qu}}
\newenvironment{Conjecture}{\begin{conject}\begin{rm}}{\end{rm}\end{conject}}
 \newcommand{\detail}[1]{}
 \newcommand{\Detail}[1]{}
%\newcommand{\detail}[1]{[#1]}
%\newcommand{\Detail}[1]{\par[{\bf Details.~~}#1]\par}
\newcommand{\nn}[1]{|#1|}
\newcommand{\bignn}[1]{\left|#1\right|}
\newcommand{\nnY}[1]{|#1|}
\newcommand{\bignnY}[1]{\left|#1\right|}
\newcommand{\Rbar}{\overline{R}}
\newcommand{\Rstar}{R^*}
\newcommand{\Rsubstar}{R_*}
\newcommand{\Rtilde}{\tilde R}
\newcommand{\barstar}[1]{{\closure{#1}\kern 1pt}^*}
\newcommand{\LE}{{\cal L}(E)}
\newcommand{\LH}{{\cal L}(H)}
\newcommand{\LW}{{\cal L}(W)}
\newcommand{\LX}{{\cal L}(X)}
\newcommand{\LY}{{\cal L}(Y)}
\newcommand{\LZ}{{\cal L}(Z)}
\newcommand{\UX}{{\cal U}(X)}
\begin{document}
%==============================================================================
\title{Topologically irreducible representations and radicals in Banach algebras}
\author{P.~G.~Dixon}
\date{15 June 1995, revised 12 February 1996}
\maketitle
\insert\footins{1991 {\it Mathematics Subject Classifications} 46H15, 46H25,
16Nxx.}
%==============================================================================
%\begin{abstract}
%It is shown that the topologically irreducible representations of a normed
%algebra define a certain topological radical in the same way that the strictly
%irreducible representations define the Jacobson radical and that this
%radical can be strictly smaller than the Jacobson radical.    An abstract theory
%of `topological radicals' in topological algebras is developed and
%used to relate this radical to the Baer radical (prime radical).    The
%relations with topologically transitive representations and standard
%representations in the sense of Meyer are also explored.
%\end{abstract}
%==============================================================================
\section{Introduction.}\label{S1}

The Jacobson radical of an associative algebra is the intersection of the
kernels of the strictly irreducible representations.  It is natural, when
studying normed algebras, to consider continuous `topologically irreducible'
representations on Banach spaces; \ie continuous homomorphisms of the algebra
onto algebras of bounded operators on Banach spaces for which no
non-trivial {\em closed} subspace is invariant, (where `non-trivial' means
having non-zero dimension and codimension).  Again, one looks at the
intersection of the kernels of all these representations of a given algebra.  We
shall show (Theorem \ref{T5}) that this is, in a reasonable sense, a
`topological radical'.

For Banach algebras, topological irreducibility is more general than strict
irreducibility, so our long-term aspiration is to use topologically irreducible
representations to study Jacobson radical Banach algebras.   However, whilst it
is easy to find continuous topologically irreducible representations which are
not strictly irreducible, it is not immediately clear that the intersection of
the kernels of these can be strictly smaller than the Jacobson radical.

One way to construct a topologically irreducible representation of a normed
algebra~$A$ is to find a continuous homomorphism $\phi:A\to B$ into a Banach
algebra~$B$ such that $\phi(A)$ is dense in~$B$ and~$B$ has strictly irreducible
representations.    Then every strictly irreducible representation of~$B$
induces a continuous topologically irreducible representation of~$A$.  This
construction is due to Meyer \cite{Meyer1}, who calls such representations {\em
standard}.     We shall use it in Section~\ref{S7} to produce a non-commutative
Banach algebra in which the radical described above is strictly smaller than the
Jacobson radical.

During the preparation of this paper, the author asked Charles Read whether his
work on the Invariant Subspace Problem could be extended to produce a {\it
quasi-nilpotent} operator on a Banach space with no closed invariant subspace.
Read was able to do this \cite{Readqn} and his example gives a second Banach
algebra in which the radical associated with topologically irreducible
representations is strictly smaller than the Jacobson radical.  Both examples
are important in our theory.   Read's example has the merit of being commutative;
ours, which is substantially easier, distinguishes the Jacobson radical from the
radical associated with the stronger condition of `topological transitivity'.

A normed representation~$\pi$ of an algebra~$A$ on a normed space~$X$ is said to
be {\em topologically transitive} if, whenever $\{x_1,\dots,x_n\}$,
$\{y_1,\dots,y_n\}$ are finite subsets of~$X$ with $\{x_1,\dots,x_n\}$ linearly
independent and $\e>0$, there is an element $a \in A$ with $\|\pi(a)x_i - y_i\|
< \e$ $(1 \le i \le n)$.  It follows from Jacobson's Density Theorem that all
standard representations of normed algebras have this property.    It is natural
to ask (\cite{Palmerbk} p.460, \cite{BD} p.132) whether every topologically
irreducible representation is topologically transitive.   We shall observe
(Corollary \ref{C2}) that Enflo's solution of the Invariant Subspace Problem for
Banach spaces gives a counterexample, because topologically transitive
representations of commutative algebras must be one-dimensional.  Generalizing
this, we show (Corollary \ref{C3}) that all topologically transitive
representations of PI-algebras are finite-dimensional.   Read's example shows
that the radicals associated with topologically irreducible and with
topologically transitive representations are distinct.

Our discussion of the radicals associated with these various types of
representation requires an abstract theory of `topological radicals'
in topological algebras.  We devote Section~\ref{S5} to setting up such a
theory.  The main problem is to choose the correct definitions: the theory seems
to have unusually `sensitive dependence on initial conditions', to borrow a
phrase from Chaos Theory.   Many reasonable variants on our chosen
axioms seem not to provide the desired results, (though we have not searched for
counterexamples to establish this, since our principal concern is with specific
radicals rather than the axiomatics).    With this theory in place, we can
produce topological radicals from maps which satisfy most but
not all of the axioms (UTRs and OTRs).   This enables us to relate the
new radicals to each other and to a topological radical derived from the Baer
radical.

Our theory of topological radicals has a variant which applies
to all normed algebras, not just to Banach algebras.  This is useful in order
to have an axiom about the radical of a continuous homomorphic image.
Unfortunately, the Jacobson radical is not a `topological radical' in this
version:  it is not necessarily closed!    However, as we show in
Section~\ref{S10}, the intersection of the kernels of the {\em continuous}
strictly irreducible representations on Banach spaces provides a good alternative
which coincides with the Jacobson radical in Banach algebras.

In Section~\ref{S9} we note the consequences of not requiring the representation
space to be complete and the paper concludes with a list of open questions.

I should like to thank Dr. John Rennison for pointing out errors in an earlier
draft of this paper.
%==============================================================================
\section{Definitions and abbreviations}\label{S2}

All algebras considered will be linear associative algebras over the complex
field.   They will not necessarily be commutative or unital.

A {\em representation} of an algebra~$A$ is a homomorphism~$\pi$ of~$A$ into the
algebra of all operators on a vector space~$X$.  We shall call~$\pi$ a {\em
normed representation} of the algebra~$A$ if~$X$ is a normed space and~$\pi$ is
a homomorphism of~$A$ into the algebra $\LX$ of all bounded operators on~$X$.
We can look at various refinements of this concept: we may make~$A$ a normed or
Banach algebra, we may then require the representation to be {\em continuous}
(with respect to the given norm on~$A$ and the operator norm on $\LX$).  We
shall generally do this; otherwise, we should be ignoring the topology on~$A$.
Also, we may require~$X$ to be a Banach space.   This too is a sensible option,
though we shall consider, in Section~\ref{S9}, the consequences of using
incomplete spaces.

A representation~$\pi$ of an algebra~$A$ on a vector space~$X$ is said to be
{\em strictly irreducible} if there is no subspace $Y \sseq X$ with $\{0\} \ne Y \ne X$
and $\pi(a)(Y) \sseq Y$ for all $a \in A$.  A normed representation $\pi:A \to
\LX$ of an algebra~$A$ on a normed space~$X$ is said to be {\em topologically
irreducible (TI)} is there is no closed subspace $Y \sseq X$ with $\{0\} \ne Y
\ne X$ and $\pi(a)(Y) \sseq Y$ for all $a \in A$.

For any algebra~$A$, we denote by $J(A)$ the {\em Jacobson radical} of~$A$,
which is the intersection of the kernels of all the strictly irreducible
representations of~$A$.
For a normed algebra~$A$, we define the {\em TI radical} $T(A)$ to be the
intersection of the kernels of all the continuous TI representations of~$A$ on
Banach spaces.
Equivalently, we can define a left Banach $A$-module to be {\em topologically
simple} if it has no closed submodule.    Then $T(A)$ is the intersection of the
annihilators of the topologically simple left Banach $A$-modules.

A representation~$\pi$ of an algebra~$A$ on a vector space~$X$ is said to be
{\em transitive} (or {\em strictly dense}) if, whenever $\{x_1,\dots,x_n\}$,
$\{y_1,\dots,y_n\}$ are finite subsets of~$X$ with $\{x_1,\dots,x_n\}$ linearly
independent, there is an element $a \in A$ with $\pi(a)x_i = y_i$ $(1 \le i \le
n)$.  In fact, if this holds for $n=2$, it holds for all~$n$ and the topological
version of Jacobson's Density Theorem (\cite{Palmerbk} 4.2.13, \cite{Rickart}
(2.4.7)) says that, for Banach algebras, every strictly irreducible representation is
transitive.  There is an obvious topological analogue:  for each positive
integer~$n$, a normed representation~$\pi$ of an algebra~$A$ on a normed
space~$X$ is said to be {\em topologically $n$-transitive} ($n$-TT) if, whenever
$\{x_1,\dots,x_n\}$, $\{y_1,\dots,y_n\}$ are subsets of~$X$ with
$\{x_1,\dots,x_n\}$ linearly independent and $\e > 0$, there is an element $a
\in A$ with $\|\pi(a)x_i - y_i\| < \e$ $(1 \le i \le n)$.   A representation is
said to be {\em topologically transitive} (TT)  if it is topologically
$n$-transitive for all positive integers~$n$.   This is `topologically
completely irreducible' in Palmer's terminology and is equivalent to saying that
$\pi(A)$ is dense in $\LX$ in the strong operator topology (the topology given
by the seminorms $T \mapsto \|Tx\|$ $(x \in X)$).    It is not known whether or
not $n$-TT for some $n \ge 2$ implies TT.  We write $T_n(A)$, $T_\infty(A)$ for the
intersection of the kernels of all the continuous $n$-TT, TT (respectively)
representations of~$A$ on Banach spaces.


A particular type of continuous TT representation arises as follows.
Let~$\rho$ be a strictly irreducible representation of a Banach algebra~$B$ on a linear
space~$X$.   Then there is a unique Banach space norm on~$X$ making the
representation normed and continuous (\cite{Palmerbk},~4.2.16(a), 4.2.15).
Let~$A$ be a normed algebra and $\phi:A \to B$ a continuous homomorphism such
that $\phi(A)$ is dense in~$B$.  Then $\pi = \rho\phi$ is a TT representation
of~$A$ on~$X$.
\detail{For if $x_1,\dots,x_n,$ $y_1,\dots,y_n$ are in~$X$ with
$\{x_1,\dots,x_n\}$ linearly independent, then there exists $b \in B$ such that
$\rho(b)(x_i) = y_i$ $(1 \le i \le n)$.   Given $\e > 0$, we find $a \in A$ such
that $\|\phi(a) - b\| < \e/\max_i\|x_i\|$ and then $\|\pi(a)(x_i) - y_i\| < \e$.}
Following Meyer \cite{Meyer1}, we call such TT representations {\em standard}.
The intersection of the kernels of all the standard TT representations of a
given normed algebra~$A$ will be denoted $S(A)$.

For a normed algebra~$A$, the inclusions
$$T(A) \sseq T_m(A) \sseq T_n(A) \sseq T_\infty(A) \sseq J(A) \qquad (m \le n)$$
are clear.   What is not immediately clear is whether any of these inclusions
can be strict.

%==============================================================================
\section{Elementary properties of TI representations}\label{S3}

Some of our later examples will produce, {\em inter alia}, TI representations
which are not strictly irreducible, but it is worth noting now that
satisfying these requirements alone is quite easy.

\begin{Example}\label{Ex1}
Let~$A= \ell^1(S_2)$ be the semigroup algebra of the free semigroup on two
generators $X,Y$.    Let~$T$ be the unilateral shift on $H=\ell^2$:
\begin{eqnarray*}
T(\xi_1,\xi_2,\xi_3,\dots) & = & (0,\xi_1,\xi_2,\dots)\\
T^*(\xi_1,\xi_2,\xi_3,\dots) & = & (\xi_2,\xi_3,\xi_4,\dots)
\end{eqnarray*}
Let~$\pi$ be the continuous representation of~$A$ on $\ell^2$ defined by
$\pi(\delta_X) = T$, $\pi(\delta_Y) = T^*$.   It is easy to see that
$\pi(A)$ is a *-subalgebra of $\LH$ with scalar commutant, so, by von Neumann's
Double Commutant Theorem, its strong closure is $\LH$, \ie $\pi$ is TT; but
$$\pi(A)((1,0,0,\dots))  = \ell^1,$$ so~$\pi$ is not strictly irreducible.
\end{Example}

\begin{Remark}
For *-representations of C*-algebras,
Kadison's Transitivity Theorem says that TI implies strictly irreducible
(\cite{Kadison}, see also \cite{Murphy} 5.2.2, \cite{Sakai} 1.21.17).  In
the example above, $\pi(A)$ is not closed in $\LH$.
\end{Remark}

We shall be seeking to relate the TI radical to radicals definable without
reference to representations.    In one direction this is easy, provided the
algebra is complete:  every strictly irreducible representation of a Banach algebra~$A$ has the
same kernel as some continuous strictly irreducible representation of~$A$ on a Banach
space (\cite{Palmerbk} 4.2.9, \cite{Rickart} (2.4.7)).    Hence the TI radical
of a Banach algebra is contained in the Jacobson radical, which has many
characterizations not directly involving representations (largest quasi-regular
ideal, largest ideal of topologically nilpotent elements, intersection of the
maximal modular left ideals).   It is not immediately clear
that this inclusion can be strict---in Example \ref{Ex1} above, the algebra~$A$
is semisimple so there are many other representations which are strictly
irreducible---but we shall give an example later where this is so.

In the other direction, the only results we know stem from the following
proposition.

\begin{prop}\label{P1} {\rm (\cite{Palmerbk} 4.2.5(a), 4.4.9(a))}
The kernel of a TI representation of an algebra~$A$ on a normed space is a prime
ideal of~$A$.   Hence, the intersection of the kernels of the TI representations of an algebra~$A$
contains the Baer radical of~$A$.
\end{prop}

The {\em Baer radical} or {\em prime radical} $\bt(A)$ of an algebra~$A$
is the intersection of all the prime ideals of~$A$; equivalently, it is the
smallest ideal~$I$ of~$A$ such that $A/I$ has no non-zero nilpotent ideals
(\cite{Palmerbk} 4.4.6).  It is the smallest of three radicals, the others being
the Levitzki radical and the nil radical, that coincide for Banach algebras
\cite{PGD7}.   However, Corollary \ref{C11} below shows that, for incomplete
normed algebras, the TI radical does not necessarily contain the other two
radicals.

When we consider {\em continuous} TI representations of a normed algebra,
we have the further information that the kernels of the representations are
closed.    Consequently, the TI radical of a normed algebra~$A$ contains the
closure of the Baer radical $\overline{\bt(A)}$.   However,
$A/\overline{\bt(A)}$ might fail to be semiprime, in which case the preimage
in~$A$ of its Baer radical is also included in the TI radical, as is its
closure, and so on.  This leads us to construct (in Corollary \ref{C4}) a new
radical, the {\em closed-Baer radical} $\barstar{\beta}$, to give a good lower
bound for the TI radical.

%==============================================================================
\section{Classical problems}\label{S4}

The difficulty of working with TI representations is well illustrated by their
relation to some famous problems of functional analysis.

\begin{prop} \label{P2}
The following are equivalent (and true):
\begin{arbenumerate}
\item there is a singly-generated (as a Banach algebra) Banach algebra with a
continuous faithful TI representation on an infinite-dimensional Banach space;
\item there is a singly-generated Banach algebra with a continuous non-zero TI
representation on an infinite-dimensional Banach space;
\item there is an operator on an infinite-dimensional Banach space with no
non-trivial closed invariant subspace.
\end{arbenumerate}
\end{prop}

\begin{Proof}
The truth of (3) is Enflo's solution of the Invariant Subspace Problem for Banach
Spaces \cite{Enflo} (see also \cite{Read}, \cite{Beauzamy} Chapter XIV).
The proof of the equivalence of (1), (2) and (3) is straightforward.
\Detail{%
\begin{description}
\item{(1) $\Rightarrow$ (2)} is trivial.
\item{(2) $\Rightarrow$ (3)}.
Let~$A$ be a Banach algebra with generator~$a$ and let~$\pi$ be a continuous TI
representation of~$A$ on a Banach space~$X$ such that $\pi(a) \ne 0$.    Then
there is no non-trivial closed invariant subspace of $\pi(a)$, since any such
subspace would be invariant for $\pi(A)$, contradicting the fact that~$\pi$ is TI.
\item{(3) $\Rightarrow$ (1)}.
Given an operator~$T$ on a Banach space~$X$ with no non-trivial closed invariant
subspace, let~$A$ be the closed subalgebra of $\LX$ generated by~$T$.   Then the
natural representation of~$A$ on~$X$ is TI, faithful and continuous.
\end{description}}
\end{Proof}

Note the sharp contrast with the situation for strictly irreducible
representations:  it follows from Schur's Lemma that strictly irreducible representations of commutative
Banach algebras must be one-dimensional (\cite{Palmerbk} 4.2.19).

\begin{prop}\label{P3}
The following are equivalent (and true):
\begin{arbenumerate}
\item there is a singly-generated (as a Banach algebra) radical Banach algebra
with a continuous faithful TI representation on an infinite-dimensional Banach
space;
\item there is a singly-generated (as a Banach algebra) radical Banach algebra
with a continuous non-zero TI representation on an
infinite-dimensional Banach space;
\item there is a quasi-nilpotent operator on an infinite-dimensional Banach
space with no non-trivial closed invariant subspace.
\end{arbenumerate}
\end{prop}

\begin{Proof}
The truth of (3) is a recent result of Read \cite{Readqn}.   We prove the
equivalence of (1), (2) and (3).

In (2)$\Rightarrow$(3), if $\pi:A\to\LX$ is a continuous, TI representation with
$\pi(a)\ne 0$ for some $a\in A$, then the desired operator $\pi(a)$ is
quasi-nilpotent.  In (3)$\Rightarrow$(1), if~$T$ is the given quasi-nilpotent operator
on~$X$, then the closed subalgebra of $\LX$ that it generates is radical.
\Detail{Since~$T$ is quasi-nilpotent, so are all the polynomials in~$T$.   The
closed subalgebra $A \sseq \LX$ generated by~$T$ is a commutative Banach
algebra, so its Jacobson radical is the set of all quasi-nilpotents and this set
is closed.  Since the polynomials in~$T$ are dense in~$A$, it follows that~$A$
is Jacobson radical.}
\end{Proof}

\begin{Remark}
Since finite-dimensional subspaces are automatically closed, all
finite-dimensional TI representations of algebras are strictly irreducible.
Hence, if a radical algebra has TI representations, they must be
infinite-dimensional.
\end{Remark}

Finally, we note that the famous problem of the existence of a topologically
simple commutative radical Banach algebra is equivalent to asking for a
commutative radical Banach algebra for which the left regular representation
is TI.
%============================================================================
\section{Topologically transitive representations}\label{S4a}

The obvious first question about TT representations is whether there are
TI representations which are not TT.   One way in which such representations
might occur is as the left regular representations of radical Banach algebras
with no non-trivial closed left ideals, if such exist.

\begin{thm}\label{T0}
If~$A$ is a Banach algebra of dimension greater than~1, then the left regular
representation of~$A$ on~$A$ is not 2-TT.
\end{thm}

\begin{Proof}
We begin by proving this under the assumption\\
(*) there are elements $x,y \in A$ such that~$x$ and $xy$ are linearly
independent.

Suppose the representation is 2-TT.  Then, for every
$\e > 0$ there is an element $a \in A$ such that
$$\|axy - x\| < \e\qquad \mbox{and} \qquad \|ax\| < \e.$$
Then, for every~$\e>0$,
$$\|x\| \le \|axy - x\| + \|ax\|\,\|y\| < \e(1+\|y\|).$$
Thus $x=0$, contradicting (*).

Now assume that (*) is false.   Then, for every $a,x_1,x_2 \in A$ with
$\{x_1,x_2\}$ linearly independent, the elements $ax_1$ and $ax_2$ lie in the
same 1-dimensional subspace (spanned by~$a$).  Thus we can not make choices of
$a$ which bring $ax_1, ax_2$ indefinitely close to two given linearly
independent vectors $y_1,y_2$; so the left regular representation is not 2-TT.
\end{Proof}

\begin{Remark}
The problem of whether there exists a radical Banach algebra with no non-trivial
closed left ideals lies between two unsolved problems: the existence of a
topologically simple radical Banach algebra \detail{no non-trivial closed
two-sided ideals}\ and the existence of a topologically simple commutative
radical Banach algebra.  \detail{Might it be equivalent to one of these questions?}
\end{Remark}

Another approach to constructing TI, non-TT representations
leads to the Invariant Subspace Problem, and therefore succeeds.
The following theorem is probably the best topological analogue of Schur's Lemma
on strictly irreducible representations.    (Remember that $\LX$ here denotes
the algebra of all {\em bounded} operators on~$X$.)

\begin{thm}\label{T1}
If $(\pi,X)$ is a 2-TT representation of a (not
necessarily normed) algebra~$A$ on a normed space~$X$, then
$$\{T \in \LX: T\pi(a) = \pi(a)T\quad(a \in A)\} = \C I.$$
In particular, if~$A$ is commutative then $\dim X = 1$.
\end{thm}

\begin{Proof}
Suppose $T\pi(a) = \pi(a)T$ and~$T$ is not a multiple of the identity.
Let $\xi \in X$ be such that~$\xi$
and $T\xi$ are linearly independent.    Let~$\eta,\zeta \in X$ be
arbitrary.    If~$\pi$ were 2-TT, we could find a sequence
$(b_n)$ in~$A$ with $\pi(b_n)\xi \to \eta$ and $\pi(b_n)(T\xi) \to \zeta$.
However,
$$\pi(b_n)(T\xi) = T(\pi(b_n)\xi) \to T\eta.$$
Therefore $\zeta = T\eta$, contradicting the arbitrariness of~$\zeta$.
\end{Proof}

\begin{cor}\label{C2b}
For a commutative Banach algebra, $T_2(A) = J(A)$.
\end{cor}

Applying Theorem \ref{T1} to the infinite-dimensional TI representation derived
from the solution to the Invariant Subspace Problem (Proposition~\ref{P2}(1)),
we obtain the following corollary.

\begin{cor}\label{C2}
There is a commutative Banach algebra with a continuous TI representation
which is not 2-TT.
\end{cor}

Read's new example (Proposition~\ref{P3}) yields a significantly stronger
statement.

\begin{cor}\label{C2a}
There is a commutative Banach algebra with $T(A) \ne T_2(A)$.
\end{cor}

\begin{Remark}
Beauzamy (\cite{Beauzamy} Chapter XIV) and Read's papers \cite{Readl1},
\cite{Readqn} on the invariant subspace problem give examples where the Banach
space is $\ell_1$, so, in these corollaries, the pathology may be confined to
the algebra (whose structure is unclear) and the representation, rather than the
Banach space.   We conjecture that there are examples with straightforward
algebras and Banach spaces, the pathology being confined just to the
representations.
\end{Remark}

It is interesting to explore generalizations of Theorem \ref{T1} to algebras
satisfying polynomial identities.  Bearing in mind the Amitsur--Levitzki
Theorem, that $M_n(\C)$ satisfies the standard polynomial identities $S_k$ for
$k \ge 2n$, we make the following conjecture.

\begin{Conjecture}
Let~$n \ge 1$. If~$A$ is an algebra satisfying the standard polynomial identity
$S_{2n-1}$, then every $n$-TT representation of~$A$ on a
normed space~$X$ has $\dim X < \infty$, (and is therefore a strictly irreducible
representation with $\dim X < n$).
\end{Conjecture}

The obvious approach goes as follows.  Suppose $\pi:A \to \LX$ is a
$n$-TT representation with $\dim X \ge n$.  Let
$\{e_1,\dots e_n\}$ be a linearly independent set in~$X$.   Now the algebra
$M_n(\C)$ of $n \times n$ matrices does not satisfy $S_{2n-1}$.   Let
$T_1,\dots,T_{2n-1}$ be linear mappings  on the span of $\{e_1,\dots e_n\}$ such
that $S_{2n-1}(T_1,\dots,T_{2n-1}) \ne 0$.   Since~$\pi$ is $n$-TT, we
can find $a_1,\dots,a_{2n-1} \in A$ with $\pi(a_i)e_j$ approximating $T_ie_j$
$(1 \le i \le 2n-1,\;1 \le j \le n)$.    Unfortunately, we have no control over
the norms $\|a_i\|$ and so the elements $\pi(a_{i_1})\dots\pi(a_{i_{2n-1}})e_j$
might not approximate to $T_{i_1}\dots T_{i_{2n-1}}e_j$.

\begin{thm}\label{T1.5}
Let~$n \ge 1$. If~$A$ is an algebra satisfying the standard polynomial
identity $S_{n+1}$, then every $2^n$-TT representation
of~$A$ on a normed space~$X$ has $\dim X < \infty$, (and is therefore a strictly
irreducible representation with $\dim X \le [(n+1)/2]$).
\end{thm}

\begin{Proof}
The following notation will be useful:  if $B = \{b_1,\dots,b_k\} \sseq A$
and~$\pi$ is our representation of~$A$, then
$$S(B) = \pm S_k(\pi(b_1),\dots,\pi(b_k)),$$
where we shall ignore the sign.  (This is a convenient shorthand; a more
detailed proof merely requires a straightforward, but obfuscating, replacement
of this notation by one dependent on particular orderings of the subsets~$B$ for
which $S(B)$ is used.)   We write $S(\emptyset) = I$, the identity operator.

We shall prove, by induction on~$n$, that if $\pi:A \to \LX$ is a
$2^n$-TT representation of an algebra~$A$ on an
infinite-dimensional normed space~$X$ and $x \in X\setminus\{0\}$, then
there exist $a_1,\dots,a_{n+1} \in A$ such that
$\{S(B)x: B \sseq \{a_1,\dots,a_{n+1}\}\}$ is linearly independent.
In particular, this implies that
$$\pi(S_{n+1}(a_1,\dots,a_{n+1})) = \pm S(\{a_1,\dots,a_{n+1}\}) \ne 0,$$
so~$A$ does not satisfy the identity $S_{n+1}$.

The induction starts trivially at $n=0$.
Suppose the result has been proved for $n-1$, where $n \ge 1$.   Given
a $2^n$-TT representation $\pi:A \to \LX$ and $x \in X\setminus\{0\}$, we use
the induction hypothesis to find $a_1,a_2,\dots,a_n \in A$ such that the set
$E_0=\{S(B)x: B \sseq \{a_1,\dots,a_n\}\}$ is linearly independent.

As a first approximation to $a_{n+1}$, we find an element $b_1 \in A$ such that
$\pi(b_1)x \nin {\rm span} E_0$, \ie the set
$$E_1 = \{S(B)x: B \sseq \{a_1,\dots,a_n\}, B=\{b_1\}\}$$
is linearly independent.

Let $W_1=\emptyset,W_2,W_3,\dots,W_{2^n}$ be an enumeration of all the subsets
of $\{a_1,\dots,a_n\}$, ordered so that $|W_i| \le |W_j|$ $(i \le j)$,
(where $|W|$ denotes the cardinality of~$W$).
We construct successive approximations
$b_k$ to the desired $a_{n+1}$ such that
$$E_k = \{S(W_i\cup \{b_k\})x:1 \le i \le k\}\cup E_0$$
is linearly independent.  We will then set $a_{n+1}=b_{2^n}$.

We have already described the construction of $b_1$.   Suppose $b_{k-1}$ has
been constructed as above.   Each $S(W_i\cup \{b_k\})x$ $(i<k)$ may be written
$$S(W_i\cup \{b_k\})x = \sum\pm S(U)\pi(b_k)S(V)x,$$
where the summation is over all partitions $U \cup V = W_i$
and is therefore a continuous function of the vectors $\pi(b_k)S(W_j)x \in X$
$(j \le i)$.   The induction hypothesis on $b_{k-1}$ means that the set
$$F_k=\{S(W_i\cup \{b_k\})x:1 \le i \le k-1\}\cup
                                  \{S(B)x: B \sseq \{a_1,\dots,a_n\}\}$$
is linearly independent when $b_k = b_{k-1}$ (making $F_k=E_{k-1}$).
If $S(W_k\cup \{b_{k-1}\})x$ is not in the linear span of $E_{k-1}$, then we
may set $b_k = b_{k-1}$. Suppose otherwise.  We shall set $b_k = b_{k-1} + d_k$
for some perturbation~$d_k$ which is `small' {\em in the sense that} the vectors
$\pi(d_k)S(W_j)x$ are small for all $j < k$.  Our basic idea is that by the
stability of linear independence under small perturbations,  (see \eg\
\cite{Jameson} Corollary 20.7), the fact that $F_k$ is linearly independent
remains true as $b_k$ is perturbed away from $b_{k-1}$, provided that $d_k$ is
sufficiently small in the above sense.

Choose a vector~$y$ outside the span of $E_{k-1}$.  Then the set
$E_{k-1} \cup \{S(W_k\cup \{b_{k-1}\})x + y\}$ is linearly independent.
By the induction hypothesis on $a_n$, the points $S(W_j)x$ $(1 \le j \le 2^n)$
are linearly independent.   These $2^n$ points may be
separated by~$\pi$;  we can therefore find `small' $d_k$ so that $\pi(d_k)S(W_k)x$
approximates~$y$.  Now
$$S(W_k\cup \{b_k\})x = S(W_k\cup \{b_{k-1}\})x + \pi(d_k)S(W_k)x
                                + \sumprime_{}\pm S(U)\pi(d_k)S(V)x$$
where the sum $\sum'$ is taken over all partitions $U \cup V = W_k$ with
$U \ne \emptyset$.  The last term is a continuous function of the vectors
$\pi(d_k)S(W_j)x$ $(j < k)$.  Therefore, if we choose sufficiently `small' $d_k$
with $\pi(d_k)S(W_k)x$ sufficiently close to ~$y$, then we can ensure that the
perturbation from
$$E_{k-1} \cup \{S(W_k\cup \{b_{k-1}\})x + y\}$$
to
$$E_k = F_k \cup \{S(W_k\cup \{b_k\})x\}$$
is small enough to preserve linear independence.  This completes the induction
step in the construction of the $b_k$.   Putting $a_{n+1}=b_{2^n}$ then
completes the induction step for the whole proof.
\end{Proof}

\begin{cor}\label{C3}
Every TT representation of a PI-algebra is finite-dimensional and hence
strictly irreducible.
\end{cor}

\begin{Proof}
If~$A$ is a PI-algebra, then $A/\bt(A)$ is PI and so, by a corollary of
Kaplansky's Theorem (\cite{Rowen} Theorem 6.1.28), satisfies a standard
identity.

Every TT representation~$\pi:A\to\LX$ has $\ker\pi\supseteq\bt(A)$ and so
induces a representation~$\pi':A/\bt(A)\to\LX$ with $\pi'(A/\bt(A)) = \pi(A)$.
In particular,~$\pi'$ is TT.   By Theorem \ref{T1.5}, $\pi'$ is strictly
irreducible, so~$\pi$ is strictly irreducible.
\end{Proof}
%==============================================================================
\section{A general theory of radicals}\label{S5}

In this section, we develop a little of a general theory of radicals in
normed algebras.    The calculations are generally straightforward once the
correct definitions are in place; but the theory is quite sensitive to
these.   Our attempts based on only slightly different definitions, such
as the obvious analogue of Divinsky's algebraic definition or a definition
including non-closed ideals in~(4) below have foundered on seemingly
insignificant technicalities.

Nevertheless, several variations do work.    In the following definition we
present simultaneously our algebraic and topological notions of a `radical', and
some of our theorems will exist in both contexts.  In Section~\ref{S10} we shall
discuss the variant of the topological version in which radicals are defined for
incomplete algebras.

\begin{Defn}
By a {\em radical} (respectively, a {\em topological radical}), we mean a
map~$R$ associating with each algebra (Banach algebra)~$A$
a (closed) ideal $R(A) \idealin A$ such that the following hold.
\begin{arbenumerate}
\item $R(R(A)) = R(A)$.
\item $R(A/R(A)) = \{0\}$, where $\{0\}$ denotes the zero coset in $A/R(A)$.
\item If $A,B$ are (Banach) algebras and $\phi:A \onto B$ is a
(continuous) epimorphism, then $\phi(R(A))\sseq R(B)$.  (`Epimorphism' here
means just `surjective homomorphism'.)
\item If $I$ is a (closed) ideal of $A$, then
\begin{alenumerate}
   \item $R(I)$ is a (closed) ideal of $A$ and
   \item $R(I) \sseq R(A) \cap I$.
\end{alenumerate}
\end{arbenumerate}
We say that~$R$ is a {\em hereditary (topological) radical} if it
satisfies~(2),~(3),~(4) and
\begin{arbenumerate}
\addtocounter{arbenum}{4}
\item If $I$ is a (closed) ideal of $A$, then $R(I) \supseteq R(A) \cap I$.
\end{arbenumerate}
(Note that (5)$\implies$(1).)

We say that an algebra~$A$ is {\it $R$-semisimple} if $R(A) = \{0\}$ and
{\it $R$-radical} if $R(A)=A$.
\end{Defn}

We have preferred to cast our theory in terms of maps, rather than `radical
property' used by other authors (e.g.\ Divinsky \cite{Divinskybk} p.3),
to facilitate generalizations (Definition \ref{D2}).  Translation between the
two forms is easy: the `property' corresponding to a radical~$R$ is~$A$ being
equal to $R(A)$; the map corresponding to a given property associates with a
(Banach) algebra~$A$ the largest (closed) ideal of~$A$ with the given property.
Divinsky's definition and its obvious topological analogue, stated in terms
of a map~$R$,  are our definitions with (3) and (4) replaced by
\begin{arbdashenumerate}
\addtocounter{arbdashenum}{2}
\item If $A,B$ are (Banach) algebras, $A=R(A)$ and $\phi:A \onto B$ is a
(continuous) epimorphism, then $B=R(B)$.
\item if~$I$ is a (closed) ideal of~$A$ with $R(I) = I$, then $I \sseq R(A)$.
\end{arbdashenumerate}
In the algebraic case, this is easily equivalent to our definition, but it is
not clear whether this is so in the topological case.  (Clearly $(3) \Rightarrow
(3)'$ and $(4) \Rightarrow (4)'$, in both cases.   Also the implication
$((2)\;\&\;(3)'\;\&\;(4)') \Rightarrow (4)$ can be proved in the topological case by the
method of Divinsky's Theorem~47 if the radical satisfies a weak
non-triviality condition:  that all Banach algebras with zero multiplication be
radical.)   We claim that our definition is at least as aesthetically satisfying
as Divinsky's and is easier to work with in the topological case and we leave the
detailed investigation of the relation between the two for others to study.

\Detail{In the algebraic case, $((1)\;\&\;(3)'\;\&\;(4)')\Rightarrow (3)$: apply
$(3)'$ to the restricted mapping $\phi: R(A) \onto \phi(R(A))$.  Then~(1)
implies $R(\phi(R(A))) = \phi(R(A))$.  Since $\phi:A \onto B$ is surjective and
$R(A) \idealin A$, we have $\phi(R(A)) \idealin B$.  An application of~$(4)'$
then gives $\phi(R(A)) \sseq R(B)$.    However, in the topological case,
$\phi(R(A))$ might not be closed, so neither~$(3)'$ nor~$(4)'$ would
necessarily apply.

In the algebraic case, also, $((2)\;\&\;(3)'\;\&\;(4)')\Rightarrow (4)$ by
Divinsky's Theorem~47 and its first corollary.  In the topological case,
however, the proof of Theorem~47 fails because sums and products of closed
subspaces are not necessarily closed.   If all null algebras are radical, the
section of Divinsky's proof which shows that $[xR(I)+R(I)]/R(I)$ is null goes
through as before.  We deduce that $\closure{xR(I)+R(I)}/R(I)$ is null and
therefore radical.  The proof concludes as before.}

Condition (5) has the following equivalent (in the presence of (4)) formulation
(see \cite{Divinskybk} p.123, Lemma 68).
\begin{arbdashenumerate}
\addtocounter{arbdashenum}{4}
\item Every (closed) ideal $I$ of~$A$ with $I \sseq R(A)$ has $I = R(I)$.
\end{arbdashenumerate}
\detail{That (5)$\implies$(5)$'$ is obvious.  Conversely, given (5)$'$, then the argument
\begin{eqnarray*}
I\cap R(A) & = & R(I \cap R(A)), \qquad\mbox{by (5)$'$ applied to }I \cap R(A) \sseq R(A)\\
&\sseq&R(I) \cap I \cap R(A), \qquad\mbox{by (4) applied to }I \cap R(A) \idealin I
\end{eqnarray*}
proves (5).}

We shall order radical and other such maps by inclusion: we write $R \le S$ to
mean $R(A) \sseq S(A)$ for all algebras~$A$ (all Banach algebras~$A$,
in the topological case).

\begin{Defn}\label{D2}
We shall say that a map $A \mapsto R(A)$ which associates with each (Banach)
algebra a (closed) ideal is an {\em under radical (UR)}, (respectively,
{\em under topological radical (UTR)}) if it satisfies
(1), (3) and (4).   We shall say that it is an {\em over radical (OR)}, (respectively,
{\em over topological radical (OTR)}) if it satisfies
(2), (3) and (4).
\end{Defn}

The reason for the terminology is that we shall show how a radical can
be constructed above a given UR (Theorem \ref{T3}) and
below a given OR (Theorem \ref{T4}).    (We avoid the
Latin prefixes `sub' and `super' lest common usage of the latter should
suggest something stronger than radical.)

One way in which UTRs arise is in trying to convert algebraic
radicals into topological radicals by taking closures.

\begin{thm}\label{T2}
Let~$R$ be a UR.   For Banach algebras~$A$, define
$$\Rbar(A) = \closure{R(A)}.$$
Then $\Rbar$ is a UTR.
\end{thm}

The proof is straightforward.
\Detail{\begin{Proof}
The map~$R$ has the algebraic properties (1), (3) and (4).  We must prove
the topological versions of (1), (3) and (4) for $\Rbar$.
\begin{arbenumerate}
%-------------------------------
\item The set $R(A)$ is an ideal of~$A$ and hence of $\Rbar(A)$.   Therefore,
by (2) and (4), $R(A) = R(R(A)) \sseq R(\Rbar(A))$.  Taking closures, we get
$\Rbar(A) \sseq \Rbar(\Rbar(A))$. The reverse inclusion is trivial.
%-------------------------------
\addtocounter{arbenum}{1}
\item If $\phi:A \onto B$ is a continuous epimorphism,
then $\phi(R(A)) \sseq R(B)$, by (3), so
$\phi(\Rbar(A)) \sseq \closure{\phi(R(A))} \sseq \Rbar(B).$
%-------------------------------
\item
If $I$ is a closed ideal of~$A$ then $R(I) \idealin A$ by (4)(a) and
$\Rbar(I)$, which is defined as the closure of $R(I)$ in~$I$, is the closure
of $R(I)$ in~$A$.  Therefore $\Rbar(I)$ is a closed ideal of~$A$.
Moreover, by (4)(b), $R(I) \sseq R(A) \cap I \sseq \Rbar(A) \cap I$.
Now $\Rbar(A) \cap I$ is closed in $I$, so $\Rbar(I) \sseq \Rbar(A) \cap I$.
%-------------------------------
\end{arbenumerate}
\end{Proof}}
Unfortunately, the map $\Rbar$ need not satisfy (2), even if~$R$ does.
The natural example of this is the following.

\begin{Example}\label{C01}
Let~$A$ be the Banach algebra $(C[0,1],*)$ of all bounded, continuous,
complex-valued functions on the unit interval, with convolution multiplication
and let~$\beta$ be the Baer radical map.   Then, using the Titchmarsh
Convolution Theorem, $\beta(A)$ is the ideal of all functions vanishing on a
neighbourhood of zero.   Hence $\overline{\beta}(A)$ is the ideal of
functions vanishing at zero.   The quotient $A/\overline{\beta}(A)$ is
the one-dimensional algebra with zero multiplication, so
condition (2) fails.
\end{Example}

If $\Rbar(A/\Rbar(A)) \neq \{0\}$, then we have to look at the inverse image
in~$A$ of $\Rbar(A/\Rbar(A))$ under the quotient map $A \onto A/\Rbar(A)$.
There is no reason why the quotient of~$A$ by this ideal should be
$\Rbar$-semisimple, so we again look at the inverse image of the radical.
We may expect to have to continue this process transfinitely to get
a topological radical.

\begin{Defn}\label{D3}
Let $R$ be a map associating with each (Banach) algebra~$A$ a (closed)
ideal $R(A)$.   We define a transfinite sequence of such maps $(R_\al)$ by:
\begin{rmenumerate}
\item $R_0(A) = \{0\}$;
\item $R_{\al+1}(A) = q^{-1}\left(R(A/R_\al(A))\right)$, where $q:A \to A/R_\al(A)$
is the quotient map,    (so, for example, $R_1 = R$);
\item for limit ordinals~$\lm$, $$R_\lm(A) = \bigcup_{\al < \lm}R_\al(A)$$ in the
algebraic case and $$R_\lm(A) = \closure{\bigcup_{\al < \lm}R_\al(A)}$$ in the
topological case.
\end{rmenumerate}
The transfinite sequence of sets $(R_\al(A))$ is monotonic non-decreasing, so it
must stabilise at the $\al$th stage, where~$\al$ is at most the cardinality
of~$A$.  We then write $\Rstar(A) = R_\al(A) = R_{\al+1}(A)$.
\end{Defn}

For example, if $R = \overline{\beta}$, then $\Rstar(A) = R_2(A)$ for the
algebra~$A$ of Example \ref{C01} above.

\begin{thm}\label{T3}
If~$R$ is a UR (respectively, a UTR), then so is $R_\al$,
for every ordinal~$\al$, and $\Rstar$ is a radical (topological radical).
\end{thm}

\begin{Proof}
We prove the topological case, which is the one of most interest to us now.
The algebraic case is similar and easier.

The fact that $\Rstar$ satisfies condition (2) is easy: because the sequence
has stabilised,
$$\Rstar(A) = q^{-1}(R(A/\Rstar(A))),$$
\ie $R(A/\Rstar(A)) = \{0\}$.   It follows that $R_\al(A/\Rstar(A)) = \{0\}$
for all~$\al$, and so $\Rstar(A/\Rstar(A)) = \{0\}$.

The rest of the proof consists of showing by transfinite induction on~$\al$,
that $R_\al$ satisfies conditions (1), (3) and (4).    The result is trivially
true for $\al = 0$, which starts the induction.   Most of the work lies in the
induction step to successor ordinals.  Suppose $R_\al$ is a topological radical.   We
write $(n)_\al$ to mean condition $(n)$ on $R_\al$.
\begin{arbenumerate}
%-------------------------------
\item
\begin{eqnarray*}
R_{\al+1}(R_{\al+1}(A)) & = & q^{-1}(R(R_{\al+1}(A)/R_\al(A)))\\
& = & q^{-1}(R(R(A/R_\al(A)))) \\
& = & q^{-1}(R(A/R_\al(A))), \qquad\mbox{ by }(1)_1, \\
& = & R_{\al+1}(A).
\end{eqnarray*}
%-------------------------------
\addtocounter{arbenum}{1}
\item
Suppose $\phi:A \onto B$ is a continuous epimorphism between
Banach algebras.    By $(3)_\al$, we have $\phi(R_\al(A))
\sseq R_\al(B)$.  Hence~$\phi$ induces a continuous homomorphism
$\psi:A/R_\al(A) \onto B/R_\al(B)$\detail{ by $\psi(a + R_\al(A)) = \phi(a) + R_\al(B)$
with $\|\psi\| \le \|\phi\|$}.  By $(3)_1$,
$$\psi(R(A/R_\al(A))) \sseq R(B/R_\al(B)).$$
Writing $q_A:A \onto A/R_\al(A)$ and $q_B:B \onto B/R_\al(B)$, we have
\begin{eqnarray*}
R_{\al+1}(B) & = & q_B^{-1}(R(B/R_\al(B)))\\
& \supseteq & q_B^{-1}(\psi(R(A/R_\al(A))))\\
%& = & q_B^{-1}(\psi(q_A(R_{\al+1}(A))))\\
& = & q_B^{-1}(\{\psi(x + R_\al(A)): x \in R_{\al+1}(A)\})\\
& = & \{\phi(x) + y: x \in R_{\al+1}(A), y \in R_\al(B)\}.
\end{eqnarray*}
Therefore $\phi(R_{\al+1}(A)) \sseq R_{\al+1}(B)$.
%-----------------------------------
\item
Suppose~$I$ is a closed ideal of the Banach algebra~$A$.  Then $(4)_\al$ implies
that $R_\al(I)$ is a closed ideal of~$A$ and $R_\al(I) \sseq R_\al(A) \cap I$.
Now $I/R_\al(I)$ is a closed ideal of $A/R_\al(I)$, so
$R(I/R_\al(I))$ is a closed ideal of $A/R_\al(I)$ by $(4)_1$(a).   By definition,
$$R_{\al+1}(I) = q_I^{-1}\left(R(I/R_\al(I))\right),$$
where $q_I:A \onto A/R_\al(I)$ is the quotient map.   Therefore
$R_{\al+1}(I)$ is a closed ideal of $q_I^{-1}\left(A/R_\al(I)\right) = A$.

By $(4)_1$(b),
$$R(I/R_\al(I)) \sseq R(A/R_\al(I)) \cap I/R_\al(I).$$
Let us write $q_A$ for the quotient map $A \onto A/R_\al(A)$.  Then
the fact that $R_\al(I) \sseq R_\al(A)$ produces a natural map
$p:A/R_\al(I) \onto A/R_\al(A)$.   Applying $(3)_1$ to~$p$ gives
$$R(A/R_\al(A)) \supseteq p(R(A/R_\al(I))) = q_Aq_I^{-1}(R(A/R_\al(I))),$$
whence
$$R_{\al+1}(A) = q_A^{-1}(R(A/R_\al(A))) \supseteq q_I^{-1}(R(A/R_\al(I)))
									\supseteq R_{\al+1}(I).$$
\end{arbenumerate}
(It is in proving $(4)_{\al+1}$ that a theory which allows non-closed ideals
in (4) has problems.    If~$I$ is not closed, then we cannot guarantee that
$R_\al(I)$ is closed in~$A$, without which $A/R_\al(I)$ is not a normed algebra
in a quotient norm.)
%-----------------------------------------------------------------------------

The induction step at limit ordinals is easier and is therefore omitted.
\Detail{ Suppose that $(1)_\al$,
$(3)_\al$ and $(4)_\al$ hold for all ordinals $\al < \lm$, where~$\lm$ is a
limit ordinal.   We show that $R_\lm$ satisfies (1), (3) and (4).
\begin{arbenumerate}
\item
We have
$$R_\lm(R_\lm(A)) =
\closure{\bigcup_{\al<\lm} R_\al(R_\lm(A))}.$$
Now each $R_\al(A)$ is a closed ideal of~$A$ and so of $R_\lm(A)$.  Therefore
$(4)_\al$ implies
$$R_\al(R_\al(A)) \sseq R_\al\left(R_\lm(A)\right).$$
Hence, using $(1)_\al$,
$$R_\lm(R_\lm(A)) \supseteq \closure{\bigcup_{\al<\lm} R_\al(R_\al(A))}
= \closure{\bigcup_{\al<\lm} R_\al(A)} = R_\lm(A).$$
%-------------------------------
\addtocounter{arbenum}{1}
\item If $A,B$ are Banach algebras and $\phi:A \onto B$ is a
continuous epimorphism, then, by $(3)_\al$, we have
$\phi(R_\al(A))\sseq R_\al(B)$, for each $\al<\lm$.  Therefore
$$\phi\left(\bigcup_{\al<\lm}R_\al(A)\right)
=     \bigcup_{\al<\lm}\phi\left(R_\al(A)\right)
\sseq \bigcup_{\al<\lm}R_\al(B).$$
Then
$$\phi(R_\lm(A)) = \phi\left(\closure{\bigcup_{\al<\lm}R_\al(A)}\right)
\sseq  \closure{\phi\left(\bigcup_{\al<\lm}R_\al(A)\right)}
\sseq  \closure{\bigcup_{\al<\lm}R_\al(B)} = R_\lm(B).$$
%-------------------------------
\item
If $I$ is a closed ideal of~$A$ then $R_\al(I)$ is a closed ideal of~$A$
for every $\al<\lm$.  Therefore $\bigcup_{\al<\lm}R_\al(I)$ is an ideal
of~$A$ and $R_\lm(I)$ is a closed ideal of~$A$.
Likewise, $R_\al(I) \sseq R_\al(A) \cap I$ for every $\al<\lm$, whence,
on taking unions and closures, $R_\lm(I) \sseq R_\lm(A) \cap I$.
\end{arbenumerate}}

The proof of Theorem \ref{T3} is now completed by observing that each of
properties (1), (3) and (4) for $\Rstar$ follows from the corresponding property
for $R_\al$ by choosing~$\al$ sufficiently large so that $\Rstar(\Gamma) =
R_\al(\Gamma)$ for the two algebras~$\Gamma$ involved; \ie for $\Gamma = A$ and
$\Rstar(A)$ in (1), $\Gamma=A$ and~$B$ in (3) and $\Gamma = I$ and~$A$ in (4).
\end{Proof}

\begin{cor}\label{C3a}
The topological radical $\barstar{R}(A)$ of a Banach algebra~$A$ is the
smallest {\em closed} ideal~$I$ of~$A$ such that $A/I$ is $R$-semisimple.
\end{cor}

\begin{cor}\label{C4}
The Baer radical~$\beta$ gives rise to the topological radical $\barstar{\beta}$
and $\barstar{\beta}(A)$, for a Banach algebra~$A$, is the smallest closed
ideal~$I$ of~$A$ such that $A/I$ contains no non-zero nilpotent ideals.
\end{cor}

(We recall that the Baer radical $\beta(A)$ can be characterized as the smallest
ideal~$I$ of~$A$ such that $A/I$ contains no non-zero nilpotent ideals.)

\Detail{In one direction, $\barstar{\beta}(A)$ is a closed ideal such that
$\closure{\beta}(A/\barstar{\beta}(A)) = \{0\}$; {\it id est}
$A/\barstar{\beta}(A)$ contains no non-zero nilpotent ideals.   Conversely,
let $I$ be a closed ideal such that $A/I$ contains no non-zero nilpotent ideals.
Then $I \supseteq \beta(A)$, since $\beta(A)$ is the smallest ideal the quotient
by which contains no non-zero nilpotent ideals.  Since~$I$ is closed,
$I \supseteq \closure{\beta}(A)$.    Now
$$(A/\closure{\beta(A)})/(I/\closure{\beta(A)}) \equiv A/I$$
contains no non-zero nilpotent ideals, so
$$I/\closure{\beta(A)} \supseteq \beta\left(A/\closure{\beta(A)}\right)$$
so, since $I/\closure{\beta(A)}$ is closed,
$$I/\closure{\beta(A)} \supseteq \closure{\beta}\left(A/\closure{\beta(A)}\right)
                               = \closure{\beta}_2(A)/\closure{\beta(A)}),$$
so $I \supseteq \closure{\beta}_2(A)$.   The proof that
$I \supseteq \closure{\beta}_\alpha(A)$ for all ordinals~$\alpha$ proceeds by
induction in the same way, the limit ordinal step being trivial.   Therefore
$I \supseteq \barstar{\beta}(A)$.     The proof of the previous corollary is
similar.}

\begin{Defn}
Let $R$ be a map associating with each (Banach) algebra~$A$, a (closed)
ideal $R(A)$.   We define a transfinite sequence $(R^\al)$ by:
\begin{rmenumerate}
\item $R^0(A) = A$;
\item $R^{\al+1}(A) = R(R^\al(A))$,    (so, for example, $R^1 = R$);
\item for limit ordinals~$\lm$, $$R^\lm(A) = \bigcap_{\al < \lm}R^\al(A).$$
\end{rmenumerate}
The transfinite sequence of sets $(R^\al(A))$ is monotonic non-increasing, so it
must stabilise at the $\al$th stage, where~$\al$ is at most the cardinality
of~$A$.  We then write $\Rsubstar(A) = R^\al(A) = R^{\al+1}(A)$.
\end{Defn}


\begin{thm}\label{T4}
If $R$ is an OR (respectively, an OTR), then so is $R^\al$,
for every ordinal~$\al$, and $\Rsubstar$ is a radical (topological radical).
\end{thm}

\begin{Proof}
Again, we prove the topological case, the algebraic case being similar and
easier.   First we observe that (4) implies that all of the $R_\al(A)$ are
closed ideals  of~$A$ and hence so is $\Rsubstar(A)$.

The fact that $\Rsubstar$ satisfies condition (1) is easy: because the sequence
has stabilised,
$$\Rsubstar(A) = R(\Rsubstar(A)),$$
It follows that $R^\al(\Rsubstar(A)) = \Rsubstar(A)$
for all~$\al$, and so $\Rsubstar(\Rsubstar(A)) = \Rsubstar(A)$.

The rest of the proof consists of showing by transfinite induction on~$\al$,
that $R^\al$ satisfies conditions (2), (3) and (4).    The result is trivially
true for $\al = 0$, which starts the induction.   Suppose $R^\al$ is an OTR.
Again, we write $(n)_\al$ to mean condition $(n)$ on $R^\al$.
\begin{arbenumerate}
%-------------------------------
\addtocounter{arbenum}{1}
\item
Since $R^{\al+1}(A) \sseq R^\al(A)$, there is a natural continuous homomorphism
$$A/R^{\al+1}(A) \onto A/R^\al(A).$$   Applying $(3)_\al$ to this map shows that
$R^\al(A/R^{\al+1}(A))$ maps into $R^\al(A/R^\al(A))$, which is the zero coset,
by $(2)_\al$.  Therefore $R^\al(A/R^{\al+1}(A))\sseq R^\al(A)/R^{\al+1}(A)$,
in fact, $R^\al(A/R^{\al+1}(A))$ is a closed ideal of $R^\al(A)/R^{\al+1}(A)$.
Applying $(4)_1$ to this ideal,
$$R^{\al+1}(A/R^{\al+1}(A)) = R(R^\al(A/R^{\al+1}(A))) \sseq R(R^\al(A)/R^{\al+1}(A))
                                                             = \{0\},$$
where the last step uses $(2)_1$.
%-------------------------------
\item
Suppose $\phi:A \onto B$ is a continuous epimorphism between
Banach algebras.    By $(3)_\al$, we have $\phi(R^\al(A)) \sseq R^\al(B)$.
The argument in the algebraic case, continues with
$$\phi(R^{\al+1}(A)) = \phi(R(R^\al(A))) \sseq R(\phi(R^\al(A)))
                                                  \sseq R(R^\al(B)),$$
but this fails in the topological case because, for the last step, we need
$\phi(R^\al(A))$ to be a {\em closed} ideal of $R^\al(B)$ to apply $(4)_1$.
Instead, we argue that $\closure{\phi(R^\al(A))}$ is a closed ideal of
$R^\al(B)$, so, by $(4)_1$,
$$R\left(\closure{\phi(R^\al(A))}\right)\sseq R(R^\al(B)) = R^{\al+1}(B).$$
Now let
$$I = \phi^{-1}\left(\closure{\phi(R^\al(A))}\right) \supseteq R^\al(A).$$
Then $R^\al(A)$ is a closed ideal of~$I$, so $R^{\al+1}(A) = R(R^\al(A))
\sseq R(I)$, by $(4)_1$.  Thus, by applying $(3)_1$ to the mapping
$\phi:I \onto \closure{\phi(R^\al(A))}$, we obtain
$$\phi(R^{\al+1}(A)) \sseq \phi(R(I)) \sseq
  R\left(\closure{\phi(R^\al(A))}\right)\sseq R^{\al+1}(B).$$
%-----------------------------------
\item
Suppose~$I$ is a closed ideal of the Banach algebra~$A$.  Then, first, $(4)_\al$(a)
implies that $R^\al(I)$ is a closed ideal of~$A$ and so, applying $(4)_1$(a)
to this ideal, we see that $R^{\al+1}(I) = R(R^\al(I))$ is a closed ideal of $A$.
Secondly, $(4)_\al$(b) implies that
$R^\al(I) \sseq R^\al(A)$, so $R^\al(I)$ is a closed ideal of $R^\al(A)$, and
$(4)_1$(b) applied to this ideal gives
$$R^{\al+1}(I) = R(R^\al(I))\sseq R(R^\al(A)) = R^{\al+1}(A).$$
%-----------------------------------------------------------------------------
\end{arbenumerate}

Again, we omit the induction step at limit ordinals, which is straightforward.
\Detail{Suppose that $(2)_\al$,
$(3)_\al$ and $(4)_\al$ hold for all ordinals $\al < \lm$, where~$\lm$ is a
limit ordinal.   We show that $R^\lm$ satisfies (2), (3) and (4).
\begin{arbenumerate}
\addtocounter{arbenum}{1}
\item
For each $\al<\lm$ we have a natural continuous homomorphism
$A/R^\lm(A) \onto A/R^\al(A)$.  Applying $(3)_\al$ shows that
$R^\al(A/R^\lm(A))$ is mapped into $R^\al(A/R^\al(A))$, which is $(0)$,
by $(2)_\al$; \ie $R^\al(A/R^\lm(A))\sseq R^\al(A)/R^\lm(A)$.  Hence
$$R^\lm(A/R^\lm(A)) = \bigcap_{\al<\lm}R^\al(A/R^\lm(A))
    \sseq \bigcap_{\al<\lm}R^\al(A)/R^\lm(A) = R^\lm(A)/R^\lm(A) = (0).$$
%-------------------------------
\item If $A,B$ are Banach algebras and $\phi:A \onto B$ is a
continuous epimorphism, then, by $(3)_\al$, we have
$\phi(R^\al(A))\sseq R^\al(B)$, for each $\al<\lm$.  Therefore
$$\phi(R^\lm(A)) = \phi\left(\bigcap_{\al<\lm}R^\al(A)\right)
             \sseq \bigcap_{\al<\lm}\phi(R^\al(A))
             \sseq \bigcap_{\al<\lm}R^\al(B) = R^\lm(B).$$
%-------------------------------
\item
If $I$ is a closed ideal of~$A$ then $R^\al(I)$ is a closed ideal of~$A$ for
every $\al<\lm$.  Therefore $R^\lm(I) =\bigcap_{\al<\lm}R^\al(I)$ is a closed
ideal of~$A$.  Likewise, $R^\al(I) \sseq R^\al(A) \cap I$ for every $\al<\lm$,
whence, on taking intersections, $R^\lm(I) \sseq R^\lm(A) \cap I$.
\end{arbenumerate}}
The proof is completed by observing that each of
properties (1), (3) and (4) for $\Rsubstar$ follows from the corresponding property
for $R^\al$ by choosing~$\al$ sufficiently large so that $\Rsubstar(\Gamma) =
R^\al(\Gamma)$ for the two algebras~$\Gamma$ involved.
%\ie for $\Gamma = A$ and $A/\Rsubstar(A)$ in (1), $\Gamma=A$ and~$B$ in (3) and
%$\Gamma = I$ and~$A$ in (4).
\end{Proof}

\begin{thm}\label{T4.1}
Let~$R$ be a UR (UTR) and~$S$ an OR (OTR), with $R \le S$.
Then $\Rstar \le S_*$.   In particular, if~$S$ is a radical (topological radical),
then $\Rstar \le S$; if~$R$ is a radical (topological radical), then $R \le S_*$.
If~$R$ is a radical and~$S$ is a topological radical, with $R\le S$, then
$\barstar R \le S$.
\end{thm}

\begin{Proof}
The proof consists of three steps.
\begin{alenumerate}
\item
We first show, by transfinite induction on~$\al$, that $R_\al \le S$.   This
is trivial for $\al=0$ and the step to limit ordinals is easy.   For the
successor step, suppose $R_\al \le S$. Then
$$R_{\al+1}(A)/R_\al(A) = R\left(A/R_\al(A)\right) \sseq S\left(A/R_\al(A)\right).$$
Since $R_\al(A) \sseq S(A)$, we have a natural (continuous) epimorphism of
$A/R_\al(A)$ onto $A/S(A)$.   This maps $S\left(A/R_\al(A)\right)$ into
$S\left(A/S(A)\right) = \{0\}$.  Therefore
$$S\left(A/R_\al(A)\right) \sseq S(A)/R_\al(A),$$
so
$$R_{\al+1}(A)/R_\al(A) \sseq S(A)/R_\al(A),$$
so $R_{\al+1}(A)\sseq S(A)$.
\item
Next, we show that $R \le S^\bt$ for all~$\bt$.   This is easier, the
successor step being that if $R \le S^\bt$ then $R(A)$ is a (closed) ideal
in $S^\bt(A)$, so
$$R(A) = R(R(A)) \sseq S(R(A)) \sseq S(S^\bt(A)) = S^{\bt+1}(A).$$
\item
Since every $R_\al$ is UR (UTR), we can apply step (b)
to $R_\al$ in place of~$R$ to get $R_\al(A) \le S^\bt(A)$ for all ordinals
$\al,\bt$ and all (Banach) algebras~$A$.  Hence $\Rstar(A) \sseq S_*(A)$.
\end{alenumerate}
\end{Proof}

\begin{cor}\label{C4.1}
If~$R$ is a UR (UTR), then~$\Rstar$ is the smallest (topological) radical greater
than or equal to~$R$: \ie $\Rstar \ge R$ and if $S$ is a (topological) radical
with $S \ge R$, then $\Rstar \le S$.    Likewise, if~$S$ is an OR (OTR), then
$S_*$ is the greatest (topological) radical less than or equal to~$S$.
\end{cor}
%==============================================================================
\section{Hereditary radicals}\label{S5a}

It is natural to ask whether, in Theorem \ref{T2}, the map~$R$ satisfying
axiom~(5) would imply $\Rbar$ satisfying~(5).    The answer is negative,
as the following example shows.

\begin{Example}\label{Ex5}
Let~$A$ be the commutative Banach algebra generated by $\{X_n: n=1,2,3,\dots\}$
subject to the relations $X_n^{n+1}=0$ $(n=1,2,3,\dots)$ and $X_iX_j=0$
$(i \ne j)$.  That is, if $A_0$ is the algebra defined, algebraically, by
these generators and relations, with the norm given by
$$\left\|\sum_{1\le j\le i} \lambda_{ij}X_i^j\right\|
                                    = \sum_{1\le j\le i} |\lambda_{ij}|,$$
then~$A$ is the completion of $(A_0,\|.\|)$.  A typical element of~$A$ is just
an infinite sum $x = \sum_{1\le j\le i} \lambda_{ij}X_i^j$ with
$\|x\| = \sum_{1\le j\le i} |\lambda_{ij}| < \infty$.

Let $y = \sum_{i=1}^\infty 2^{-i}X_i$ and let~$B$ be the set of elements
of~$A$ of the form $\sum_{2\le j\le i} \lambda_{ij}X_i^j$.    Then
$I = \C y + B$ is a closed ideal of~$A$.   Now an element
$\sum_{1\le j\le i} \lambda_{ij}X_i^j$ of~$A$ is nilpotent if and only if
$\sup\{i/j:\lambda_{ij} \ne 0\} < \infty$ and, since~$A$ is commutative,
$\beta(A)$ and $\beta(I)$ are just the sets of nilpotent elements of~$A$ and~$I$,
respectively.  Thus $\overline{\beta}(A) = A$ and $\overline{\beta}(I) = B$;
so $\overline{\beta}(A) \cap I \not\sseq \overline{\beta}(I)$.
\end{Example}

\begin{Remark}
In this example (5) seems to fail in a rather trivial way.  Indeed, for
every hereditary radical~$R$, if~$I$ is a closed ideal in a Banach
algebra~$A$,  we have $(\Rbar(A) \cap I)^2 \sseq \Rbar(I)$.  One consequence of this
is that if $R \ge \beta$ then $\Rbar(A) \cap I \sseq \Rbar_2(I)$, where
$\Rbar_2$ is constructed from $\Rbar$ as in Definition \ref{D3}.
\Detail{Let $x,y \in \Rbar(A) \cap I$.  There exist $x_n \in R(A)$ $(n =
1,2,3,\dots)$ with $x_n \to x$.   Then $x_ny \in R(A)I \sseq R(A) \cap I \sseq
R(I)$, because~$R$ is hereditary.  So $xy \in \Rbar(I)$, whence the result.

The second statement follows because the condition $R \ge \beta$ is equivalent
to saying that every nilpotent ideal of an algebra~$A$ is contained in $R(A)$;
a property sometimes labelled {\it supernilpotent}.}
It is tempting to conjecture that this will form the start of a
transfinite induction leading to $\barstar{R}$ being hereditary, but our
attempts to carry out this plan have been thwarted by that perennial problem of
Banach algebra theory: the fact that the sum of two closed ideals is not
necessarily closed.  We do not even know whether or not $\barstar{\beta}$ is
hereditary.
\end{Remark}



%==============================================================================
\section{The TT radicals}\label{S6}


\begin{thm}\label{T5}
For $n = 1,2,\dots,\infty$, the map $A \to T_n(A)$ which associates with
every Banach algebra~$A$ its topologically $n$-transitive radical is a
hereditary topological radical.
\end{thm}

\begin{Proof}
\begin{arbenumerate}
\item follows from (5) below and (2) and (3) are straightforward.
\Detail{For (2) we observe that
every continuous $n$-TT representation $\pi:A \to \LX$ induces a continuous
$n$-TT representation $\tilde\pi:A/T_n(A) \to \LX$ with kernel $(\ker\pi)/T_n(A)$.
Therefore the intersection of all the kernels of continuous $n$-TT representations
of $A/T_n(A)$ is zero.

To show (3), we argue that if $\phi:A \onto B$ is a continuous epimorphism, then
every continuous $n$-TT representation $\pi:B \to \LX$ induces a continuous
representation $\pi\phi:A \to \LX$, which is $n$-TT (because~$\phi$ is
surjective).  The intersection of the kernels of these representations of~$A$ is
$\phi^{-1}(T_n(B))$.   Thus $T_n(A)\sseq\phi^{-1}(T_n(B))$, whence
$\phi(T_n(A))\sseq T_n(B)$.}
\addtocounter{arbenum}{2}
%--------------------------------------
\item(a)
We must show that if~$I$ is a closed ideal of~$A$, then  $T_n(I)$ is an ideal
of~$A$.   Suppose $a\in A$ and $b\in T_n(I)$; then $\pi(b) = 0$ for every
continuous $n$-TT representation~$\pi$ of~$I$ and we need to show that
$\pi(ab) = 0$ and $\pi(ba) = 0$ for all such representations.

Consider $\pi(ab)$ where $\pi:I \to \LX$ is $n$-TT.  If $x \in X$ and $c \in I$
then
$$\pi(c)\pi(ab)x = \pi(cab)x = \pi(ca)\pi(b)x = 0.$$
This, for all $c\in I$, implies $\pi(ab)x=0$, because~$\pi$ is TI.   Hence
$\pi(ab) = 0$.  Likewise, $\pi(ba)\pi(c)x = \pi(b)\pi(ac)x = 0$ and
$\{\pi(c)x: c \in I,\; x\in X\}$ is dense in~$X$, so $\pi(ba)=0$.
%--------------------------------------
\addtocounter{arbenum}{-1}
\item(b)
Every continuous representation~$\pi$ of~$A$ restricts to a continuous
representation of a closed ideal~$I$.  It suffices to show that, for $n < \infty$,
if~$\pi$ is $n$-TT then $\pi|I$ is $n$-TT or zero.    It will then follow that
$T_n(I) \sseq T_n(A)$ for all~$n$. \detail{If $a \in I\setminus T_n(A)$, then
there is a continuous $n$-TT representation~$\pi$ such that $\pi(a) \neq 0$, so
$\pi|I$ is a continuous $n$-TT representation of~$I$ with $(\pi|I)(a) \neq 0$,
so $a \nin T_n(I)$.}

We begin by showing that if $\pi|I$ is non-zero, then it is TI.  To see this,
suppose $x,y \in X$ with $x \neq 0$ and $\e>0$; let~$b$ be any element of~$I$
such that $\pi(b)x \neq 0$.  (If no such~$b$ exists, then $\pi(I)(\pi(A)x) \sseq
\pi(IA)x \sseq \pi(I)x = \{0\}$, so $\pi(I)X = 0$, contrary to assumption.)   We
may then find $a \in A$ such that $\|\pi(a)(\pi(b)x) - y\| < \e$, so we have
$\|\pi(ab)x - y\| < \e$ and $ab \in I$.

We now show that $\pi|I$ is $n$-TT.  Let
$x_1,\dots,x_n, y_1,\dots,y_n \in X$ with $x_1,\dots,x_n$ linearly independent
and $\e > 0$.  Choose $\eta > 0$ such that every set $\{z_1,\dots,z_n\}$ with
$\|z_i - x_i\| < \eta$ $(1 \le i \le n)$ is linearly independent.  Since~$\pi$
restricts to a TI representation on~$I$, by the above, we can find, for each~$i$
an element $b_i \in I$ such that $\|\pi(b_i)x_i - x_i\| < \eta/2n$.    We then
find $c_i \in A$ $(1 \le i \le n)$ such that
$$\|\pi(c_i)x_j - \delta_{ij}x_i\| < \eta/2n\|\pi(b_i)\| \qquad (1 \le i,j \le n).$$
Then
$$\|\pi(b_ic_i)x_j - \delta_{ij}\pi(b_i)x_i\| < \eta/2n \qquad (1 \le i,j \le n).$$
So
$$\|\pi(b_ic_i)x_j - \delta_{ij}x_i\| < \eta/n \qquad (1 \le i,j \le n).$$
Let
$$b = \sum_{i=1}^n b_ic_i \in I.$$
Then
$$\|\pi(b)x_j - x_j\| < \eta \qquad (1 \le j \le n).$$
Therefore the set $\{\pi(b)x_1,\dots,\pi(b)x_n\}$ is linearly independent.
Let $a \in A$ be such that
$$\|\pi(a)\pi(b)x_j - y_j\| < \e$$
and $ab \in I$ is the desired element such that $\pi(ab)$ sends $x_j$ near to
$y_j$ for all~$j$.
%-------------------------------------------------------------------------
\item The fact that, for every closed ideal~$I$ of a Banach algebra, $T_n(I)
\supseteq T_n(A) \cap I$ follows from Lemma \ref{L1} below.
\end{arbenumerate}
\end{Proof}

\begin{lemma}\label{L1}
If~$I$ is a closed ideal of a Banach algebra~$A$ and $1 \le n \le \infty$,
then for every continuous $n$-TT representation~$\pi$ of~$I$ on a Banach space
$(X,\|.\|)$ there is a continuous $n$-TT representation~$\rho$ of~$A$ on a
Banach space $(W,\nnY{.})$ such that $\ker\pi = \ker\rho \cap I$.
\end{lemma}

\begin{PoL}
Let $\pi:I\to \LX$ be a continuous TI representation of the ideal~$I$ on a
Banach space~$X$.   We shall describe the construction of a continuous
representation $\rho:A\to\LW$ and then show that, for every $k < \infty$,
if~$\pi$ is $k$-TT, then so is~$\rho$.

Let
$$Y = \pi(I)X = \left\{\sum_{i=1}^n \pi(b_i)x_i:
                    b_i\in I,\; x_i\in X \;(1 \le i \le n),\; n=1,2,3,\dots\right\}$$
with norm
$$\nnY{y} = \inf\left\{\sum_{i=1}^n \|b_i\|\,\|x_i\|:
           y = \sum_{i=1}^n \pi(b_i)x_i  \mbox{ as above}\right\}.$$
Let~$Z$ be the completion of $(Y,\nnY{.})$. Then the inclusion map $Y \to X$ is
continuous of norm at most $\|\pi\|$ and so extends to a continuous map
$\theta:Z \to X$.  Let $(W,\nnY{.}) = (Z,\nnY{.})/\ker\theta$ and denote
by~$\tilde\theta:W \to X$ the injective map induced by~$\theta$.

We wish to define $\pi_1:A \to \LY$ by
$$\pi_1(a)\left(\sum_{i=1}^n \pi(b_i)x_i\right) = \sum_{i=1}^n \pi(ab_i)x_i
                  \qquad(b_i\in I,\; x_i\in X \;(1 \le i \le n),\; n=1,2,3,\dots),$$
but we first need to show that this is well defined.   If
$$\sum_{i=1}^n \pi(b_i)x_i = \sum_{j=1}^m \pi(b_j')x_j',$$
then, for all $c \in I$,
\begin{eqnarray*}
\pi(c)\left(\sum_{i=1}^n \pi(ab_i)x_i\right)
& = & \sum_{i=1}^n\pi(cab_i)x_i \\
& = & \pi(ca)\sum_{i=1}^n\pi(b_i)x_i \\
& = & \pi(ca)\sum_{j=1}^m\pi(b_j')x_j' \\
& = & \sum_{j=1}^m\pi(cab_j')x_j' \\
& = & \pi(c)\left(\sum_{j=1}^m \pi(ab_j')x_j'\right).
\end{eqnarray*}
Because~$\pi$ is TI, it follows that
$$\sum_{i=1}^n \pi(ab_i)x_i = \sum_{j=1}^m \pi(ab_j')x_j'.$$

To prove that $\pi_1(a) \in \LY$ for all $a\in A$ and that~$\pi_1$ is
continuous, we note that, for every presentation
$$y = \sum_{i=1}^n \pi(b_i)x_i$$
of a given element $y \in Y$ we have
\begin{eqnarray*}
\nnY{\pi_1(a)y}
& = & \bignnY{\sum_{i=1}^n \pi(ab_i)x_i} \\
&\le& \sum_{i=1}^n \|ab_i\|\,\|x_i\| \\
&\le& \|a\|\sum_{i=1}^n \|b_i\|\,\|x_i\|.
\end{eqnarray*}
It follows that
$$\nnY{\pi_1(a)y} \le \|a\|\,\nnY{y}.$$

We can now extend each $\pi_1(a) \in \LY$ to $\pi_2(a) \in \LZ$ by
continuity, thus defining a continuous representation $\pi_2:A \to \LZ$.
For $b \in I$, and $y = \sum_{i=1}^n \pi(b_i)x_i \in Y$ we have
$$\pi_1(b)(y) = \sum_{i=1}^n \pi(bb_i)x_i
                  = \sum_{i=1}^n \pi(b)\pi(b_i)x_i
                  = \pi(b)y;$$
\ie
$$\pi_2(b)(y)   = \pi(b)(\theta(y)).$$
By continuity (\ie $\pi_2(b) \in \LZ$, $\pi(b) \in \LX$ and
$\theta:Z \to X$ being continuous) and the fact that~$Y$ is dense in~$Z$,
it follows that
\begin{equation} \label{E1}
\pi_2(b)(z) = \pi(b)(\theta(z)) \qquad (z \in Z,\; b \in I).
\end{equation}
Let $a \in A$, $z \in Z$; then for all $c \in I$  we have
$$\pi(c)\theta(\pi_2(a)z) = \pi_2(c)\pi_2(a)z = \pi_2(ca)z = \pi(ca)\theta(z).$$
Therefore, if $\theta(z)=0$ then $\pi(c)\theta(\pi_2(a)z) = 0$ for all $c \in I$
and so, because~$\pi$ is TI on~$X$, we have $\theta(\pi_2(a)z) = 0$.   Thus, for
each $a \in A$, there is a well-defined mapping $\rho(a):W \to W$ such that
$$\rho(a)(z + \ker\theta) = \pi_2(a)z + \ker\theta.$$
Clearly~$\rho$ is a continuous representation of~$A$ on the Banach space~$W$.
Equation (\ref{E1}) implies that
\begin{equation} \label{E3}
\rho(b)(w) = \pi(b)(\tilde\theta(w)) \qquad (w \in W,\; b \in I).
\end{equation}

Now suppose that~$\pi$ is $k$-TT with $k < \infty$.
We show that~$\rho$ is $k$-TT.  Since~$Y/\ker\theta$ is dense in~$W$, it
suffices to show that for every linearly independent
$w^{(1)},\dots,w^{(k)} \in W$, every $y^{(1)},\dots,y^{(k)} \in Y/\ker\theta$
and $\e > 0$ there is an $a \in A$ with
$\nnY{\rho(a)w^{(j)}- y^{(j)}} < \e$ $(1\le j\le k)$.    We shall, in fact,
show that this can be done with an $a \in I$: thus (by (\ref{E3})) we shall
show
$$\nnY{\pi(a)x^{(j)} - y^{(j)}} < \e\qquad(1\le j\le k),$$
where $x^{(j)} = \tilde\theta(w^{(j)})$.   Since~$\tilde\theta$ is injective,
the set $\{x^{(1)},\dots,x^{(k)}\}$ is linearly independent.

We can write $y^{(j)} = \sum_{i=1}^n \pi(b_i)x_i^{(j)}$ $(1\le j\le k)$, with
the $b_i\in I$,
$x_i^{(j)}\in X$, by the simple expedient of making the sets $N_j =
\{i:x_i^{(j)}\ne 0\}$ disjoint.  Because the original representation~$\pi$ is
$k$-TT, we can find $c_i \in I$ $(1 \le i \le n)$ so that
$$\|\pi(c_i)x^{(j)} - x_i^{(j)}\| < {\e\over n\|b_i\|}
                                \qquad (1 \le i \le n,\;1\le j\le k).$$
Let $a = \sum_{i=1}^n b_ic_i$.
Then
\begin{eqnarray*}
\nnY{\pi(a)x^{(j)} - y^{(j)}}
& = & \bignnY{\sum_{i=1}^n \pi(b_ic_i)x^{(j)} - \sum_{i=1}^n \pi(b_i)x_i^{(j)}}\\
& = & \bignnY{\sum_{i=1}^n \pi(b_i)\left(\pi(c_i)x^{(j)} - x_i^{(j)}\right)} \\
& \le & \sum_{i=1}^n \|b_i\|\,\|\pi(c_i)x^{(j)} - x_i^{(j)}\|, \qquad \mbox{by the definition of $\nnY{.}$,}\\
& < & \e.
\end{eqnarray*}

Finally, equation (\ref{E3}), together with the fact that $\tilde\theta(W) =
\theta(Z)$ is dense in~$X$ shows that if $b \in I$, then $\rho(b) = 0$ if and
only if $\pi(b) = 0$.  Thus $\ker\pi = \ker\rho \cap I$.
\end{PoL}
%==============================================================================
\section{Standard TI representations}\label{S7}

Let us recall that a standard TI representation of a normed algebra~$A$
is a representation~$\pi$ of~$A$ on a Banach space~$X$ which is of the
form $\pi = \phi\rho$ where~$\phi$ is a continuous homomorphism of~$A$ onto
a dense subalgebra of a Banach algebra~$B$ and~$\rho$ is a strictly irreducible
representation of~$B$ on~$X$.

Notice that there is no point in broadening this definition by dropping the
completeness requirements on either~$X$ or~$B$, separately.   We have already
remarked that if~$X$ were incomplete, or just a general linear space then it
would automatically have a suitable Banach space structure.   If~$B$ were
incomplete, we should require that~$\rho$ be continuous, in order to
make~$\pi$ continuous.  This being so, we need only complete~$B$ and
extend~$\rho$ to the completion to regain our original scenario.

Since~$\rho$ may be factored through the primitive algebra $B/\ker\rho$, we can
characterize $S(A)$ as the intersection of the kernels of the continuous
homomorphisms of~$A$ onto dense subalgebras of primitive (or semisimple) Banach
algebras.

\begin{thm} \label{T8}
The map $A \to S(A)$, defined for Banach algebras~$A$, is an OTR.
Hence $S_*$ is a topological radical.
\end{thm}

\begin{Proof}
Notation: throughout this proof,~$B$ will be an arbitrary semisimple Banach
algebra and~$\phi$ mapping into~$B$ will be a continuous homomorphism with dense
range.   We shall use the characterisation of $S(A)$ as the intersection of the
kernels of such mappings $\phi:A \to B$.
\begin{arbenumerate}
\addtocounter{arbenum}{1}
\item
Every $\phi:A\to B$ induces a continuous homomorphism $\tilde\phi:A/S(A)\to B$
with the same range.   The intersection of the kernels of these induced
homomorphisms is zero, so $S(A/S(A)) = \{0\}$.
%-----------------------------------------------------------------------------
\item
If $A_1$, $A_2$ are Banach algebras and $\psi:A_1\to A_2$ is a continuous
epimorphism, then every $\phi:A_2\to B$ gives rise to a continuous
homomorphism $\phi\psi:A_1\to B$ with dense range.  The fact that
$\psi(S(A_1)) \sseq S(A_2)$ follows immediately.
%-----------------------------------------------------------------------------
\item(a)
Let~$I$ be a closed ideal of~$A$ and let $a \in A$, $b \in S(I)$; we show that
$ab \in S(I)$, the case of~$ba$ being similar.  Given $\phi:I \to B$ as above,
we have $\phi(b) = 0$.   For arbitrary $c \in I$ we have
$$\phi(c)\phi(ab) = \phi(cab) = \phi(ca)\phi(b) = 0.$$
Since $\phi(I)$ is dense in~$B$, it follows that $B\phi(ab) = \{0\}$ and
hence, since~$B$ is semisimple, that $\phi(ab) = 0$.
%--------------------------------------
\addtocounter{arbenum}{-1}
\item(b)
Let~$I$ be a closed ideal of~$A$.   Let $\phi:A\to B$ as usual.  Now $I \idealin
A$ implies $\phi(I) \idealin B$, (because $\phi(A)$ is dense in~$B$).  Therefore
$\closure{\phi(I)} \idealin B$.    Therefore $\closure{\phi(I)}$ is semisimple; (the
Jacobson radical is hereditary).  Thus $\phi:I \to \closure{\phi(I)}$ is a
continuous homomorphism into a semisimple Banach algebra with dense range.
The fact that $S(I) \sseq S(A) \cap I$ follows immediately.
\end{arbenumerate}

\end{Proof}


We can summarize the relationships between our various topological radicals as
follows.

\begin{thm}\label{T12}
The topological radicals $S_*$ and $T_n$ $(1 \le n \le \infty)$
are related to the closed-Baer radical $\barstar\beta$  and the Jacobson
radical~$J$ by the inequalities
$$\barstar\beta \le T_1 \le T_m \le T_n \le T_\infty \le S_* \le J
                                       \qquad (1 \le m \le n \le \infty).$$
\end{thm}

\begin{Proof}
The inequality $\overline\beta \le T_1$ is Proposition \ref{P1} and the other
inequalities are trivial.   The result follows by Theorem~\ref{T4.1}.
\end{Proof}

\vspace{2ex}
The main result of this section is that the last of these inequalities is
proper.

\begin{Example}\label{MainEx}
We construct an example of a radical Banach algebra~$A$ which has an
injective, dense embedding~$\phi$ into a semisimple Banach algebra~$B$.  Hence,
$S(A) = \{0\}$.

The algebra~$B$ is the algebra regrettably called~$A$ in \cite{PGD12}.   Let $A_0$
be the algebra on symbols $X_1, X_2, \dots$ subject to the following relations:
every monomial $X_{i_1}\dots X_{i_r}$ containing more than~$n$ occurrences of
$X_n$, where $n = \max\{i_1,\dots,i_r\}$, must vanish.   It follows that
$X_{i_1}\dots X_{i_r} = 0$ if $r \ge (n+1)!$, where $n = \max\{i_1,\dots,i_r\}$.
This algebra is given a norm $\|.\|$ by
$$\left\|\sum_{i=0}^n \lambda_iM_i\right\| = \sum_{i=0}^n |\lambda_i|,$$
where the $\lambda_i$ are scalars and the $M_i$ monomials.   The Banach
algebra~$B$ is the completion of $(A_0,\|.\|)$.

We shall construct the radical Banach algebra~$A$ as the completion of
$A_0$ in a larger norm, so that there is a natural continuous embedding
of~$A$ into~$B$, whose range contains $A_0$ and is therefore dense.

For each monomial $M = X_{i_1}\dots X_{i_r}$, we define
$n(M) = \max\{i_1,\dots,i_r\}$ and \linebreak $\nn{M} = ((n+1)!)^{(n+1)!}$.
Consider a product of $2k$ monomials $M_1\dots M_{2k}$, where $k \ge 1$.
We distinguish three cases.
\begin{alenumerate}
\item If $(n(M_i)+1)! \le k$ $(1\le i\le k)$, then $M_1\dots M_k = 0$.
\item If $(n(M_i)+1)! \le k$ $(k+1\le i\le 2k)$, then
$M_{k+1}\dots M_{2k} = 0$.
\item If neither (a) nor (b) hold, then there are at least two values of~$i$ for
which \linebreak
$(n(M_i)+1)! > k$ and it follows from the definition of $\nn{M_i}$ that
\begin{equation}\label{E2}
\nn{M_1\dots M_{2k}} \le k^{-k} \nn{M_1}\dots\nn{M_{2k}}.
\end{equation}
\end{alenumerate}
In all three cases, (\ref{E2}) holds.

We extend the norm to general elements of $A_0$ by defining
$$\bignn{\sum_{i=0}^n \lambda_iM_i} = \sum_{i=0}^n |\lambda_i|\nn{M_i},$$
where the $\lambda_i$ are scalars and the $M_i$ monomials.
We then define~$A$ to be the completion of $A_0$ in this norm, which is
clearly identified with the set of infinite sums
$$\sum_{i=0}^\infty \lambda_iM_i$$ such that
$$\sum_{i=0}^n |\lambda_i|\nn{M_i} < \infty.$$
Hence the natural embedding of $A_0$ into~$B$ extends to a natural
embedding of $A$ into~$B$.

It follows from (\ref{E2}) that
$$\nn{x_1\dots x_{2k}} \le k^{-k} \nn{x_1}\dots\nn{x_{2k}}$$
for all $x_1,\dots,x_{2k} \in A$.   Since
$$\left(k^{-k}\right)^{1/2k} \to 0$$
as $k\to\infty$, we see that~$A$ is topologically nilpotent, and
so {\em a fortiori} radical.
\end{Example}

Notice that the TI representations of~$A$ restrict to TI representations
of the (incomplete) normed algebra $A_0$.    This is interesting, since
$A_0$ is locally nilpotent and hence nil, but not semiprime.   This example
prevents us from extending Proposition \ref{P1} to the Levitzki and nil radicals.

\begin{cor}\label{C11}
There is a locally nilpotent normed algebra with a separating family of
continuous TI representations.
\end{cor}

It would be interesting to know how general this construction can be made,
so as to produce a wide variety of $S_*$-semisimple, Jacobson-radical algebras.
The following theorem is a first step in that direction.

\begin{thm}\label{T9}
Let $(B,\|.\|)$ be a Banach algebra with an increasing family of nilpotent subalgebras
$M_n$ $(n = 1,2,3,\dots)$ such that $M_n^n = \{0\}$.   Then there is a radical
Banach algebra~$(A,|.|)$ and a continuous injective homomorphism $\phi:A \to B$
such that $\bigcup_{n=1}^\infty M_n \sseq \phi(A) \sseq B$.
\end{thm}

\begin{Proof}
Let $M = \bigcup_{n=1}^\infty M_n$.  For $x \in M$ define
$$|x| := \inf\left\{\sum_{i=1}^k \nu(i)\|m_i\|: x = \sum_{i=1}^k m_i,\quad
                                                        m_i \in M_i\right\},$$
where the increasing sequence of positive real numbers $\nu(i)$ will be defined later.

The function $|.|$ is clearly a norm on~$M$.   If $\nu(i) \ge 1$ for all~$i$
then $|.|$ is submultiplicative and $\|x\| \le |x|$ for all $x \in M$.
\Detail{If
$$x = \sum_{i=1}^k m_i \qquad \mbox{and} \qquad y = \sum_{j=1}^\ell m_i'$$
then
$$xy = \sum_{i,j} m_im_j'.$$
Since the $M_n$ are subalgebras, $m_im_j' \in M_{\max\{i,j\}}$, so
$$|xy| \le \sum_{i,j} \nu(\max\{i,j\})\|m_i\|\,\|m_j'\|
       \le \sum_{i,j} \nu(i)\nu(j)\|m_i\|\,\|m_j'\|
		  = |x|\,|y|.$$
If $x \in M$ and $x = \sum_{i=1}^k m_i$ as above, then
$$\|x\| \le \sum_{i=1}^k \|m_i\| \le \sum_{i=1}^k \nu(i)\|m_i\|.$$
Therefore $\|x\| \le |x|$ for all $x \in M$.}
Let~$A$ be the completion of $(M,|.|)$.

Now consider $x^{(1)}, \dots,x^{(2N)} \in M$ with
$$x^{(n)} = \sum_{i_n} m_{i_n}^{(n)}\qquad(1 \le n \le 2N),$$
for some $m_{i_n}^{(n)}\in M_{i_n}$.  Then
$m_{i_1}^{(1)}\dots m_{i_{2N}}^{(2N)} \in M_r$ where
$r = \max\{i_1,\dots,i_{2N}\}$, so
\begin{eqnarray}
\left|x^{(1)}\dots x^{(2N)}\right|
& \le & \sum_{i_1}\dots\sum_{i_{2N}}\max_n\nu(i_n)\|m_{i_1}^{(1)}\dots m_{i_{2N}}^{(2N)}\|\nonumber\\
& \le & \sum_{i_1}\dots\sum_{i_{2N}}{\nu(i_1)\dots\nu(i_{2N})\over\nu(N)}
                                  \|m_{i_1}^{(1)}\dots m_{i_{2N}}^{(2N)}\|\label{keystep}\\
& \le & \sum_{i_1}\dots\sum_{i_{2N}}{\nu(i_1)\dots\nu(i_{2N})\over\nu(N)}
                                     \|m_{i_1}^{(1)}\|\dots\| m_{i_{2N}}^{(2N)}\|, \nonumber
\end{eqnarray}
where (\ref{keystep}) holds because at least two of the $i_j$ exceed~$N$; for
otherwise, the sequence $m_{i_1}^{(1)},\dots, m_{i_{2N}}^{(2N)}$ would contain
at least~$N$ consecutive terms belonging to $M_N$ and therefore
$m_{i_1}^{(1)}\dots m_{i_{2N}}^{(2N)}= 0,$ since $M_N^N=\{0\}$.  Thus
$$|x^{(1)}\dots x^{(2N)}| \le {1\over\nu(N)}|x^{(1)}|\dots |x^{(2N)}|$$
for all $x^{(1)}, \dots,x^{(2N)}$ in~$M$ and hence, by continuity, in~$A$.
Taking $\nu(n) = n^n$ makes~$A$ topologically nilpotent, and hence radical.

The inequality $\|x\| \le |x|$ $(x \in M)$ shows that the natural embedding
$\phi: (M,|.|) \to (B,\|.\|)$ is continuous and therefore extends to a
continuous homomorphism $\phi:A \to B$.   However, it is not clear that
$\phi:A \to B$ is necessarily injective.   If not, we obtain an injective map
by simply replacing~$A$ by $A/\ker\phi$.
\end{Proof}
%==============================================================================
\section{Radicals in incomplete algebras}\label{S10}

Our theory of topological radicals can be developed equally well in
the context of incomplete normed algebras: simply replace `Banach algebra' by
`normed algebra' throughout Section~\ref{S5}.   Section~\ref{S6} may be treated
likewise: the maps $T_n$ are topological radicals for normed algebras.  In
Section~\ref{S7}, the same recipe applies,  except that the algebra~$B$ must
remain Banach and the inequality $S_* \le J$ of Corollary~\ref{T12} no longer
applies (see Example~\ref{Ex4} below).

This alternative theory has the advantage that its axiom (3) gives information
about the behaviour of the radical under all continuous homomorphisms, not just
those with complete range.    However, it has a major disadvantage: it excludes
the Jacobson radical because the Jacobson radical of an incomplete normed
algebra is not necessarily closed.

\begin{Example}\label{Ex3}
Let $B$ be the subalgebra of $(C[0,1],*)$ consisting of the polynomials.
Then~$B$ is algebraically isomorphic to the subalgebra of $\C[X]$
consisting of polynomials without constant term.   Therefore
$J(B) = \{0\}$, indeed,~$B$ has a separating family of
(discontinuous) transitive 1-dimensional representations.

Now let $A$ be the subalgebra of $(C[0,1],*)$ consisting of all functions
which are polynomial on a neighbourhood of~0.   There is an obvious
homomorphism of~$A$ onto~$B$ and so the Jacobson radical $J(A)$ is
contained in the inverse image of $\{0\}$; that is, the set of functions
vanishing in a neighbourhood of~$0$.  The reverse inclusion is
obvious: in fact, if~$f \in A$ vanishes on a neighbourhood of~$0$,
then~$f$ is nilpotent in~$A$.  Thus
$$J(A) = \{f \in A: f \equiv 0 \mbox{ on a neighbourhood of }0\},$$
which is not closed: $\closure{J(A)}$ is the ideal of functions
vanishing at zero.

\detail{Alternatively: if $f(x) = p(x)$ $(0 \le x \le \delta)$ for some
non-zero polynomial~$p$ and some $\delta > 0$, then $f$ cannot be
quasi-regular, for if~$g$ were a quasi-inverse for~$f$ in~$A$ then
$g(x) = q(x)$ $(0 \le q(x) \le \e)$ for some polynomial~$q$ and some
$\e > 0$.  But if
$$p(x) = a_nx^n + \mbox{ lower order terms}$$
and
$$q(x) = b_mx^m + \mbox{ lower order terms},$$
then
$$p(x) + q(x) + (p*q)(x) = a_nb_m(x^n*x^m) + \mbox{ lower order terms}$$
and $x^n*x^m$ is a non-zero multiple of $x^{n+m+1}$.  Hence
$p + q + p*q \ne 0$; contradiction.}%
\end{Example}

This example is very similar to Example \ref{C01}.  As there, the quotient
$A/\overline{J}(A)$ is the one-dimensional algebra with zero multiplication,
so $\overline{J}$ does not satisfy axiom (2), and we have to go to
$\barstar J$ to obtain a topological radical.   However, this is only one of the possible ways
to get a topological radical for normed algebras which reduces to the Jacobson
radical for Banach algebras.
In order to study these, we need some basic information about representations of
incomplete algebras.   This is a little-studied topic.   It is well-known that
strictly irreducible representations of Banach algebras are transitive, (which
means that we do not have to consider radicals based on different degrees of
transitivity),  but the following variation in which the completeness
hypothesis is on the space rather than the algebra seems not to be available
in the literature.

\begin{thm}\label{T10}
Every strictly irreducible normed representation of an algebra on a Banach
space is transitive.
\end{thm}

\begin{Proof}
Let $\pi:A \to \LX$ be a strictly irreducible representation of~$A$ on the
Banach space~$X$.
Let us write $\UX$ for the algebra of all linear
endomorphisms of~$X$.   Then Schur's Lemma tells us that
$$\{S \in \UX: S\pi(a) = \pi(a)S\quad(a \in A)\}$$
is a division algebra.    We are interested in the set
$$\pi(A)' = \{S \in \LX: S\pi(a) = \pi(a)S\quad(a \in A)\}.$$
Then every $S \in \pi(A)'$ is bijective, and therefore, by
Banach's Isomorphism Theorem, has an inverse in $\pi(A)'$.  Thus
$\pi(A)'$ is a normed division algebra.  By the Gelfand-Mazur Theorem,
$\pi(A)'$ consists of the scalar multiples of the identity.   The remainder
of the proof is standard (see \cite{Palmerbk} Theorem 4.2.13).
\end{Proof}

\begin{Remark}
If $\pi:A \to \LX$ is a continuous strictly irreducible representation
of a normed algebra~$A$ on a Banach space~$X$, then we may regard
$\pi:A \to \closure{\pi(A)}$ as a continuous homomorphism of~$A$ into the
Banach algebra $\closure{\pi(A)}$ and the identity map
$\closure{\pi(A)} \to \LX$ as a strictly irreducible
representation of $\closure{\pi(A)}$.  Thus~$\pi$ is standard.
\end{Remark}

In looking for topological radicals of normed algebras corresponding to the
Jacobson radical, let us restrict our attention to those based on continuous
representations on Banach spaces.  The natural concept to define is the
following.

\begin{Defn}
For a normed algebra~$A$, let $I(A)$ denote the intersection of the kernels of
the continuous strictly irreducible representations of~$A$ on Banach spaces.
\end{Defn}

If~$A$ is Banach, $J(A) = I(A)$.    We shall show that~$I$ is a
(hereditary) topological radical, different from $\barstar J$.

\begin{thm}\label{T11}
The mapping~$I$ is a hereditary topological radical.
\end{thm}

\begin{Proof}
The proof proceeds as in Theorem \ref{T5}, the analogue of Lemma \ref{L1}
going as follows.

If~$I$ is a closed ideal of a normed algebra~$A$ and~$\pi$ is a
continuous strictly irreducible representation of~$I$ on a Banach
space~$X$, we show that there is a continuous strictly irreducible
representation~$\hat\pi$ of~$A$ on~$X$ such that
$\ker\pi = \ker\hat\pi\cap I$.

Let $\tilde A$ be the completion of~$A$ and let $\tilde I$ be the
closure of~$I$ in $\tilde A$. Then $\tilde I$ is a closed ideal in
$\tilde A$. Since~$\pi$ is continuous, it extends to a continuous
strictly irreducible representation $\tilde\pi:\tilde I \to \LX$. We use
the usual algebraic method to extend this to a representation $\hat\pi$
mapping~$\tilde A$ into the algebra of endomorphisms of the vector
space~$X$: we choose any nonzero $x_0 \in X$; every $y \in X$ may be
written in the form $y = \tilde\pi(b)x_0$ for some $b \in \tilde I$; for
$a \in \tilde A$ we define $\hat\pi(a)$ by $\hat\pi(a)y =
\tilde\pi(ab)x_0$. It is easy to check that $\hat\pi$ is a well-defined
homomorphism. The restriction $\hat\pi|A$ to~$A$ is strictly
irreducible, since it is an extension of the strictly irreducible
representation~$\pi$. It remains to show that $\hat\pi(a) \in \LX$
and that $\hat\pi$ is continuous.

The mapping $b \mapsto \tilde\pi(b)x_0:\tilde I \to X$ is continuous,
surjective, and therefore open.   Thus, there is a constant $K > 0$ such that
for every $y \in X$ there is a $b \in \tilde I$ with $\|b\| \le K\|y\|$ such
that $y = \tilde\pi(b)x_0$.   Then
$$\|\hat\pi(a)y\| \le \|\tilde\pi\|\,\|ab\|\,\|x_0\|
                  \le K\|\tilde\pi\|\,\|a\|\,\|y\|\,\|x_0\|\qquad(a \in A,\;y\in X),$$
$$\|\hat\pi(a)\| \le K\|\tilde\pi\|\,\|a\|\,\|x_0\|\qquad(a \in A).$$
Thus $\hat\pi(a)\in \LX$ and $\hat\pi$ is continuous of norm at most
$K\|\tilde\pi\|\,\|x_0\|$.
\end{Proof}

We have $J(A) \sseq I(A)$ for all normed algebras~$A$.   Therefore, by
Theorem \ref{T4.1}, $\barstar J \le I$.
The following example shows that we can have $\barstar J \ne I$.

\begin{Example}\label{Ex4}
Let $B$ be the subalgebra of $(C[0,1],*)$ consisting of the polynomials,
discussed in Example~\ref{Ex3}. Then $J(B) = \{0\}$, and so
$\barstar J(A) = \{0\}$. On the other hand, any continuous TI representation
of~$B$ would extend by continuity to a TI representation of $(C[0,1],*)$,
which is impossible, since $\barstar\beta(C[0,1]) = C[0,1]$. Therefore
$T_1(B) = B$, so $I(B) = B$.
\end{Example}

This example shows that to make Theorem~\ref{T12}
work for incomplete normed algebras we have to replace~$J$ by~$I$:

\begin{thm}\label{T12i}
The topological radicals of general normed algebras are related by
$$\barstar\beta \le T_1 \le T_m \le T_n \le T_\infty \le S_* \le I
                                       \qquad (1 \le m \le n \le \infty).$$
$$\barstar\beta \le \barstar J \le I$$
\end{thm}

\begin{Proof}
Every continuous, strictly irreducible representation $\pi:A \to \LX$ extends to an
irreducible representation $\tilde\pi:\tilde A \to \LX$, and is therefore a
standard representation of~$A$.  Thus $S \le I$ and it follows from \detail{the
general normed algebra version of} Theorem \ref{T4.1} that $S_* \le I$.    The
other inequalities are obvious.
\end{Proof}

\begin{Remark}
It is tempting to try to define a topological radical for normed algebras by
$J'(A) = A \cap J(\tilde A)$, where $\tilde A$ is the completion of~$A$.
However, while this $J'$ does satisfy axioms (1), (2), (4) and (5), it fails to
satisfy (3).
\Detail{If~$R$ is a hereditary topological radical for Banach algebras, then $R'$ is a
hereditary topological radical for normed algebras, except insofar as it might fail to
satisfy (3).

\begin{Proof}
Since~$R$ is a hereditary topological radical, $R(\tilde A)$ is a closed ideal of $\tilde A$.
Hence $R'(A)$ is a closed ideal of~$A$.  We check the axioms (2), (4) and (5).
Axiom (1) follows from (5).

{\bf (2)}~~Using the fact that $(A/I)\:\tilde{}\: = \tilde A/\tilde I$ we see that
$$R'(A/R'(A)) = A/R'(A) \cap R((A/R'(A))\:\tilde{}\:)
            = A/R'(A) \cap R(\tilde A/R(\tilde A)) = \{0\}.$$

{\bf (4), (5)}~~If~$I$ is a closed ideal of~$A$, then $\tilde I$ is a closed
ideal of $\tilde A$, so $R(\tilde I)$ is a closed ideal of $\tilde A$ and
$R(\tilde I) = R(\tilde A) \cap \tilde I$.   Therefore
$R'(I) = I \cap R(\tilde I)$ is a closed ideal of~$A$ and
$$R'(I) = A \cap R(\tilde A) \cap \tilde I
        = (A \cap R(\tilde A)) \cap (A \cap\tilde I)
        = R'(A) \cap I.$$
\end{Proof}}

Let $(A_0,\nn{.})$, $(A_0,\|.\|)$ be as in Example \ref{MainEx}, with
completions~$A$,~$B$ respectively, and let~$\phi:(A_0,\nn{.})\to(A_0,\|.\|)$
be the identity map.   Then $J'(A_0,\nn{.}) = A_0 \cap J(A) = A_0$ and
$J'(A_0,\|.\|) = A_0 \cap J(B) = \{0\}$, so $\phi(J'(A_0,\nn{.}))$ is not
contained in $J'(\phi(A_0,\nn{.}))$.
\end{Remark}

%==============================================================================
\section{Representations on incomplete spaces}\label{S9}

Let us now consider representations of Banach algebras on incomplete spaces.
We define $U_n(A)$ to be the intersection of the kernels of the continuous $n$-TT
representations $\pi:A\to \LX$ of~$A$ on normed spaces~$X$.   As before, $U_n$
is a topological radical, the proof being the same except for the construction
of the space~$Z$ in Lemma \ref{L1}, where we simply put $Z=Y$.

Generally, we expect representations on incomplete spaces to be less
interesting, but the relaxed condition does make it easier to construct
examples.   The Banach space analogue of the following theorem is the
example (Proposition \ref{P3}) based on Read's recent work \cite{Readqn}.

\begin{thm}
There is a commutative, singly-generated, Jacobson-radical Banach algebra~$A$
with a faithful continuous TI representation on a normed space; hence
$\{0\} = U_1(A) \subset U_2(A) = A$.
\end{thm}

\begin{Proof}
Our algebra~$A$ will be the disc algebra with convolution multiplication
\begin{equation}\label{E4}
(f*g)(z) = \int_0^z f(w)g(z-w)\,\dw,
\end{equation}
the integral being taken along any path in the unit disc from~0 to~$z$.
The norm is the usual supremum norm
$$\|f\|_A = \sup_{|z|\le 1} |f(z)|.$$
This is a well-known example of a commutative Jacobson-radical Banach algebra
which is an integral domain (see \cite{Hille} p.478,
\cite{Palmerbk} 4.8.3, \cite{Rickart} A.2.11).
The restriction map $\phi:A\to L^1[0,1]$ is injective.  Let~$X = \phi(A)$.
Let $\pi:A\to \LX$ be defined by
$$\pi(a)(x) = \phi(a*\phi^{-1}(x)) = \phi(a)*x \qquad (a \in A, x \in X).$$
(The integral (\ref{E4}) may be taken along the real axis when~$z \in [0,1]$.)
It is easy to see that $\|\pi(a)(x)\|_1 \le \|a\|_A\|x\|_1$ $(a \in A, x \in X)$,
so~$\pi$ is a continuous normed representation.   Since~$\pi$ is, algebraically,
the left regular representation of an integral domain,~$\pi$ is faithful.

We must show that~$\pi$ has no nontrivial closed invariant subspaces.
This is equivalent to saying that the algebra $(X,*)$ has no nontrivial
closed ideals.   Since~$X$ is dense in $(L^1[0,1],*)$, the closed ideals
of~$X$ are of the form $X \cap I$ where~$I$ is a closed ideal of $(L^1[0,1],*)$.
\Detail{If~$A$ is a dense subalgebra of a Banach algebra~$B$ and~$I$ is a closed
ideal of~$A$, then $J=\closure{I+BI+IB+BIB}$ is a closed ideal of~$B$ with
$I \sseq J\cap A$.  We show that $J \cap A \sseq I$.  If $y \in J \cap A$, then
for all $\e > 0$ there exist $x, x_i, x_i', x_i'' \in I$,
$b_i, b_i', b_i'', b_i''' \in B$  $(1 \le i \le n)$ such that
$$\|x + \sum b_ix_i + \sum x_i'b_i' + \sum b_i''x_i''b_i''' - y\| < \e/2.$$
Since~$A$ is dense in~$B$, we may then approximate the $b_i, b_i', b_i'', b_i'''$
by $a_i, a_i', a_i'', a_i'''$ so that
$$\|x + \sum a_ix_i + \sum x_i'a_i' + \sum a_i''x_i''a_i''' - y\| < \e.$$
Thus~$y$ may be approximated arbitrarily closely by elements of~$I$.   Since
$y \in A$ and~$I$ is closed in~$A$, it follows that $y \in I$.   The fact that
every $J\cap A$ is a closed ideal of~$A$ whenever~$J$ is a closed ideal of~$B$
is obvious.}
Now the only proper closed ideals~$I$ of $(L^1[0,1],*)$ consist of
functions vanishing in a neighbourhood of zero \cite{Donoghue} and so have
$X \cap I = \{0\}$.  The result follows.
\end{Proof}

\Detail{The other relaxation we can make is to drop the requirement that our
representations be continuous.  Thus we may consider the intersection
$T_d(A)$ of all TI representations on a given algebra~$A$ on a Banach space.
This is an ideal containing $\beta(A)$, but it need not be closed; in Example
\ref{Ex3}, $\beta(A)=J(A)$ and $A/J(A)$ is isomorphic to the maximal ideal of
$\C[X]$ and therefore has a separating family of irreducible 1-dimensional
representations, so $T_d(A) = \beta(A)$ which is not closed.}
%==============================================================================
\section{Open questions}\label{S8}

The open questions in this subject outnumber the theorems and we can
pick out only a few here.   Let us begin with a wild conjecture.

\begin{Conjecture}\label{Conj1}
For all Banach algebras~$A$, $T_1(A) = \barstar \beta (A)$.
\end{Conjecture}

This would mean that TI representations play the same r\^ole for semiprime
Banach algebras that strictly irreducible representations do for Jacobson
semisimple algebras.   Note that this conjecture implies that the topological
radical $\barstar\beta$ is hereditary.

Topologically transitive representations are little understood.  The following
are two of the basic questions.

\begin{Question}\label{Q2}
Is every continuous 2-TT representation of a Banach algebra on a Banach space
topologically transitive?
\end{Question}

\begin{Question}\label{Q3}
Are the radicals $T_n$ $(2 \le n \le \infty)$ distinct?
\end{Question}

Moving further up the list of radicals we ask:

\begin{Question}\label{Q4}
Is every continuous TT representation of a Banach algebra
on a Banach space standard?
\end{Question}

\begin{Question}\label{Q5}
Are the radicals $T_\infty$ and $S_*$ distinct?
\end{Question}

Finally, there are also many questions on the general theory of radicals in
normed algebras that we have left open.  We have not been concerned to make an
exhaustive study of all the possible choices of axioms, we only wanted to find
one that worked.  Nevertheless, it would be interesting to know if any of the
alternatives we rejected can be made to work and if others can be proved
inequivalent to ours.



\begin{eqnarray*}
\nnY{\pi(a)x^{(j)} - y^{(j)}}
& = & \bignnY{\sum_{i=1}^n \pi(b_ic_i)x^{(j)} - \sum_{i=1}^n \pi(b_i)x_i^{(j)}}\\
& = & \bignnY{\sum_{i=1}^n \pi(b_i)\left(\pi(c_i)x^{(j)} - x_i^{(j)}\right)} \\
& \le & \sum_{i=1}^n \|b_i\|\,\|\pi(c_i)x^{(j)} - x_i^{(j)}\|, \qquad \mbox{by the definition of $\nnY{.}$,}\\
& < & \e.
\end{eqnarray*}





\begin{eqnarray}
                              B_r=-\frac{\mathrm{\partial} f}{\mathrm{\partial} z}{B_{0z}}{G}-r\frac{\mathrm{\partial} {B_{bz}}}{\mathrm{\partial} z},\quad
                              B_\phi =0,\quad B_z=\frac{\mathrm{\partial} f}{\mathrm{\partial} r}{B_{0z}}{G}+2{B_{bz}},\nonumber\\
\end{eqnarray}





\begin{equation}
                              \rho {_{\rm v}}=\rho _{\rm ref}(z)+\rho _0\exp \left(-\frac{z}{z_\alpha }\right),
                              \end{equation}


\begin{equation}
                              \frac{\,{\rm d}p{_{\rm v}}}{\,{\rm d}z}=\rho {_{\rm v}}g \Rightarrow \quad p{_{\rm v}}(z) = p_{\rm ref}({z_{\min }})+\int
                              _{z_{\min }}^z \rho {_{\rm v}}(z^*) g \,{\rm d}z^*,
 \end{equation}



\begin{equation}
                              {{\nabla P}} = \nabla p_{_{\rm v}}+ \nabla p_{_{\rm h}}+\nabla \frac{|\boldsymbol {B}|^2}{2} + (\boldsymbol {B}\cdot \nabla
                              )\boldsymbol {B} = (\rho _h+\rho _v) \boldsymbol {g},
                              \end{equation}



%==============================================================================
\begin{thebibliography}{99}
\bibitem{Beauzamy} B.~Beauzamy, {\em Introduction to operator theory and
invariant subspaces}, (North-Holland, North-Holland Mathematical Library {\bf
42}, Amsterdam 1988).
\bibitem{BD} F.~F.~Bonsall and J.~Duncan, {\em Complete normed algebras.}
(Springer, Berlin--Heidelberg--New York, 1973).
\bibitem{Divinskybk} N.~J.~Divinsky, {\em Rings and radicals}, (George Allen
\& Unwin, London, 1965).
\bibitem{PGD7}  P.~G.~Dixon, ``Semiprime Banach algebras'', {\em J. London Math. Soc.} (2),
{\bf 6} (1973), 676--678.
\bibitem{PGD12} P.~G.~Dixon, ``A Jacobson-semisimple Banach algebra with a dense nil
subalgebra'', {\em Colloq. Math.,} {\bf 37} (1977), 81--82.
\bibitem{Donoghue} W.~F.~Donoghue Jr., ``The lattice of invariant subspaces of
a completely continuous quasi-nilpotent transformation''{\em Pacific J. Math.,} {\bf 7} (1957)
1031--1035.  %{\bf MR} 19 \#1066
\bibitem{Enflo} P.~Enflo, ``On the invariant subspace problem in Banach spaces'',
{\em Acta Math.}, {\bf 158} (1987), 213--313. %{\bf MR} 88j:47006
\bibitem{Hille} E.~Hille, {\em Functional analysis and semigroups},
(American Math.\ Soc., Colloquium Publications {\bf 31}, Providence, R.I., 1948)
\bibitem{Jacobson} N.~Jacobson, {\em Structure of Rings}, third edition
(Amer.\ Math.\ Soc.\ Coll.\ Publ.\ {\bf 37}, Providence, R.I., 1968).
\bibitem{Jameson} G.~J.~O.~Jameson, {\em Topology and normed spaces},
(Chapman \& Hall, 1974).
%\bibitem{BEJ} B.~E.~Johnson, ``The uniqueness of the (complete) norm topology'',
{\em Bull.\ Amer.\ Math.\ Soc.,} {\bf 73} (1967), 537--539. %{\bf MR 35} #2142
\bibitem{Kadison} R.~V.~Kadison, ``Irreducible operator algebras'',
{\em Proc.\ Nat.\ Acad.\ Sci.\ U.S.A.}, {\bf 43} (1957), 273--276.
\bibitem{Meyer1} M.~J.~Meyer, ``Continuous dense embeddings of strong Moore
algebras'', {\em Proc.\ Amer.\ Math.\ Soc.,} {\bf 116} (1992), 727--735.
\bibitem{Murphy} G.~J.~Murphy, {\em C*-algebras and operator theory}
(Academic Press, London, 1990).
\bibitem{Palmerbk} Th.~W.~Palmer, {\em Banach algebras and the general theory of *-algebras,
volume I: algebras and Banach algebras} (C.U.P., Cambridge, 1994)
\bibitem{Read} C.~J.~Read, ``A solution to the invariant subspace problem'',
{\em Bull.\ London Math.\ Soc.,} {\bf 16} (1984), 337--401. %{\bf MR} 86f:47005
\bibitem{Readl1} C.~J.~Read, ``A solution to the invariant subspace problem
on the space $\ell_1$'', {\em Bull.\ London Math.\ Soc.,} {\bf 17} (1985),
305--317. %{\bf MR} 87e:47013
\bibitem{Readqn} C.~J.~Read, ``Quasinilpotent operators and the invariant
subspace problem'', (preprint, Trinity College, Cambridge, 1995).
\bibitem{Rickart} C.~E.~Rickart, {\em General theory of Banach algebras}
(van Nostrand, Princeton, 1960).
\bibitem{Rowen} L.~H.~Rowen, {\em Ring theory: student edition.}
(Academic Press, San Diego, Ca., 1991).
\bibitem{Sakai} S.~Sakai, {\em C*-algebras and W*-algebras},
(Springer, Ergebnisse {\bf 60}, Berlin--Heidelberg--New York 1971).
\end{thebibliography}
\vspace{\baselineskip}

\noindent
Pure Mathematics Section, \\
School of Mathematics and Statistics,\\
University of Sheffield,\\
SHEFFIELD,  S3 7RH,\\
England. \\[1 ex]
e-mail:~P.Dixon@sheffield.ac.uk
\end{document}
%=======================================================================
