%LaTeX2e
\documentclass[11pt]{article}
\usepackage{amsfonts,amssymb}
\usepackage{tcolorbox}
\usepackage{courier}
\textheight 9in
\textwidth 6.2in
\topmargin 0in
\headheight 0in
\headsep 0in
\oddsidemargin 0in
\newcounter{rmenum}
\newcounter{Rmenum}
\newcounter{alenum}
\newcounter{Alenum}
\newcounter{arenum}
\newcounter{arbenum}
\newcounter{arbdashenum}
\newenvironment{rmenumerate}{\begin{list}{(\roman{rmenum})}%
{\usecounter{rmenum}}}{\end{list}}
\newenvironment{Rmenumerate}{\begin{list}{(\Roman{Rmenum})}%
{\usecounter{Rmenum}}}{\end{list}}
\newenvironment{alenumerate}{\begin{list}{(\alph{alenum})}%
{\usecounter{alenum}}}{\end{list}}
\newenvironment{Alenumerate}{\begin{list}{(\Alph{Alenum})}%
{\usecounter{Alenum}}}{\end{list}}
\newenvironment{arenumerate}{\begin{list}{{\rm\arabic{arenum}}}%
{\usecounter{arenum}}}{\end{list}}
\newenvironment{arbenumerate}{\begin{list}{{\rm(\arabic{arbenum})}}%
{\usecounter{arbenum}}}{\end{list}}
\newenvironment{arbdashenumerate}{\begin{list}{{\rm(\arabic{arbdashenum})}$'$}%
{\usecounter{arbdashenum}}}{\end{list}}

\newenvironment{centre}{\begin{center}}{\end{center}}
\newcommand{\eg}{e.g.\ }
\newcommand{\ie}{i.e.\ }
\newcommand{\etc}{{\em et cetera }}
\newcommand{\sseq}{\subseteq}
\newcommand{\ra}{\rightarrow}
\newcommand{\Ra}{\Rightarrow}
\newcommand{\implies}{\;\;\Rightarrow\;\;}
\newcommand{\And}{\;\;\&\;\;}
\def\Bbb#1{\ensuremath{\mathbb{#1}}}
\def\MakeBbb#1{%
   \edef\tmp{%
      \noexpand\def\csname#1\endcsname
         {\noexpand\protect\csname p#1\endcsname}%
      \noexpand\def\csname p#1\endcsname
         {\noexpand\Bbb{#1}}%
   }%
   \tmp
}
\MakeBbb{C}
\MakeBbb{F}
\MakeBbb{N}
\MakeBbb{Q}
\MakeBbb{R}
\MakeBbb{T}
\MakeBbb{Z}

\let\leq\undefined  \let\geq\undefined
\DeclareMathSymbol{\leq}     {\mathrel}{AMSa}{"36}
\DeclareMathSymbol{\geq}     {\mathrel}{AMSa}{"3E}
\let\le\leq   \let\ge\geq
  \let\mho\undefined            \let\sqsupset\undefined
  \let\Join\undefined           \let\lhd\undefined
  \let\Box\undefined            \let\unlhd\undefined
  \let\Diamond\undefined        \let\rhd\undefined
  \let\leadsto\undefined        \let\unrhd\undefined
  \let\sqsubset\undefined

  \DeclareMathSymbol\mho     {\mathord}{AMSb}{"66}
  \DeclareMathSymbol\Box     {\mathord}{AMSa}{"03}
  \let\square\Box
  \DeclareMathSymbol\Diamond {\mathord}{AMSa}{"06}
  \DeclareMathSymbol\leadsto {\mathrel}{AMSa}{"20}
  \DeclareMathSymbol\sqsubset{\mathrel}{AMSa}{"40}
  \DeclareMathSymbol\sqsupset{\mathrel}{AMSa}{"41}
  \DeclareMathSymbol\lhd     {\mathrel}{AMSa}{"43}
  \DeclareMathSymbol\unlhd   {\mathrel}{AMSa}{"45}
  \DeclareMathSymbol\rhd     {\mathrel}{AMSa}{"42}
  \DeclareMathSymbol\unrhd   {\mathrel}{AMSa}{"44}
  \def\Join{\mathrel{{\rhd}\mkern-4mu{\lhd}}}


\def\nra{\hbox{$~\rightarrow~\kern-1.4em\hbox{/}~\kern+0.4em$}}
\def\sumprime_#1{\setbox0=\hbox{$\scriptstyle{#1}$}
	 \setbox2=\hbox{$\displaystyle{\sum}$}
	 \setbox4=\hbox{${}'\mathsurround=0pt$}
	 \dimen0=.5\wd0 \advance\dimen0 by-.5\wd2
	 \ifdim\dimen0>0pt
	 \ifdim\dimen0>\wd \kern\wd4 \else\kern\dimen0\fi\fi
  \mathop{{\sum}'}_{\kern-\wd4 #1}}
\newcommand{\nin}{\not\in}
\newcommand{\projtensor}{\hat{\otimes}}
\newcommand{\sgnbar}{\overline{\mbox{sgn}}}
\newcommand{\closure}[1]{\overline{#1}}
\renewcommand{\epsilon}{\varepsilon}
\renewcommand{\emptyset}{\mbox{\O}}
\newcommand{\dw}{dw}
\newcommand{\al}{\alpha}
\newcommand{\bt}{\beta}
\newcommand{\dl}{\delta}
\newcommand{\e}{\varepsilon}
\newcommand{\lm}{\lambda}
\newcommand{\onto}{\twoheadrightarrow}
\newcommand{\idealin}{\lhd}
\newtheorem{thm}{Theorem}[section]
\newtheorem{prop}[thm]{Proposition}
\newtheorem{cor}[thm]{Corollary}
\newtheorem{lemma}[thm]{Lemma}
\newenvironment{Proof}{{\it Proof. }}{}%no end of proof sign
\newenvironment{PoC}{{\it Proof of Corollary. }}{}
\newenvironment{PoL}{{\it Proof of Lemma. }}{}
\newenvironment{PoT}{{\it Proof of Theorem. }}{}
\newcommand{\defn}{\par\noindent{\bf Definition.}}
\newenvironment{defnun}{\par\noindent{\bf Definition. }}{\par}
\newenvironment{questionun}{\par\noindent{\bf Question. }}{\par}
\newenvironment{conjectureun}{\par\noindent{\bf Conjecture. }}{\par}
\newenvironment{thmun}{\par\noindent{\bf Theorem. }\begin{it}}{\end{it}\par}
\newenvironment{lemmaun}{\par\noindent{\bf Lemma. }\begin{it}}{\end{it}\par}
\newenvironment{propun}{\par\noindent{\bf Proposition. }\begin{it}}{\end{it}\par}
\newenvironment{corun}{\par\noindent{\bf Corollary. }\begin{it}}{\end{it}\par}
\newenvironment{defnno}[1]{\par\noindent{\bf Definition #1. }}{\par}
\newenvironment{questionno}[1]{\par\noindent{\bf Question #1. }}{\par}
\newenvironment{thmno}[1]{\par\noindent{\bf Theorem #1. }\begin{it}}{\end{it}\par}
\newenvironment{lemmano}[1]{\par\noindent{\bf Lemma #1. }\begin{it}}{\end{it}\par}
\newenvironment{propno}[1]{\par\noindent{\bf Proposition #1. }\begin{it}}{\end{it}\par}
\newenvironment{corno}[1]{\par\noindent{\bf Corollary #1. }\begin{it}}{\end{it}\par}
\newtheorem{definition}[thm]{Definition}
\newtheorem{example}[thm]{Example}
\newtheorem{examples}[thm]{Examples}
\newtheorem{exercise}[thm]{Exercise}
\newtheorem{remk}[thm]{Remark}
\newtheorem{remks}[thm]{Remarks}
\newtheorem{qu}[thm]{Question}
\newtheorem{conject}[thm]{Conjecture}
\newenvironment{Defn}{\begin{definition}\begin{rm}}{\end{rm}\end{definition}}
\newenvironment{Example}{\begin{example}\begin{rm}}{\end{rm}\end{example}}
\newenvironment{Examples}{\begin{examples}\begin{rm}}{\end{rm}\end{examples}}
\newenvironment{Exercise}{\begin{exercise}\begin{rm}}{\end{rm}\end{exercise}}
\newenvironment{Remark}{\begin{remk}\begin{rm}}{\end{rm}\end{remk}}
\newenvironment{Remarks}{\begin{remks}\begin{rm}}{\end{rm}\end{remks}}
\newenvironment{Question}{\begin{qu}\begin{rm}}{\end{rm}\end{qu}}
\newenvironment{Conjecture}{\begin{conject}\begin{rm}}{\end{rm}\end{conject}}
 \newcommand{\detail}[1]{}
 \newcommand{\Detail}[1]{}
%\newcommand{\detail}[1]{[#1]}
%\newcommand{\Detail}[1]{\par[{\bf Details.~~}#1]\par}
\newcommand{\nn}[1]{|#1|}
\newcommand{\bignn}[1]{\left|#1\right|}
\newcommand{\nnY}[1]{|#1|}
\newcommand{\bignnY}[1]{\left|#1\right|}
\newcommand{\Rbar}{\overline{R}}
\newcommand{\Rstar}{R^*}
\newcommand{\Rsubstar}{R_*}
\newcommand{\Rtilde}{\tilde R}
\newcommand{\barstar}[1]{{\closure{#1}\kern 1pt}^*}
\newcommand{\LE}{{\cal L}(E)}
\newcommand{\LH}{{\cal L}(H)}
\newcommand{\LW}{{\cal L}(W)}
\newcommand{\LX}{{\cal L}(X)}
\newcommand{\LY}{{\cal L}(Y)}
\newcommand{\LZ}{{\cal L}(Z)}
\newcommand{\UX}{{\cal U}(X)}
\begin{document}
%==============================================================================
\title{Algebraic Expressions}
\author{M.~K.~Griffiths}
\date{7 Jan 2018, revised 7 January 2018}
\maketitle
\insert\footins{1991 {\it Tuition notes for AS and A level Maths} 46H15, 46H25,
16Nxx.}
%==============================================================================
%\begin{abstract}
%It is shown that the topologically irreducible representations of a normed
%algebra define a certain topological radical in the same way that the strictly
%irreducible representations define the Jacobson radical and that this
%radical can be strictly smaller than the Jacobson radical.    An abstract theory
%of `topological radicals' in topological algebras is developed and
%used to relate this radical to the Baer radical (prime radical).    The
%relations with topologically transitive representations and standard
%representations in the sense of Meyer are also explored.
%\end{abstract}
%==============================================================================
\section{Introduction.}\label{S1}

Practice manipiulation of algebraic expressions

will use computer algebra progam but should practice this craft by hand!

Examples will be highlighted using green text boxes:

\begin{tcolorbox}[colback=green!5!white,colframe=green!75!black]
 
After simlification the expression
\begin{equation}
5x^2y+6xy^2+12xy
\end{equation}

becomes
\begin{equation}
xy(5x+6y+12)
\end{equation}

\end{tcolorbox}


Examples using maxima will be shown in a red textbox these will feature examples of how to use maxima


\begin{tcolorbox}[colback=red!5!white,colframe=red!75!black]
A simplification example using the ratsimp maxima command. Simplify the expression:
\begin{equation}
5x^2y+6xy^2+12xy
\end{equation}

 \begin{verbatim}
(%i1)f:6*x*y^2+5*x^2*y+12*x*y;
\end{verbatim}
 $$6\,x\,y^2+5\,x^2\,y+12\,x\,y\leqno{\tt (\%o1)}$$
\begin{verbatim}
(%i2)ratsimp(f);
\end{verbatim}
 $$6\,x\,y^2+\left(5\,x^2+12\,x\right)\,y\leqno{\tt (\%o2)}$$

\end{tcolorbox}


\begin{description}
\item[$\bullet$] Multiply and divide integer powers
\item[$\bullet$] Expand a single term over brackets and collect like terms
\item[$\bullet$] Expand the product of multiple expressions
\item[$\bullet$] Factorise linear, quadratic and cubic expressions
\item[$\bullet$] Knowledge and use of the law of indices
\item[$\bullet$] Simplify and use the rules of surds
\item[$\bullet$] Rationalise denominatorsd
\end{description}

Extra information from \cite{khanacademyalgebra1}


%==============================================================================
\section{Revision}\label{S2}







An important skill for mathematicians is the use of algebra to solve problems and develop models. A basic skill is to be able to manipulate simple expressions.
Simplify expressions such as
\begin{tcolorbox}[colback=green!5!white,colframe=green!75!black]
 
\begin{equation}
5x^2y+6x^2y^2+12xy+4x^2y^2-2xy
\end{equation}
or
\begin{equation}
4ab+2a^2b^2+4b^2+4a^2
\end{equation}

\end{tcolorbox}


Understand indices and be able to write the following expressions as a single power
\begin{tcolorbox}[colback=green!5!white,colframe=green!75!black]
 
\begin{equation}
3^3 \times 3^6
\end{equation}
or
\begin{equation}
(2^2)^3
\end{equation}

\begin{equation}
(4^5) \over (4^2)
\end{equation}

\end{tcolorbox}


Expand expressions such as

\begin{tcolorbox}[colback=green!5!white,colframe=green!75!black]
 
\begin{equation}
2(a+2b)
\end{equation}
or
\begin{equation}
5(2-2x)
\end{equation}

\begin{equation}
4(3x+4y)
\end{equation}

\end{tcolorbox}


Identify the highest common factors for expressions such as

\begin{tcolorbox}[colback=green!5!white,colframe=green!75!black]
 
36 and 108 \newline
or \newline

$$ 6x $$  and $$ 36x^3 $$

or \newline

$$   12x^2y  $$  and $$  48xy^2 $$


\end{tcolorbox}


Simplify the following


\begin{tcolorbox}[colback=green!5!white,colframe=green!75!black]
 
$$     15x \over 3 $$ \newline
or \newline

$$ 40a \over 8 $$   \newline


\end{tcolorbox}










%==============================================================================
\section{Index Laws}\label{S3}
Simplification of operations involving powers requires the laws of indices. The rules for simplifying powers of the same base is shown below.

\begin{tcolorbox}[colback=green!5!white,colframe=green!75!black]
 
$$  a^{m} \times a^{n}=a^{m+n}  $$ \newline
 \newline

$$  a^{m} \div a^{n}=a^{m-n}  $$ \newline
 \newline

$$  (a^{m})^{n}=a^{mn}  $$ \newline
 \newline

$$  (ab)^{n}=a^{n}b^{n}  $$ \newline
 \newline

\end{tcolorbox}

Some example expansions are shown below

\begin{tcolorbox}[colback=red!5!white,colframe=red!75!black]

 \begin{verbatim}
(%i1)2*x^3*3*x^7;
\end{verbatim}

$$6\,x^{10}\leqno{\tt (\%o1)}$$

\end{tcolorbox}





An expansion example \newline

\begin{tcolorbox}[colback=red!5!white,colframe=red!75!black]

$$ 3x(5x+4)-4(3x+4)$$ \newline

\begin{verbatim}
(%i4)f:3*x*(5*x+4)-4*(3*x+4);
\end{verbatim}

\begin{verbatim}
(%i5)ratsimp(f);
\end{verbatim}

$$15\,x^2-16\leqno{\tt (\%o5)}$$ \newline

\end{tcolorbox}




An expansion requiring index subtraction \newline

\begin{tcolorbox}[colback=red!5!white,colframe=red!75!black]

$$ 6x^4-12x^7 \over{3x^4}$$ \newline

\begin{verbatim}
(%i6)g:(6*x^4-12*x^7)/(3*x^4);
\end{verbatim}

\begin{verbatim}
(%i7)ratsimp(%o6);
\end{verbatim}

$$2-4\,x^3\leqno{\tt (\%o7)}$$ \newline


\end{tcolorbox}


%==============================================================================
\section{Expanding  Brackets}\label{EB}
When we manipulate and expand expressions we need to understand the rule for finding the product of expressions enclosed within brackets. An example of such an expression is shown below:\newline
$$  (2x+4y)(4x-2y+1)  $$ \newline

To determine the product we multiply each term in one bracket with every term in the other bracket.


\begin{tcolorbox}[colback=green!5!white,colframe=green!75!black]
$$(2x+4y)(4x-2y+1)$$ \newline

$$ 2x(4x-2y+1)+4y(4x-2y+1)    $$ \newline

Multiply the first bracket \newline
$$ 8x^2 - 4xy+2x+4y(4x-2y+1)    $$ \newline

Multiply the second bracket \newline
$$ 8x^2 - 4xy+2x+16xy-8y^2+4y    $$ \newline

Group similar terms together \newline
$$ 8x^2-8y^2+2x+4y +16xy - 4xy   $$ \newline

Simplify last two terms \newline
$$ 8x^2-8y^2+2x+4y +12xy    $$ \newline

\end{tcolorbox}


Or use the expand command in maxima

\begin{tcolorbox}[colback=red!5!white,colframe=red!75!black]

\begin{verbatim}
(%i11)f:(-2*y+4*x+1)*(4*y+2*x);
\end{verbatim}

$$\left(-2\,y+4\,x+1\right)\,\left(4\,y+2\,x\right)\leqno{\tt (\%o11)}$$  \newline

\begin{verbatim}
(%i12)expand(f);
\end{verbatim}

$$-8\,y^2+12\,x\,y+4\,y+8\,x^2+2\,x\leqno{\tt (\%o12)}$$ \newline

\end{tcolorbox}

Later expansions such as the Binomial and Taylor expansions will be used to expand expressions.

%============================================================================
\section{Factorising}\label{FAC}

As well as expanding brackets it is necessary to perform the reverse operation. The reverse operation of expansion is factorisation.

The expression: \newline
$$ 8x^2-8y^2+2x+4y +12xy    $$ \newline
when factorised becomes:
$$(2x+4y)(4x-2y+1)$$ \newline
We will focus later on quadratic expressions such as
$$ax^2+bx+c$$ \newline
For example consider the quadratic equation
$$2x^2+7x-4$$ \newline
Here, $$a=2$$, $$b=7$$ and $$c=-3$$. Find two factors $$f_{1}$$ and $$f_{2}$$ such that \newline
$$f_{1}+f_{2}=b$$\newline
Using this rule, the quadratic can be written as \newline
$$ 2x^2+8x-x-4  $$\newline

The terms can be grouped as\newline
$$  (2x-1)x+(2x-1)4  $$ \newline
Finally, take out the comman factor to give
$$ (2x-1)(x+4)$$\newline

\begin{tcolorbox}[colback=green!5!white,colframe=green!75!black]
$$  4x+8=4(x+2)   $$\newline

$$  2x^2-x=x(2x-1)  $$\newline

$$  3xy+9xy^2=3xy(1+3y)  $$\newline
\end{tcolorbox}


%==============================================================================
\section{Negative Indices, Fractional Indices and Surds}\label{S5}
Rational numbers are numbers which can be written as a ratio of the form \newline
$$ \frac{a}{b}$$ where a and b are integers. Rules for the laws of indices with any rational power are as follows
\begin{tcolorbox}[colback=green!5!white,colframe=green!75!black]

$$  a^{\frac{1}{m}}=\sqrt[m]a  $$ \newline

$$   a^{\frac{n}{m}}=\sqrt[m]{(a^{n})}  $$ \newline


$$  a^{-m}=\frac{1}{a^{m}}  $$ \newline


$$  a^{0}=1  $$ \newline

\end{tcolorbox}

Consider the examples \newline
\begin{tcolorbox}[colback=green!5!white,colframe=green!75!black]
$$ x^{\frac{2}{3}}x^{\frac{4}{3}}=x^2$$ \newline

$$ (x^3)^{\frac{2}{3}}=x^2$$ \newline

$$  \sqrt[3]{125x^6}=(125x^6)^{\frac{1}{3}}  $$ \newline

$$  \frac{2x^2-x}{x^5} =\frac{2}{x^3}-\frac{1}{x^4}  $$ \newline
\end{tcolorbox}

%==============================================================================
\section{Surds and Rationalising Denominators}\label{S5a}
Surds are examples of irrational numbers i.e. numbers which cannot be written in the form $$ \frac{a}{b}$$


Rules for manipulating surds
\begin{tcolorbox}[colback=green!5!white,colframe=green!75!black]
$$ \sqrt{ab}=\sqrt{a}\times\sqrt{b}  $$\newline

$$\sqrt{\frac{a}{b}}=\frac{\sqrt{a}}{\sqrt{b}}$$ \newline
\end{tcolorbox}



A fraction with a surd can be rearranged so that the denominator is a rational number. Rules for rationalising denominators are as follows. 
\begin{tcolorbox}[colback=green!5!white,colframe=green!75!black]
$$ \frac{1}{\sqrt{a}}$$, multiply numerator and denominator by $$\sqrt{a}$$ \newline

$$\frac{1}{a+\sqrt{b}}$$, mu;tiply numerator and denominator by $$  a-\sqrt{b} $$  \newline

$$\frac{1}{a-\sqrt{b}}$$, mu;tiply numerator and denominator by $$  a+\sqrt{b} $$  \newline

\end{tcolorbox}


An example of simplifying surds
\begin{tcolorbox}[colback=green!5!white,colframe=green!75!black]

$$(2-\sqrt{3})(5+\sqrt{3})$$ \newline

Expanding this becomes \newline
$$ = 2(5+\sqrt{3})-\sqrt{3}(5+\sqrt{3})   $$ \newline

Collect like terms together \newline
$$ =10+2\sqrt{3}-5\sqrt{3}-\sqrt{9}  $$

$$ 7-3\sqrt{3} $$ \newline

\end{tcolorbox}


Factorising the denominator
\begin{tcolorbox}[colback=green!5!white,colframe=green!75!black]
$$  \frac{\sqrt{5}+\sqrt{2}}{\sqrt{5}-\sqrt{2}}$$\newline

$$ = \frac{\sqrt{5}+\sqrt{2}}{\sqrt{5}-\sqrt{2}}\frac{\sqrt{5}+\sqrt{2}}{\sqrt{5}+\sqrt{2}}$$\newline

Use $$\sqrt{2}\sqrt{5}=\sqrt{10}$$ \newline
$$  =\frac{5+\sqrt{5}\sqrt{2}+\sqrt{2}\sqrt{5}+2}{5-2} $$ \newline

$$ = \frac{7+2\sqrt{10}}{3}$$ \newline

\end{tcolorbox}

%==============================================================================
\begin{thebibliography}{99}
\bibitem{khanacademyalgebra1} Khan Academy, {\em Algebra1}, (https://www.khanacademy.org/math/algebra {\bf
42}, 2018 Khan Academy).
\bibitem{Beauzamy} B.~Beauzamy, {\em Introduction to operator theory and
invariant subspaces}, (North-Holland, North-Holland Mathematical Library {\bf
42}, Amsterdam 1988).
\bibitem{BD} F.~F.~Bonsall and J.~Duncan, {\em Complete normed algebras.}
(Springer, Berlin--Heidelberg--New York, 1973).
\bibitem{Divinskybk} N.~J.~Divinsky, {\em Rings and radicals}, (George Allen
\& Unwin, London, 1965).
\bibitem{PGD7}  P.~G.~Dixon, ``Semiprime Banach algebras'', {\em J. London Math. Soc.} (2),
{\bf 6} (1973), 676--678.
\bibitem{PGD12} P.~G.~Dixon, ``A Jacobson-semisimple Banach algebra with a dense nil
subalgebra'', {\em Colloq. Math.,} {\bf 37} (1977), 81--82.
\bibitem{Donoghue} W.~F.~Donoghue Jr., ``The lattice of invariant subspaces of
a completely continuous quasi-nilpotent transformation''{\em Pacific J. Math.,} {\bf 7} (1957)
1031--1035.  %{\bf MR} 19 \#1066
\bibitem{Enflo} P.~Enflo, ``On the invariant subspace problem in Banach spaces'',
{\em Acta Math.}, {\bf 158} (1987), 213--313. %{\bf MR} 88j:47006
\bibitem{Hille} E.~Hille, {\em Functional analysis and semigroups},
(American Math.\ Soc., Colloquium Publications {\bf 31}, Providence, R.I., 1948)
\bibitem{Jacobson} N.~Jacobson, {\em Structure of Rings}, third edition
(Amer.\ Math.\ Soc.\ Coll.\ Publ.\ {\bf 37}, Providence, R.I., 1968).
\bibitem{Jameson} G.~J.~O.~Jameson, {\em Topology and normed spaces},
(Chapman \& Hall, 1974).
%\bibitem{BEJ} B.~E.~Johnson, ``The uniqueness of the (complete) norm topology'',
{\em Bull.\ Amer.\ Math.\ Soc.,} {\bf 73} (1967), 537--539. %{\bf MR 35} #2142
\bibitem{Kadison} R.~V.~Kadison, ``Irreducible operator algebras'',
{\em Proc.\ Nat.\ Acad.\ Sci.\ U.S.A.}, {\bf 43} (1957), 273--276.
\bibitem{Meyer1} M.~J.~Meyer, ``Continuous dense embeddings of strong Moore
algebras'', {\em Proc.\ Amer.\ Math.\ Soc.,} {\bf 116} (1992), 727--735.
\bibitem{Murphy} G.~J.~Murphy, {\em C*-algebras and operator theory}
(Academic Press, London, 1990).
\bibitem{Palmerbk} Th.~W.~Palmer, {\em Banach algebras and the general theory of *-algebras,
volume I: algebras and Banach algebras} (C.U.P., Cambridge, 1994)
\bibitem{Read} C.~J.~Read, ``A solution to the invariant subspace problem'',
{\em Bull.\ London Math.\ Soc.,} {\bf 16} (1984), 337--401. %{\bf MR} 86f:47005
\bibitem{Readl1} C.~J.~Read, ``A solution to the invariant subspace problem
on the space $\ell_1$'', {\em Bull.\ London Math.\ Soc.,} {\bf 17} (1985),
305--317. %{\bf MR} 87e:47013
\bibitem{Readqn} C.~J.~Read, ``Quasinilpotent operators and the invariant
subspace problem'', (preprint, Trinity College, Cambridge, 1995).
\bibitem{Rickart} C.~E.~Rickart, {\em General theory of Banach algebras}
(van Nostrand, Princeton, 1960).
\bibitem{Rowen} L.~H.~Rowen, {\em Ring theory: student edition.}
(Academic Press, San Diego, Ca., 1991).
\bibitem{Sakai} S.~Sakai, {\em C*-algebras and W*-algebras},
(Springer, Ergebnisse {\bf 60}, Berlin--Heidelberg--New York 1971).
\end{thebibliography}
\vspace{\baselineskip}

\noindent
peakadventurelearning, \\
Chesterfield,\\
Derbyshire,\\
England. \\[1 ex]
e-mail:~mikeg2105@gmail.com
\end{document}
%=======================================================================
