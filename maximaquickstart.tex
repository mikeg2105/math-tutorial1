%LaTeX2e
\documentclass[11pt]{article}
\usepackage{amsfonts,amssymb}
\usepackage{amsmath}
\usepackage{tcolorbox}
\usepackage{courier}
\textheight 9in
\textwidth 6.2in
\topmargin 0in
\headheight 0in
\headsep 0in
\oddsidemargin 0in
\newcounter{rmenum}
\newcounter{Rmenum}
\newcounter{alenum}
\newcounter{Alenum}
\newcounter{arenum}
\newcounter{arbenum}
\newcounter{arbdashenum}
\newenvironment{rmenumerate}{\begin{list}{(\roman{rmenum})}%
{\usecounter{rmenum}}}{\end{list}}
\newenvironment{Rmenumerate}{\begin{list}{(\Roman{Rmenum})}%
{\usecounter{Rmenum}}}{\end{list}}
\newenvironment{alenumerate}{\begin{list}{(\alph{alenum})}%
{\usecounter{alenum}}}{\end{list}}
\newenvironment{Alenumerate}{\begin{list}{(\Alph{Alenum})}%
{\usecounter{Alenum}}}{\end{list}}
\newenvironment{arenumerate}{\begin{list}{{\rm\arabic{arenum}}}%
{\usecounter{arenum}}}{\end{list}}
\newenvironment{arbenumerate}{\begin{list}{{\rm(\arabic{arbenum})}}%
{\usecounter{arbenum}}}{\end{list}}
\newenvironment{arbdashenumerate}{\begin{list}{{\rm(\arabic{arbdashenum})}$'$}%
{\usecounter{arbdashenum}}}{\end{list}}

\newenvironment{centre}{\begin{center}}{\end{center}}
\newcommand{\eg}{e.g.\ }
\newcommand{\ie}{i.e.\ }
\newcommand{\etc}{{\em et cetera }}
\newcommand{\sseq}{\subseteq}
\newcommand{\ra}{\rightarrow}
\newcommand{\Ra}{\Rightarrow}
%\newcommand{\implies}{\;\;\Rightarrow\;\;}
%\newcommand{\And}{\;\;\&\;\;}
\def\Bbb#1{\ensuremath{\mathbb{#1}}}
\def\MakeBbb#1{%
   \edef\tmp{%
      \noexpand\def\csname#1\endcsname
         {\noexpand\protect\csname p#1\endcsname}%
      \noexpand\def\csname p#1\endcsname
         {\noexpand\Bbb{#1}}%
   }%
   \tmp
}
\MakeBbb{C}
\MakeBbb{F}
\MakeBbb{N}
\MakeBbb{Q}
\MakeBbb{R}
\MakeBbb{T}
\MakeBbb{Z}

\let\leq\undefined  \let\geq\undefined
\DeclareMathSymbol{\leq}     {\mathrel}{AMSa}{"36}
\DeclareMathSymbol{\geq}     {\mathrel}{AMSa}{"3E}
\let\le\leq   \let\ge\geq
  \let\mho\undefined            \let\sqsupset\undefined
  \let\Join\undefined           \let\lhd\undefined
  \let\Box\undefined            \let\unlhd\undefined
  \let\Diamond\undefined        \let\rhd\undefined
  \let\leadsto\undefined        \let\unrhd\undefined
  \let\sqsubset\undefined

  \DeclareMathSymbol\mho     {\mathord}{AMSb}{"66}
  \DeclareMathSymbol\Box     {\mathord}{AMSa}{"03}
  \let\square\Box
  \DeclareMathSymbol\Diamond {\mathord}{AMSa}{"06}
  \DeclareMathSymbol\leadsto {\mathrel}{AMSa}{"20}
  \DeclareMathSymbol\sqsubset{\mathrel}{AMSa}{"40}
  \DeclareMathSymbol\sqsupset{\mathrel}{AMSa}{"41}
  \DeclareMathSymbol\lhd     {\mathrel}{AMSa}{"43}
  \DeclareMathSymbol\unlhd   {\mathrel}{AMSa}{"45}
  \DeclareMathSymbol\rhd     {\mathrel}{AMSa}{"42}
  \DeclareMathSymbol\unrhd   {\mathrel}{AMSa}{"44}
  \def\Join{\mathrel{{\rhd}\mkern-4mu{\lhd}}}


\def\nra{\hbox{$~\rightarrow~\kern-1.4em\hbox{/}~\kern+0.4em$}}
\def\sumprime_#1{\setbox0=\hbox{$\scriptstyle{#1}$}
	 \setbox2=\hbox{$\displaystyle{\sum}$}
	 \setbox4=\hbox{${}'\mathsurround=0pt$}
	 \dimen0=.5\wd0 \advance\dimen0 by-.5\wd2
	 \ifdim\dimen0>0pt
	 \ifdim\dimen0>\wd \kern\wd4 \else\kern\dimen0\fi\fi
  \mathop{{\sum}'}_{\kern-\wd4 #1}}
\newcommand{\nin}{\not\in}
\newcommand{\projtensor}{\hat{\otimes}}
\newcommand{\sgnbar}{\overline{\mbox{sgn}}}
\newcommand{\closure}[1]{\overline{#1}}
\renewcommand{\epsilon}{\varepsilon}
\renewcommand{\emptyset}{\mbox{\O}}
\newcommand{\dw}{dw}
\newcommand{\al}{\alpha}
\newcommand{\bt}{\beta}
\newcommand{\dl}{\delta}
\newcommand{\e}{\varepsilon}
\newcommand{\lm}{\lambda}
\newcommand{\onto}{\twoheadrightarrow}
\newcommand{\idealin}{\lhd}
\newtheorem{thm}{Theorem}[section]
\newtheorem{prop}[thm]{Proposition}
\newtheorem{cor}[thm]{Corollary}
\newtheorem{lemma}[thm]{Lemma}
\newenvironment{Proof}{{\it Proof. }}{}%no end of proof sign
\newenvironment{PoC}{{\it Proof of Corollary. }}{}
\newenvironment{PoL}{{\it Proof of Lemma. }}{}
\newenvironment{PoT}{{\it Proof of Theorem. }}{}
\newcommand{\defn}{\par\noindent{\bf Definition.}}
\newenvironment{defnun}{\par\noindent{\bf Definition. }}{\par}
\newenvironment{questionun}{\par\noindent{\bf Question. }}{\par}
\newenvironment{conjectureun}{\par\noindent{\bf Conjecture. }}{\par}
\newenvironment{thmun}{\par\noindent{\bf Theorem. }\begin{it}}{\end{it}\par}
\newenvironment{lemmaun}{\par\noindent{\bf Lemma. }\begin{it}}{\end{it}\par}
\newenvironment{propun}{\par\noindent{\bf Proposition. }\begin{it}}{\end{it}\par}
\newenvironment{corun}{\par\noindent{\bf Corollary. }\begin{it}}{\end{it}\par}
\newenvironment{defnno}[1]{\par\noindent{\bf Definition #1. }}{\par}
\newenvironment{questionno}[1]{\par\noindent{\bf Question #1. }}{\par}
\newenvironment{thmno}[1]{\par\noindent{\bf Theorem #1. }\begin{it}}{\end{it}\par}
\newenvironment{lemmano}[1]{\par\noindent{\bf Lemma #1. }\begin{it}}{\end{it}\par}
\newenvironment{propno}[1]{\par\noindent{\bf Proposition #1. }\begin{it}}{\end{it}\par}
\newenvironment{corno}[1]{\par\noindent{\bf Corollary #1. }\begin{it}}{\end{it}\par}
\newtheorem{definition}[thm]{Definition}
\newtheorem{example}[thm]{Example}
\newtheorem{examples}[thm]{Examples}
\newtheorem{exercise}[thm]{Exercise}
\newtheorem{remk}[thm]{Remark}
\newtheorem{remks}[thm]{Remarks}
\newtheorem{qu}[thm]{Question}
\newtheorem{conject}[thm]{Conjecture}
\newenvironment{Defn}{\begin{definition}\begin{rm}}{\end{rm}\end{definition}}
\newenvironment{Example}{\begin{example}\begin{rm}}{\end{rm}\end{example}}
\newenvironment{Examples}{\begin{examples}\begin{rm}}{\end{rm}\end{examples}}
\newenvironment{Exercise}{\begin{exercise}\begin{rm}}{\end{rm}\end{exercise}}
\newenvironment{Remark}{\begin{remk}\begin{rm}}{\end{rm}\end{remk}}
\newenvironment{Remarks}{\begin{remks}\begin{rm}}{\end{rm}\end{remks}}
\newenvironment{Question}{\begin{qu}\begin{rm}}{\end{rm}\end{qu}}
\newenvironment{Conjecture}{\begin{conject}\begin{rm}}{\end{rm}\end{conject}}
 \newcommand{\detail}[1]{}
 \newcommand{\Detail}[1]{}
%\newcommand{\detail}[1]{[#1]}
%\newcommand{\Detail}[1]{\par[{\bf Details.~~}#1]\par}
\newcommand{\nn}[1]{|#1|}
\newcommand{\bignn}[1]{\left|#1\right|}
\newcommand{\nnY}[1]{|#1|}
\newcommand{\bignnY}[1]{\left|#1\right|}
\newcommand{\Rbar}{\overline{R}}
\newcommand{\Rstar}{R^*}
\newcommand{\Rsubstar}{R_*}
\newcommand{\Rtilde}{\tilde R}
\newcommand{\barstar}[1]{{\closure{#1}\kern 1pt}^*}
\newcommand{\LE}{{\cal L}(E)}
\newcommand{\LH}{{\cal L}(H)}
\newcommand{\LW}{{\cal L}(W)}
\newcommand{\LX}{{\cal L}(X)}
\newcommand{\LY}{{\cal L}(Y)}
\newcommand{\LZ}{{\cal L}(Z)}
\newcommand{\UX}{{\cal U}(X)}
\begin{document}
%==============================================================================
\title{Quickstart for maxima}
\author{M.~K.~Griffiths}
\date{7 Jan 2018, revised 7 January 2018}
\maketitle
\insert\footins{1991 {\it Tuition notes for AS and A level Maths} 46H15, 46H25,
16Nxx.}
%==============================================================================
%\begin{abstract}
%It is shown that the topologically irreducible representations of a normed
%algebra define a certain topological radical in the same way that the strictly
%irreducible representations define the Jacobson radical and that this
%radical can be strictly smaller than the Jacobson radical.    An abstract theory
%of `topological radicals' in topological algebras is developed and
%used to relate this radical to the Baer radical (prime radical).    The
%relations with topologically transitive representations and standard
%representations in the sense of Meyer are also explored.
%\end{abstract}
%==============================================================================
\section{Introduction.}\label{S1}



The following command reference is helpful
http://www.hippasus.com/resources/symmath/maximasym.html


See runtime environment


\begin{tcolorbox}[colback=red!5!white,colframe=red!75!black]
  My box.


$E = mc^2$  Formula of the universe


\end{tcolorbox}


\begin{tcolorbox}[colback=green!5!white,colframe=green!75!black]
  My box.


$E = mc^2$  Formula of the universe
This is not Courier font. \texttt{This is Courier font.}

\end{tcolorbox}


Notes
when you first run wxMaxima, if your firewall software complains that a socket is being opened, allow always. This is a local socket that wxMaxima (the user-friendly graphical front end) uses to communicate with Maxima (the computation engine). Ref: detailed installation walkthrough
run a 2D plot to test your installation:

$f(x): 3*x+3$
$plot2d(f(x), [x,-50,50])$


More information here.
to see details of your current Maxima/wxMaxima installation, run:

$wxbuild_info()$;





$maxima_userdir$   variable returns current working directory to set it

On that page is a link to a set of wxMaxima tutorials on the page
http://wxmaxima.sourceforge.net/wiki/index.php/Tutorials.


$http://sourceforge.net/forum/forum.php?forum_id=435775$



$http://maxima.sourceforge.net/docs/manual/maxima.html$




Good tutorials for maxima
http://web.csulb.edu/~woollett/

Practice manipiulation of algebraic expressions

will use computer algebra progam but should practice this craft by hand!


\begin{description}
\item[$\bullet$] Multiply and divide integer powers
\item[$\bullet$] Expand a single term over brackets and collect like terms
\item[$\bullet$] Expand the product of multiple expressions
\item[$\bullet$] Factorise linear, quadratic and cubic expressions
\item[$\bullet$] Knowledge and use of the law of indices
\item[$\bullet$] Simplify and use the rules of surds
\item[$\bullet$] Rationalise denominatorsd
\end{description}

Extra information from \cite{khanacademyalgebra1}


%==============================================================================
\section{Revision}\label{S2}

Some simple expressions to start with


%==============================================================================
\section{Index Laws}\label{S3}

Some of our later examples will produce, {\em inter alia}, TI representations
which are not strictly irreducible, but it is worth noting now that
satisfying these requirements alone is quite easy.

\begin{Example}\label{Ex1}
Let~$A= \ell^1(S_2)$ be the semigroup algebra of the free semigroup on two
generators $X,Y$.    Let~$T$ be the unilateral shift on $H=\ell^2$:
\begin{eqnarray*}
T(\xi_1,\xi_2,\xi_3,\dots) & = & (0,\xi_1,\xi_2,\dots)\\
T^*(\xi_1,\xi_2,\xi_3,\dots) & = & (\xi_2,\xi_3,\xi_4,\dots)
\end{eqnarray*}
Let~$\pi$ be the continuous representation of~$A$ on $\ell^2$ defined by
$\pi(\delta_X) = T$, $\pi(\delta_Y) = T^*$.   It is easy to see that
$\pi(A)$ is a *-subalgebra of $\LH$ with scalar commutant, so, by von Neumann's
Double Commutant Theorem, its strong closure is $\LH$, \ie $\pi$ is TT; but
$$\pi(A)((1,0,0,\dots))  = \ell^1,$$ so~$\pi$ is not strictly irreducible.
\end{Example}

\begin{Remark}
For *-representations of C*-algebras,
Kadison's Transitivity Theorem says that TI implies strictly irreducible
(\cite{Kadison}, see also \cite{Murphy} 5.2.2, \cite{Sakai} 1.21.17).  In
the example above, $\pi(A)$ is not closed in $\LH$.
\end{Remark}

We shall be seeking to relate the TI radical to radicals definable without
reference to representations.    In one direction this is easy, provided the
algebra is complete:  every strictly irreducible representation of a Banach algebra~$A$ has the
same kernel as some continuous strictly irreducible representation of~$A$ on a Banach
space (\cite{Palmerbk} 4.2.9, \cite{Rickart} (2.4.7)).    Hence the TI radical
of a Banach algebra is contained in the Jacobson radical, which has many
characterizations not directly involving representations (largest quasi-regular
ideal, largest ideal of topologically nilpotent elements, intersection of the
maximal modular left ideals).   It is not immediately clear
that this inclusion can be strict---in Example \ref{Ex1} above, the algebra~$A$
is semisimple so there are many other representations which are strictly
irreducible---but we shall give an example later where this is so.



%==============================================================================
\section{Expanding  Brackets}\label{S4}


%============================================================================
\section{Factorising}\label{S4a}

The obvious first question about TT representations is whether there are
TI representations which are not TT.   One way in which such representations
might occur is as the left regular representations of radical Banach algebras
with no non-trivial closed left ideals, if such exist.


\begin{Remark}
The problem of whether there exists a radical Banach algebra with no non-trivial
closed left ideals lies between two unsolved problems: the existence of a
topologically simple radical Banach algebra \detail{no non-trivial closed
two-sided ideals}\ and the existence of a topologically simple commutative
radical Banach algebra.  \detail{Might it be equivalent to one of these questions?}
\end{Remark}


%==============================================================================
\section{Negative and Fractional Indices}\label{S5}

In this section, we develop a little of a general theory of radicals in
normed algebras.    The calculations are generally straightforward once the
correct definitions are in place; 
%==============================================================================
\section{Rationalising Denominators}\label{S5a}

It is natural to ask whether, in Theorem \ref{T2}, the map~$R$ satisfying
axiom~(5) would imply $\Rbar$ satisfying~(5).    The answer is negative,
as the following example shows.



%==============================================================================
\section{Summary of Key Points}\label{S6}


%==============================================================================
\begin{thebibliography}{99}
\bibitem{khanacademyalgebra1} Khan Academy, {\em Algebra1}, (https://www.khanacademy.org/math/algebra {\bf
42}, 2018 Khan Academy).
\bibitem{Beauzamy} B.~Beauzamy, {\em Introduction to operator theory and
invariant subspaces}, (North-Holland, North-Holland Mathematical Library {\bf
42}, Amsterdam 1988).
\bibitem{BD} F.~F.~Bonsall and J.~Duncan, {\em Complete normed algebras.}
(Springer, Berlin--Heidelberg--New York, 1973).
\bibitem{Divinskybk} N.~J.~Divinsky, {\em Rings and radicals}, (George Allen
\& Unwin, London, 1965).
\bibitem{PGD7}  P.~G.~Dixon, ``Semiprime Banach algebras'', {\em J. London Math. Soc.} (2),
{\bf 6} (1973), 676--678.
\bibitem{PGD12} P.~G.~Dixon, ``A Jacobson-semisimple Banach algebra with a dense nil
subalgebra'', {\em Colloq. Math.,} {\bf 37} (1977), 81--82.
\bibitem{Donoghue} W.~F.~Donoghue Jr., ``The lattice of invariant subspaces of
a completely continuous quasi-nilpotent transformation''{\em Pacific J. Math.,} {\bf 7} (1957)
1031--1035.  %{\bf MR} 19 \#1066
\bibitem{Enflo} P.~Enflo, ``On the invariant subspace problem in Banach spaces'',
{\em Acta Math.}, {\bf 158} (1987), 213--313. %{\bf MR} 88j:47006
\bibitem{Hille} E.~Hille, {\em Functional analysis and semigroups},
(American Math.\ Soc., Colloquium Publications {\bf 31}, Providence, R.I., 1948)
\bibitem{Jacobson} N.~Jacobson, {\em Structure of Rings}, third edition
(Amer.\ Math.\ Soc.\ Coll.\ Publ.\ {\bf 37}, Providence, R.I., 1968).
\bibitem{Jameson} G.~J.~O.~Jameson, {\em Topology and normed spaces},
(Chapman \& Hall, 1974).
%\bibitem{BEJ} B.~E.~Johnson, ``The uniqueness of the (complete) norm topology'',
{\em Bull.\ Amer.\ Math.\ Soc.,} {\bf 73} (1967), 537--539. %{\bf MR 35} #2142
\bibitem{Kadison} R.~V.~Kadison, ``Irreducible operator algebras'',
{\em Proc.\ Nat.\ Acad.\ Sci.\ U.S.A.}, {\bf 43} (1957), 273--276.
\bibitem{Meyer1} M.~J.~Meyer, ``Continuous dense embeddings of strong Moore
algebras'', {\em Proc.\ Amer.\ Math.\ Soc.,} {\bf 116} (1992), 727--735.
\bibitem{Murphy} G.~J.~Murphy, {\em C*-algebras and operator theory}
(Academic Press, London, 1990).
\bibitem{Palmerbk} Th.~W.~Palmer, {\em Banach algebras and the general theory of *-algebras,
volume I: algebras and Banach algebras} (C.U.P., Cambridge, 1994)
\bibitem{Read} C.~J.~Read, ``A solution to the invariant subspace problem'',
{\em Bull.\ London Math.\ Soc.,} {\bf 16} (1984), 337--401. %{\bf MR} 86f:47005
\bibitem{Readl1} C.~J.~Read, ``A solution to the invariant subspace problem
on the space $\ell_1$'', {\em Bull.\ London Math.\ Soc.,} {\bf 17} (1985),
305--317. %{\bf MR} 87e:47013
\bibitem{Readqn} C.~J.~Read, ``Quasinilpotent operators and the invariant
subspace problem'', (preprint, Trinity College, Cambridge, 1995).
\bibitem{Rickart} C.~E.~Rickart, {\em General theory of Banach algebras}
(van Nostrand, Princeton, 1960).
\bibitem{Rowen} L.~H.~Rowen, {\em Ring theory: student edition.}
(Academic Press, San Diego, Ca., 1991).
\bibitem{Sakai} S.~Sakai, {\em C*-algebras and W*-algebras},
(Springer, Ergebnisse {\bf 60}, Berlin--Heidelberg--New York 1971).
\end{thebibliography}
\vspace{\baselineskip}

\noindent
peakadventurelearning, \\
Chesterfield,\\
Derbyshire,\\
England. \\[1 ex]
e-mail:~mikeg2105@gmail.com
\end{document}
%=======================================================================
