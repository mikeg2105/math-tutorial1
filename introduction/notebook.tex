
% Default to the notebook output style

    


% Inherit from the specified cell style.




    
\documentclass[11pt]{article}

    
    
    \usepackage[T1]{fontenc}
    % Nicer default font (+ math font) than Computer Modern for most use cases
    \usepackage{mathpazo}

    % Basic figure setup, for now with no caption control since it's done
    % automatically by Pandoc (which extracts ![](path) syntax from Markdown).
    \usepackage{graphicx}
    % We will generate all images so they have a width \maxwidth. This means
    % that they will get their normal width if they fit onto the page, but
    % are scaled down if they would overflow the margins.
    \makeatletter
    \def\maxwidth{\ifdim\Gin@nat@width>\linewidth\linewidth
    \else\Gin@nat@width\fi}
    \makeatother
    \let\Oldincludegraphics\includegraphics
    % Set max figure width to be 80% of text width, for now hardcoded.
    \renewcommand{\includegraphics}[1]{\Oldincludegraphics[width=.8\maxwidth]{#1}}
    % Ensure that by default, figures have no caption (until we provide a
    % proper Figure object with a Caption API and a way to capture that
    % in the conversion process - todo).
    \usepackage{caption}
    \DeclareCaptionLabelFormat{nolabel}{}
    \captionsetup{labelformat=nolabel}

    \usepackage{adjustbox} % Used to constrain images to a maximum size 
    \usepackage{xcolor} % Allow colors to be defined
    \usepackage{enumerate} % Needed for markdown enumerations to work
    \usepackage{geometry} % Used to adjust the document margins
    \usepackage{amsmath} % Equations
    \usepackage{amssymb} % Equations
    \usepackage{textcomp} % defines textquotesingle
    % Hack from http://tex.stackexchange.com/a/47451/13684:
    \AtBeginDocument{%
        \def\PYZsq{\textquotesingle}% Upright quotes in Pygmentized code
    }
    \usepackage{upquote} % Upright quotes for verbatim code
    \usepackage{eurosym} % defines \euro
    \usepackage[mathletters]{ucs} % Extended unicode (utf-8) support
    \usepackage[utf8x]{inputenc} % Allow utf-8 characters in the tex document
    \usepackage{fancyvrb} % verbatim replacement that allows latex
    \usepackage{grffile} % extends the file name processing of package graphics 
                         % to support a larger range 
    % The hyperref package gives us a pdf with properly built
    % internal navigation ('pdf bookmarks' for the table of contents,
    % internal cross-reference links, web links for URLs, etc.)
    \usepackage{hyperref}
    \usepackage{longtable} % longtable support required by pandoc >1.10
    \usepackage{booktabs}  % table support for pandoc > 1.12.2
    \usepackage[inline]{enumitem} % IRkernel/repr support (it uses the enumerate* environment)
    \usepackage[normalem]{ulem} % ulem is needed to support strikethroughs (\sout)
                                % normalem makes italics be italics, not underlines
    

    
    
    % Colors for the hyperref package
    \definecolor{urlcolor}{rgb}{0,.145,.698}
    \definecolor{linkcolor}{rgb}{.71,0.21,0.01}
    \definecolor{citecolor}{rgb}{.12,.54,.11}

    % ANSI colors
    \definecolor{ansi-black}{HTML}{3E424D}
    \definecolor{ansi-black-intense}{HTML}{282C36}
    \definecolor{ansi-red}{HTML}{E75C58}
    \definecolor{ansi-red-intense}{HTML}{B22B31}
    \definecolor{ansi-green}{HTML}{00A250}
    \definecolor{ansi-green-intense}{HTML}{007427}
    \definecolor{ansi-yellow}{HTML}{DDB62B}
    \definecolor{ansi-yellow-intense}{HTML}{B27D12}
    \definecolor{ansi-blue}{HTML}{208FFB}
    \definecolor{ansi-blue-intense}{HTML}{0065CA}
    \definecolor{ansi-magenta}{HTML}{D160C4}
    \definecolor{ansi-magenta-intense}{HTML}{A03196}
    \definecolor{ansi-cyan}{HTML}{60C6C8}
    \definecolor{ansi-cyan-intense}{HTML}{258F8F}
    \definecolor{ansi-white}{HTML}{C5C1B4}
    \definecolor{ansi-white-intense}{HTML}{A1A6B2}

    % commands and environments needed by pandoc snippets
    % extracted from the output of `pandoc -s`
    \providecommand{\tightlist}{%
      \setlength{\itemsep}{0pt}\setlength{\parskip}{0pt}}
    \DefineVerbatimEnvironment{Highlighting}{Verbatim}{commandchars=\\\{\}}
    % Add ',fontsize=\small' for more characters per line
    \newenvironment{Shaded}{}{}
    \newcommand{\KeywordTok}[1]{\textcolor[rgb]{0.00,0.44,0.13}{\textbf{{#1}}}}
    \newcommand{\DataTypeTok}[1]{\textcolor[rgb]{0.56,0.13,0.00}{{#1}}}
    \newcommand{\DecValTok}[1]{\textcolor[rgb]{0.25,0.63,0.44}{{#1}}}
    \newcommand{\BaseNTok}[1]{\textcolor[rgb]{0.25,0.63,0.44}{{#1}}}
    \newcommand{\FloatTok}[1]{\textcolor[rgb]{0.25,0.63,0.44}{{#1}}}
    \newcommand{\CharTok}[1]{\textcolor[rgb]{0.25,0.44,0.63}{{#1}}}
    \newcommand{\StringTok}[1]{\textcolor[rgb]{0.25,0.44,0.63}{{#1}}}
    \newcommand{\CommentTok}[1]{\textcolor[rgb]{0.38,0.63,0.69}{\textit{{#1}}}}
    \newcommand{\OtherTok}[1]{\textcolor[rgb]{0.00,0.44,0.13}{{#1}}}
    \newcommand{\AlertTok}[1]{\textcolor[rgb]{1.00,0.00,0.00}{\textbf{{#1}}}}
    \newcommand{\FunctionTok}[1]{\textcolor[rgb]{0.02,0.16,0.49}{{#1}}}
    \newcommand{\RegionMarkerTok}[1]{{#1}}
    \newcommand{\ErrorTok}[1]{\textcolor[rgb]{1.00,0.00,0.00}{\textbf{{#1}}}}
    \newcommand{\NormalTok}[1]{{#1}}
    
    % Additional commands for more recent versions of Pandoc
    \newcommand{\ConstantTok}[1]{\textcolor[rgb]{0.53,0.00,0.00}{{#1}}}
    \newcommand{\SpecialCharTok}[1]{\textcolor[rgb]{0.25,0.44,0.63}{{#1}}}
    \newcommand{\VerbatimStringTok}[1]{\textcolor[rgb]{0.25,0.44,0.63}{{#1}}}
    \newcommand{\SpecialStringTok}[1]{\textcolor[rgb]{0.73,0.40,0.53}{{#1}}}
    \newcommand{\ImportTok}[1]{{#1}}
    \newcommand{\DocumentationTok}[1]{\textcolor[rgb]{0.73,0.13,0.13}{\textit{{#1}}}}
    \newcommand{\AnnotationTok}[1]{\textcolor[rgb]{0.38,0.63,0.69}{\textbf{\textit{{#1}}}}}
    \newcommand{\CommentVarTok}[1]{\textcolor[rgb]{0.38,0.63,0.69}{\textbf{\textit{{#1}}}}}
    \newcommand{\VariableTok}[1]{\textcolor[rgb]{0.10,0.09,0.49}{{#1}}}
    \newcommand{\ControlFlowTok}[1]{\textcolor[rgb]{0.00,0.44,0.13}{\textbf{{#1}}}}
    \newcommand{\OperatorTok}[1]{\textcolor[rgb]{0.40,0.40,0.40}{{#1}}}
    \newcommand{\BuiltInTok}[1]{{#1}}
    \newcommand{\ExtensionTok}[1]{{#1}}
    \newcommand{\PreprocessorTok}[1]{\textcolor[rgb]{0.74,0.48,0.00}{{#1}}}
    \newcommand{\AttributeTok}[1]{\textcolor[rgb]{0.49,0.56,0.16}{{#1}}}
    \newcommand{\InformationTok}[1]{\textcolor[rgb]{0.38,0.63,0.69}{\textbf{\textit{{#1}}}}}
    \newcommand{\WarningTok}[1]{\textcolor[rgb]{0.38,0.63,0.69}{\textbf{\textit{{#1}}}}}
    
    
    % Define a nice break command that doesn't care if a line doesn't already
    % exist.
    \def\br{\hspace*{\fill} \\* }
    % Math Jax compatability definitions
    \def\gt{>}
    \def\lt{<}
    % Document parameters
    \title{BasicAlgebraRevision}
    
    
    

    % Pygments definitions
    
\makeatletter
\def\PY@reset{\let\PY@it=\relax \let\PY@bf=\relax%
    \let\PY@ul=\relax \let\PY@tc=\relax%
    \let\PY@bc=\relax \let\PY@ff=\relax}
\def\PY@tok#1{\csname PY@tok@#1\endcsname}
\def\PY@toks#1+{\ifx\relax#1\empty\else%
    \PY@tok{#1}\expandafter\PY@toks\fi}
\def\PY@do#1{\PY@bc{\PY@tc{\PY@ul{%
    \PY@it{\PY@bf{\PY@ff{#1}}}}}}}
\def\PY#1#2{\PY@reset\PY@toks#1+\relax+\PY@do{#2}}

\expandafter\def\csname PY@tok@gd\endcsname{\def\PY@tc##1{\textcolor[rgb]{0.63,0.00,0.00}{##1}}}
\expandafter\def\csname PY@tok@gu\endcsname{\let\PY@bf=\textbf\def\PY@tc##1{\textcolor[rgb]{0.50,0.00,0.50}{##1}}}
\expandafter\def\csname PY@tok@gt\endcsname{\def\PY@tc##1{\textcolor[rgb]{0.00,0.27,0.87}{##1}}}
\expandafter\def\csname PY@tok@gs\endcsname{\let\PY@bf=\textbf}
\expandafter\def\csname PY@tok@gr\endcsname{\def\PY@tc##1{\textcolor[rgb]{1.00,0.00,0.00}{##1}}}
\expandafter\def\csname PY@tok@cm\endcsname{\let\PY@it=\textit\def\PY@tc##1{\textcolor[rgb]{0.25,0.50,0.50}{##1}}}
\expandafter\def\csname PY@tok@vg\endcsname{\def\PY@tc##1{\textcolor[rgb]{0.10,0.09,0.49}{##1}}}
\expandafter\def\csname PY@tok@vi\endcsname{\def\PY@tc##1{\textcolor[rgb]{0.10,0.09,0.49}{##1}}}
\expandafter\def\csname PY@tok@vm\endcsname{\def\PY@tc##1{\textcolor[rgb]{0.10,0.09,0.49}{##1}}}
\expandafter\def\csname PY@tok@mh\endcsname{\def\PY@tc##1{\textcolor[rgb]{0.40,0.40,0.40}{##1}}}
\expandafter\def\csname PY@tok@cs\endcsname{\let\PY@it=\textit\def\PY@tc##1{\textcolor[rgb]{0.25,0.50,0.50}{##1}}}
\expandafter\def\csname PY@tok@ge\endcsname{\let\PY@it=\textit}
\expandafter\def\csname PY@tok@vc\endcsname{\def\PY@tc##1{\textcolor[rgb]{0.10,0.09,0.49}{##1}}}
\expandafter\def\csname PY@tok@il\endcsname{\def\PY@tc##1{\textcolor[rgb]{0.40,0.40,0.40}{##1}}}
\expandafter\def\csname PY@tok@go\endcsname{\def\PY@tc##1{\textcolor[rgb]{0.53,0.53,0.53}{##1}}}
\expandafter\def\csname PY@tok@cp\endcsname{\def\PY@tc##1{\textcolor[rgb]{0.74,0.48,0.00}{##1}}}
\expandafter\def\csname PY@tok@gi\endcsname{\def\PY@tc##1{\textcolor[rgb]{0.00,0.63,0.00}{##1}}}
\expandafter\def\csname PY@tok@gh\endcsname{\let\PY@bf=\textbf\def\PY@tc##1{\textcolor[rgb]{0.00,0.00,0.50}{##1}}}
\expandafter\def\csname PY@tok@ni\endcsname{\let\PY@bf=\textbf\def\PY@tc##1{\textcolor[rgb]{0.60,0.60,0.60}{##1}}}
\expandafter\def\csname PY@tok@nl\endcsname{\def\PY@tc##1{\textcolor[rgb]{0.63,0.63,0.00}{##1}}}
\expandafter\def\csname PY@tok@nn\endcsname{\let\PY@bf=\textbf\def\PY@tc##1{\textcolor[rgb]{0.00,0.00,1.00}{##1}}}
\expandafter\def\csname PY@tok@no\endcsname{\def\PY@tc##1{\textcolor[rgb]{0.53,0.00,0.00}{##1}}}
\expandafter\def\csname PY@tok@na\endcsname{\def\PY@tc##1{\textcolor[rgb]{0.49,0.56,0.16}{##1}}}
\expandafter\def\csname PY@tok@nb\endcsname{\def\PY@tc##1{\textcolor[rgb]{0.00,0.50,0.00}{##1}}}
\expandafter\def\csname PY@tok@nc\endcsname{\let\PY@bf=\textbf\def\PY@tc##1{\textcolor[rgb]{0.00,0.00,1.00}{##1}}}
\expandafter\def\csname PY@tok@nd\endcsname{\def\PY@tc##1{\textcolor[rgb]{0.67,0.13,1.00}{##1}}}
\expandafter\def\csname PY@tok@ne\endcsname{\let\PY@bf=\textbf\def\PY@tc##1{\textcolor[rgb]{0.82,0.25,0.23}{##1}}}
\expandafter\def\csname PY@tok@nf\endcsname{\def\PY@tc##1{\textcolor[rgb]{0.00,0.00,1.00}{##1}}}
\expandafter\def\csname PY@tok@si\endcsname{\let\PY@bf=\textbf\def\PY@tc##1{\textcolor[rgb]{0.73,0.40,0.53}{##1}}}
\expandafter\def\csname PY@tok@s2\endcsname{\def\PY@tc##1{\textcolor[rgb]{0.73,0.13,0.13}{##1}}}
\expandafter\def\csname PY@tok@nt\endcsname{\let\PY@bf=\textbf\def\PY@tc##1{\textcolor[rgb]{0.00,0.50,0.00}{##1}}}
\expandafter\def\csname PY@tok@nv\endcsname{\def\PY@tc##1{\textcolor[rgb]{0.10,0.09,0.49}{##1}}}
\expandafter\def\csname PY@tok@s1\endcsname{\def\PY@tc##1{\textcolor[rgb]{0.73,0.13,0.13}{##1}}}
\expandafter\def\csname PY@tok@dl\endcsname{\def\PY@tc##1{\textcolor[rgb]{0.73,0.13,0.13}{##1}}}
\expandafter\def\csname PY@tok@ch\endcsname{\let\PY@it=\textit\def\PY@tc##1{\textcolor[rgb]{0.25,0.50,0.50}{##1}}}
\expandafter\def\csname PY@tok@m\endcsname{\def\PY@tc##1{\textcolor[rgb]{0.40,0.40,0.40}{##1}}}
\expandafter\def\csname PY@tok@gp\endcsname{\let\PY@bf=\textbf\def\PY@tc##1{\textcolor[rgb]{0.00,0.00,0.50}{##1}}}
\expandafter\def\csname PY@tok@sh\endcsname{\def\PY@tc##1{\textcolor[rgb]{0.73,0.13,0.13}{##1}}}
\expandafter\def\csname PY@tok@ow\endcsname{\let\PY@bf=\textbf\def\PY@tc##1{\textcolor[rgb]{0.67,0.13,1.00}{##1}}}
\expandafter\def\csname PY@tok@sx\endcsname{\def\PY@tc##1{\textcolor[rgb]{0.00,0.50,0.00}{##1}}}
\expandafter\def\csname PY@tok@bp\endcsname{\def\PY@tc##1{\textcolor[rgb]{0.00,0.50,0.00}{##1}}}
\expandafter\def\csname PY@tok@c1\endcsname{\let\PY@it=\textit\def\PY@tc##1{\textcolor[rgb]{0.25,0.50,0.50}{##1}}}
\expandafter\def\csname PY@tok@fm\endcsname{\def\PY@tc##1{\textcolor[rgb]{0.00,0.00,1.00}{##1}}}
\expandafter\def\csname PY@tok@o\endcsname{\def\PY@tc##1{\textcolor[rgb]{0.40,0.40,0.40}{##1}}}
\expandafter\def\csname PY@tok@kc\endcsname{\let\PY@bf=\textbf\def\PY@tc##1{\textcolor[rgb]{0.00,0.50,0.00}{##1}}}
\expandafter\def\csname PY@tok@c\endcsname{\let\PY@it=\textit\def\PY@tc##1{\textcolor[rgb]{0.25,0.50,0.50}{##1}}}
\expandafter\def\csname PY@tok@mf\endcsname{\def\PY@tc##1{\textcolor[rgb]{0.40,0.40,0.40}{##1}}}
\expandafter\def\csname PY@tok@err\endcsname{\def\PY@bc##1{\setlength{\fboxsep}{0pt}\fcolorbox[rgb]{1.00,0.00,0.00}{1,1,1}{\strut ##1}}}
\expandafter\def\csname PY@tok@mb\endcsname{\def\PY@tc##1{\textcolor[rgb]{0.40,0.40,0.40}{##1}}}
\expandafter\def\csname PY@tok@ss\endcsname{\def\PY@tc##1{\textcolor[rgb]{0.10,0.09,0.49}{##1}}}
\expandafter\def\csname PY@tok@sr\endcsname{\def\PY@tc##1{\textcolor[rgb]{0.73,0.40,0.53}{##1}}}
\expandafter\def\csname PY@tok@mo\endcsname{\def\PY@tc##1{\textcolor[rgb]{0.40,0.40,0.40}{##1}}}
\expandafter\def\csname PY@tok@kd\endcsname{\let\PY@bf=\textbf\def\PY@tc##1{\textcolor[rgb]{0.00,0.50,0.00}{##1}}}
\expandafter\def\csname PY@tok@mi\endcsname{\def\PY@tc##1{\textcolor[rgb]{0.40,0.40,0.40}{##1}}}
\expandafter\def\csname PY@tok@kn\endcsname{\let\PY@bf=\textbf\def\PY@tc##1{\textcolor[rgb]{0.00,0.50,0.00}{##1}}}
\expandafter\def\csname PY@tok@cpf\endcsname{\let\PY@it=\textit\def\PY@tc##1{\textcolor[rgb]{0.25,0.50,0.50}{##1}}}
\expandafter\def\csname PY@tok@kr\endcsname{\let\PY@bf=\textbf\def\PY@tc##1{\textcolor[rgb]{0.00,0.50,0.00}{##1}}}
\expandafter\def\csname PY@tok@s\endcsname{\def\PY@tc##1{\textcolor[rgb]{0.73,0.13,0.13}{##1}}}
\expandafter\def\csname PY@tok@kp\endcsname{\def\PY@tc##1{\textcolor[rgb]{0.00,0.50,0.00}{##1}}}
\expandafter\def\csname PY@tok@w\endcsname{\def\PY@tc##1{\textcolor[rgb]{0.73,0.73,0.73}{##1}}}
\expandafter\def\csname PY@tok@kt\endcsname{\def\PY@tc##1{\textcolor[rgb]{0.69,0.00,0.25}{##1}}}
\expandafter\def\csname PY@tok@sc\endcsname{\def\PY@tc##1{\textcolor[rgb]{0.73,0.13,0.13}{##1}}}
\expandafter\def\csname PY@tok@sb\endcsname{\def\PY@tc##1{\textcolor[rgb]{0.73,0.13,0.13}{##1}}}
\expandafter\def\csname PY@tok@sa\endcsname{\def\PY@tc##1{\textcolor[rgb]{0.73,0.13,0.13}{##1}}}
\expandafter\def\csname PY@tok@k\endcsname{\let\PY@bf=\textbf\def\PY@tc##1{\textcolor[rgb]{0.00,0.50,0.00}{##1}}}
\expandafter\def\csname PY@tok@se\endcsname{\let\PY@bf=\textbf\def\PY@tc##1{\textcolor[rgb]{0.73,0.40,0.13}{##1}}}
\expandafter\def\csname PY@tok@sd\endcsname{\let\PY@it=\textit\def\PY@tc##1{\textcolor[rgb]{0.73,0.13,0.13}{##1}}}

\def\PYZbs{\char`\\}
\def\PYZus{\char`\_}
\def\PYZob{\char`\{}
\def\PYZcb{\char`\}}
\def\PYZca{\char`\^}
\def\PYZam{\char`\&}
\def\PYZlt{\char`\<}
\def\PYZgt{\char`\>}
\def\PYZsh{\char`\#}
\def\PYZpc{\char`\%}
\def\PYZdl{\char`\$}
\def\PYZhy{\char`\-}
\def\PYZsq{\char`\'}
\def\PYZdq{\char`\"}
\def\PYZti{\char`\~}
% for compatibility with earlier versions
\def\PYZat{@}
\def\PYZlb{[}
\def\PYZrb{]}
\makeatother


    % Exact colors from NB
    \definecolor{incolor}{rgb}{0.0, 0.0, 0.5}
    \definecolor{outcolor}{rgb}{0.545, 0.0, 0.0}



    
    % Prevent overflowing lines due to hard-to-break entities
    \sloppy 
    % Setup hyperref package
    \hypersetup{
      breaklinks=true,  % so long urls are correctly broken across lines
      colorlinks=true,
      urlcolor=urlcolor,
      linkcolor=linkcolor,
      citecolor=citecolor,
      }
    % Slightly bigger margins than the latex defaults
    
    \geometry{verbose,tmargin=1in,bmargin=1in,lmargin=1in,rmargin=1in}
    
    

    \begin{document}
    
    
    \maketitle
    
    

    
    \hypertarget{basic-algebra-revision}{%
\section{Basic Algebra Revision}\label{basic-algebra-revision}}

\begin{enumerate}
\def\labelenumi{\arabic{enumi}.}
\tightlist
\item
  Expressions, equations, variables and constants
\item
  Functions
\item
  Substitution
\item
  Re-arranging an expression
\end{enumerate}

    \hypertarget{expressions-equations-variables-and-constants}{%
\subsection{Expressions, equations, variables and
constants}\label{expressions-equations-variables-and-constants}}

An expression is a collection of numbers, letters and symbols such as
+,-,* and division. Example expressions are shown below.

    \hypertarget{addition}{%
\subsubsection{Addition}\label{addition}}

a plus 5 \[ a + 5 \] \#\#\# Subtraction a minus 5 \[  a-5 \] \#\#\#
Multiplication 4 times a (note we forgot the times symbol)! \[ 4a \]\\
\#\#\# Division a divided by 2 \[ a/2 \]

The examples are shown below
a=var('a')
b=var('b')

a+5
a-5
4*a
a/2
a-b
a-x
    The letters in the above expressions are known as variables. When the
expression is written as shown above we do not know what number that
letter is equal to.

We evaluate an equation when we set a variable to a given value for
example

    \begin{Verbatim}[commandchars=\\\{\}]
{\color{incolor}In [{\color{incolor}2}]:} \PY{n}{a}\PY{o}{=}\PY{l+m+mi}{1}
        \PY{n}{b}\PY{o}{=}\PY{l+m+mi}{2}
\end{Verbatim}


    \begin{Verbatim}[commandchars=\\\{\}]
{\color{incolor}In [{\color{incolor}4}]:} \PY{k}{print}\PY{p}{(}\PY{n}{a}\PY{o}{+}\PY{l+m+mi}{5}\PY{p}{)}
        \PY{k}{print}\PY{p}{(}\PY{n}{a}\PY{o}{\PYZhy{}}\PY{l+m+mi}{5}\PY{p}{)}
        \PY{k}{print}\PY{p}{(}\PY{l+m+mi}{4}\PY{o}{*}\PY{n}{a}\PY{p}{)}
        \PY{k}{print}\PY{p}{(}\PY{n}{a}\PY{o}{/}\PY{l+m+mi}{2}\PY{p}{)}
        \PY{k}{print}\PY{p}{(}\PY{n}{a}\PY{o}{\PYZhy{}}\PY{n}{b}\PY{p}{)}
        \PY{k}{print}\PY{p}{(}\PY{n}{a}\PY{o}{\PYZhy{}}\PY{n}{x}\PY{p}{)}
\end{Verbatim}


    \begin{Verbatim}[commandchars=\\\{\}]
6
-4
4
1/2
-1
-x + 1

    \end{Verbatim}

    We use substitution to replace the letters in the expression with the
value given by the equation.

An equation has contains an equals sign and has an expression on each
side of the sign. Example expressions are shown below.

    \begin{Verbatim}[commandchars=\\\{\}]
{\color{incolor}In [{\color{incolor}5}]:} \PY{n}{b}\PY{o}{=}\PY{n}{a}\PY{o}{+}\PY{l+m+mi}{5}
        \PY{n}{b}\PY{o}{=}\PY{n}{a}\PY{o}{\PYZhy{}}\PY{n}{x}
        \PY{n}{y}\PY{o}{=}\PY{l+m+mi}{5}\PY{o}{*}\PY{n}{a}\PY{o}{\PYZhy{}}\PY{n}{x}
\end{Verbatim}


    \hypertarget{exercise}{%
\subsubsection{Exercise}\label{exercise}}

In the examples below which examples are expressions and which examples
are equations? 1. \[ a+1 \] 2. \[ 5*a \] 3. \[ a=x+1 \] 4. \[ y=x-1 \]
5. \[ x+y-5*a \]

    \hypertarget{exercise}{%
\subsubsection{Exercise}\label{exercise}}

In the examples can you count the number of variables, write each of the
variables down? 1. \[ a+1 \] 2. \[ 5*a \] 3. \[ a=x+1 \] 4. \[ y=x-1 \]
5. \[ x+y-5*a \]

    \hypertarget{functions}{%
\subsection{Functions}\label{functions}}

A function is a mathematical expression which when provided with inputs
produces an output. We call the inputs the variables. To calculate the
value which is output by the we substitute the input values into the
expression.

    \begin{Verbatim}[commandchars=\\\{\}]
{\color{incolor}In [{\color{incolor}6}]:} \PY{n}{f}\PY{p}{(}\PY{n}{x}\PY{p}{)}\PY{o}{=}\PY{l+m+mi}{2}\PY{o}{*}\PY{n}{x}\PY{o}{+}\PY{l+m+mi}{1}
        \PY{n}{g}\PY{p}{(}\PY{n}{y}\PY{p}{)}\PY{o}{=}\PY{l+m+mi}{2}\PY{o}{*}\PY{n}{y}\PY{o}{+}\PY{l+m+mi}{1}
        \PY{n}{h}\PY{p}{(}\PY{n}{x}\PY{p}{,}\PY{n}{y}\PY{p}{)}\PY{o}{=}\PY{n}{x}\PY{o}{+}\PY{n}{y}
\end{Verbatim}


    \hypertarget{substitution}{%
\subsection{Substitution}\label{substitution}}

\hypertarget{function-of-one-variable}{%
\subsubsection{Function of One
Variable}\label{function-of-one-variable}}

Lets try some examples of substituting value into an expression which
has one variable. Here, we will use SageMath to compute the result of
the substitution

First we tell SageMath the variables we are using.

    \begin{Verbatim}[commandchars=\\\{\}]
{\color{incolor}In [{\color{incolor}7}]:} \PY{n}{x}\PY{o}{=}\PY{n}{var}\PY{p}{(}\PY{l+s+s1}{\PYZsq{}}\PY{l+s+s1}{x}\PY{l+s+s1}{\PYZsq{}}\PY{p}{)}
\end{Verbatim}


    Define an expression using our newly defined SageMath variable, we give
this function the name f

    \begin{Verbatim}[commandchars=\\\{\}]
{\color{incolor}In [{\color{incolor}8}]:} \PY{n}{f}\PY{o}{=}\PY{l+m+mi}{5}\PY{o}{*}\PY{n}{x}
        \PY{k}{print}\PY{p}{(}\PY{n}{f}\PY{p}{)}
\end{Verbatim}


    \begin{Verbatim}[commandchars=\\\{\}]
5*x

    \end{Verbatim}

    \begin{Verbatim}[commandchars=\\\{\}]
{\color{incolor}In [{\color{incolor}9}]:} \PY{n}{f}\PY{o}{=}\PY{l+m+mi}{5}\PY{o}{*}\PY{n}{x}\PY{o}{+}\PY{l+m+mi}{3}
        \PY{k}{print}\PY{p}{(}\PY{n}{f}\PY{p}{)}
\end{Verbatim}


    \begin{Verbatim}[commandchars=\\\{\}]
5*x + 3

    \end{Verbatim}

    \begin{Verbatim}[commandchars=\\\{\}]
{\color{incolor}In [{\color{incolor}10}]:} \PY{n}{f}\PY{o}{=}\PY{l+m+mi}{5}\PY{o}{*}\PY{n}{x}\PY{o}{+}\PY{l+m+mi}{6}
         \PY{k}{print}\PY{p}{(}\PY{n}{f}\PY{p}{)}
\end{Verbatim}


    \begin{Verbatim}[commandchars=\\\{\}]
5*x + 6

    \end{Verbatim}

    Now we shall actually substitute a value into our expression. We
substitute the value \[ x=1 \] into the equation \[ f= 5x+6 \]

    We use the SageMath substitute command note how we use the syntax
f.substitute also note the double equal sign (==)

    \begin{Verbatim}[commandchars=\\\{\}]
{\color{incolor}In [{\color{incolor}11}]:} \PY{k}{print}\PY{p}{(}\PY{n}{f}\PY{o}{.}\PY{n}{substitute}\PY{p}{(}\PY{n}{x}\PY{o}{==}\PY{l+m+mi}{1}\PY{p}{)}\PY{p}{)}
         \PY{k}{print}\PY{p}{(}\PY{n}{f}\PY{p}{)}
\end{Verbatim}


    \begin{Verbatim}[commandchars=\\\{\}]
11
5*x + 6

    \end{Verbatim}

    \begin{Verbatim}[commandchars=\\\{\}]
{\color{incolor}In [{\color{incolor}12}]:} \PY{n}{g}\PY{p}{(}\PY{n}{x}\PY{p}{)}\PY{o}{=}\PY{l+m+mi}{2}\PY{o}{*}\PY{n}{x}\PY{o}{+}\PY{l+m+mi}{1}
         \PY{k}{print}\PY{p}{(}\PY{n}{g}\PY{p}{(}\PY{l+m+mi}{1}\PY{p}{)}\PY{p}{)}
\end{Verbatim}


    \begin{Verbatim}[commandchars=\\\{\}]
3

    \end{Verbatim}

    \hypertarget{now-try-some-substitutions-for-yourself}{%
\subsubsection{Now try some substitutions for
yourself}\label{now-try-some-substitutions-for-yourself}}

For each problem substitute the value provided for the variable

\begin{enumerate}
\def\labelenumi{\arabic{enumi}.}
\item
  Substitute \[ x=1 \] into \[ (x-1) \]
\item
  Substitute \[ x=2 \] into \[ (x+2) \]
\item
  Substitute \[ x=2 \] into \[ 2x \]
\item
  Substitute \[ x=3 \] into \[ 5x+3 \]
\item
  Substitute \[ x=3 \] into \[ 2(6x+2) \]
\end{enumerate}

    \hypertarget{function-of-two-variables}{%
\subsubsection{Function of Two
Variables}\label{function-of-two-variables}}

We can introduce more variables into our equations and expressions. When
we use SageMath we use the command var(`variablename') to introduce
another variable.

Note below that we use functions to substitute or input the variables
into the expressions.

    \begin{Verbatim}[commandchars=\\\{\}]
{\color{incolor}In [{\color{incolor}13}]:} \PY{n}{y}\PY{o}{=}\PY{n}{var}\PY{p}{(}\PY{l+s+s1}{\PYZsq{}}\PY{l+s+s1}{y}\PY{l+s+s1}{\PYZsq{}}\PY{p}{)}
         
         \PY{n}{h}\PY{p}{(}\PY{n}{x}\PY{p}{,}\PY{n}{y}\PY{p}{)}\PY{o}{=}\PY{n}{x}\PY{o}{+}\PY{n}{y}
         \PY{k}{print}\PY{p}{(}\PY{n}{h}\PY{p}{(}\PY{l+m+mi}{1}\PY{p}{,}\PY{l+m+mi}{1}\PY{p}{)}\PY{p}{)}
\end{Verbatim}


    \begin{Verbatim}[commandchars=\\\{\}]
2

    \end{Verbatim}

    \begin{Verbatim}[commandchars=\\\{\}]
{\color{incolor}In [{\color{incolor}14}]:} \PY{n}{f}\PY{p}{(}\PY{n}{x}\PY{p}{,}\PY{n}{y}\PY{p}{)}\PY{o}{=}\PY{l+m+mi}{5}\PY{o}{*}\PY{n}{x}\PY{o}{+}\PY{n}{y}
         \PY{k}{print}\PY{p}{(}\PY{n}{f}\PY{p}{(}\PY{n}{x}\PY{p}{,}\PY{n}{y}\PY{p}{)}\PY{p}{)}
\end{Verbatim}


    \begin{Verbatim}[commandchars=\\\{\}]
5*x + y

    \end{Verbatim}

    \begin{Verbatim}[commandchars=\\\{\}]
{\color{incolor}In [{\color{incolor}15}]:} \PY{n}{f}\PY{p}{(}\PY{n}{x}\PY{p}{,}\PY{n}{y}\PY{p}{)}\PY{o}{=}\PY{l+m+mi}{5}\PY{o}{*}\PY{n}{x}\PY{o}{+}\PY{l+m+mi}{3}\PY{o}{*}\PY{n}{y}
         \PY{k}{print}\PY{p}{(}\PY{n}{f}\PY{p}{(}\PY{n}{x}\PY{p}{,}\PY{n}{y}\PY{p}{)}\PY{p}{)}
\end{Verbatim}


    \begin{Verbatim}[commandchars=\\\{\}]
5*x + 3*y

    \end{Verbatim}

    \begin{Verbatim}[commandchars=\\\{\}]
{\color{incolor}In [{\color{incolor}16}]:} \PY{n}{f}\PY{p}{(}\PY{n}{x}\PY{p}{,}\PY{n}{y}\PY{p}{)}\PY{o}{=}\PY{l+m+mi}{5}\PY{o}{*}\PY{n}{x}\PY{o}{+}\PY{l+m+mi}{6}\PY{o}{*}\PY{n}{y}
         \PY{k}{print}\PY{p}{(}\PY{n}{f}\PY{p}{(}\PY{n}{x}\PY{p}{,}\PY{n}{y}\PY{p}{)}\PY{p}{)}
\end{Verbatim}


    \begin{Verbatim}[commandchars=\\\{\}]
5*x + 6*y

    \end{Verbatim}

    \begin{Verbatim}[commandchars=\\\{\}]
{\color{incolor}In [{\color{incolor}17}]:} \PY{n}{f}\PY{p}{(}\PY{n}{x}\PY{p}{,}\PY{n}{y}\PY{p}{)}\PY{o}{=}\PY{l+m+mi}{5}\PY{o}{*}\PY{n}{x}\PY{o}{+}\PY{l+m+mi}{6}\PY{o}{*}\PY{n}{y}
         \PY{k}{print}\PY{p}{(}\PY{n}{f}\PY{p}{(}\PY{n}{x}\PY{p}{,}\PY{l+m+mi}{1}\PY{p}{)}\PY{p}{)}
\end{Verbatim}


    \begin{Verbatim}[commandchars=\\\{\}]
5*x + 6

    \end{Verbatim}

    \hypertarget{substitutions-with-sagemath}{%
\paragraph{Substitutions with
SageMath}\label{substitutions-with-sagemath}}

To substitute values into our expressions we can use the SageMath
substitute command note the use of the double equals sign and the use of
the f.substitute sysntax.

    \begin{Verbatim}[commandchars=\\\{\}]
{\color{incolor}In [{\color{incolor}47}]:} \PY{n}{f}\PY{o}{=}\PY{l+m+mi}{5}\PY{o}{*}\PY{n}{x}\PY{o}{+}\PY{n}{y}
         \PY{k}{print}\PY{p}{(}\PY{n}{f}\PY{o}{.}\PY{n}{substitute}\PY{p}{(}\PY{n}{x}\PY{o}{==}\PY{l+m+mi}{1}\PY{p}{)}\PY{p}{)}
\end{Verbatim}


    \begin{Verbatim}[commandchars=\\\{\}]
y + 5

    \end{Verbatim}

    \begin{Verbatim}[commandchars=\\\{\}]
{\color{incolor}In [{\color{incolor}48}]:} \PY{k}{print}\PY{p}{(}\PY{n}{f}\PY{o}{.}\PY{n}{substitute}\PY{p}{(}\PY{n}{x}\PY{o}{==}\PY{l+m+mi}{1}\PY{p}{,}\PY{n}{y}\PY{o}{==}\PY{l+m+mi}{1}\PY{p}{)}\PY{p}{)}
\end{Verbatim}


    \begin{Verbatim}[commandchars=\\\{\}]
6

    \end{Verbatim}

    \begin{Verbatim}[commandchars=\\\{\}]
{\color{incolor}In [{\color{incolor}49}]:} \PY{n}{f}\PY{o}{=}\PY{l+m+mi}{5}\PY{o}{*}\PY{n}{x}\PY{o}{+}\PY{l+m+mi}{6}\PY{o}{*}\PY{n}{y}
         \PY{k}{print}\PY{p}{(}\PY{n}{f}\PY{o}{.}\PY{n}{substitute}\PY{p}{(}\PY{n}{x}\PY{o}{==}\PY{l+m+mi}{1}\PY{p}{,}\PY{n}{y}\PY{o}{==}\PY{l+m+mi}{1}\PY{p}{)}\PY{p}{)}
\end{Verbatim}


    \begin{Verbatim}[commandchars=\\\{\}]
11

    \end{Verbatim}

    \begin{Verbatim}[commandchars=\\\{\}]
{\color{incolor}In [{\color{incolor}50}]:} \PY{k}{print}\PY{p}{(}\PY{n}{f}\PY{o}{.}\PY{n}{substitute}\PY{p}{(}\PY{n}{x}\PY{o}{==}\PY{l+m+mi}{1}\PY{p}{,}\PY{n}{y}\PY{o}{==}\PY{l+m+mi}{2}\PY{p}{)}\PY{p}{)}
\end{Verbatim}


    \begin{Verbatim}[commandchars=\\\{\}]
17

    \end{Verbatim}

    \hypertarget{exercise}{%
\paragraph{Exercise}\label{exercise}}

Now try some substitutions for yourself. Try the substitutions using pen
and paper and also use SageMath.

Hint: For each of these problems use the function definition to
substitute the values. e.g. \[ f(x,y)=x+y+1 \] to substitute x=1 y=1 use
the sage math print command: print(f(1,1)

For each problem substitute the value provided for the variable.

\begin{enumerate}
\def\labelenumi{\arabic{enumi}.}
\item
  Substitute \[ x=1, y=1 \] into \[ (x+y-1) \]
\item
  Substitute \[ x=2, y=2 \] into \[ (x-y+2) \]
\item
  Substitute \[ x=2, y=2 \] into \[ 2x+y \]
\item
  Substitute \[ x=3, y=1 \] into \[ 5x+5y+3 \]
\item
  Substitute \[ x=3, y=1 \] into \[ 5(x+y)+3 \]
\item
  Substitute \[ x=3, y=1 \] into \[ 2(x+y+3) \]
\item
  Substitute \[ x=3, y=2 \] into \[ (x+y+3)/y \]
\end{enumerate}

    \begin{Verbatim}[commandchars=\\\{\}]
{\color{incolor}In [{\color{incolor} }]:} \PY{c+c1}{\PYZsh{}Test your substitutions here}
\end{Verbatim}


    \hypertarget{re-arranging-expressions}{%
\subsection{Re-Arranging Expressions}\label{re-arranging-expressions}}

Balancing an equation

What ever you do on the left hand side of an equation you must do on the
right

Consider the equation \[ x-5 =0 \]

What value must x take so that this equation is true, try and guess?

The process of finding what value of the variable x makes the equation
true is called solving the equation

    We can use the Sage math solve command to find the solution, note how we
use the double equal sign, we also specify that want to determine the
value x which solves the equation

    \begin{Verbatim}[commandchars=\\\{\}]
{\color{incolor}In [{\color{incolor}12}]:} \PY{k}{print}\PY{p}{(}\PY{n}{solve}\PY{p}{(}\PY{n}{x}\PY{o}{\PYZhy{}}\PY{l+m+mi}{5}\PY{o}{==}\PY{l+m+mi}{0}\PY{p}{,}\PY{n}{x}\PY{p}{)}\PY{p}{)}
\end{Verbatim}


    \begin{Verbatim}[commandchars=\\\{\}]
[
x == 1
]

    \end{Verbatim}

    We can also solve this equation by defining a function. We can guess
values for x and substitute these into the function we keep on making
guesses until our equation \[ x-5=0 \] is satisfied.

    \begin{Verbatim}[commandchars=\\\{\}]
{\color{incolor}In [{\color{incolor} }]:} \PY{n}{f}\PY{p}{(}\PY{n}{x}\PY{p}{)}\PY{o}{=}\PY{n}{x}\PY{o}{\PYZhy{}}\PY{l+m+mi}{5}
        \PY{k}{print}\PY{p}{(}\PY{n}{f}\PY{p}{(}\PY{l+m+mi}{10}\PY{p}{)}\PY{p}{)}
        \PY{k}{print}\PY{p}{(}\PY{n}{f}\PY{p}{(}\PY{l+m+mi}{8}\PY{p}{)}\PY{p}{)}
        \PY{k}{print}\PY{p}{(}\PY{n}{f}\PY{p}{(}\PY{l+m+mi}{6}\PY{p}{)}\PY{p}{)}
        \PY{k}{print}\PY{p}{(}\PY{n}{f}\PY{p}{(}\PY{l+m+mi}{4}\PY{p}{)}\PY{p}{)}
\end{Verbatim}


    \begin{Verbatim}[commandchars=\\\{\}]
{\color{incolor}In [{\color{incolor} }]:} \PY{k}{print}\PY{p}{(}\PY{n}{solve}\PY{p}{(}\PY{n}{f}\PY{p}{(}\PY{n}{x}\PY{p}{)}\PY{o}{==}\PY{l+m+mi}{0}\PY{p}{,}\PY{n}{x}\PY{p}{)}\PY{p}{)}
\end{Verbatim}


    Lets practice with some further examples. As an exercise try finding the
solution by defining a function and using substitution to guess the
solution

    \begin{Verbatim}[commandchars=\\\{\}]
{\color{incolor}In [{\color{incolor}13}]:} \PY{k}{print}\PY{p}{(}\PY{n}{solve}\PY{p}{(}\PY{n}{x}\PY{o}{\PYZhy{}}\PY{l+m+mi}{2}\PY{o}{==}\PY{l+m+mi}{0}\PY{p}{,}\PY{n}{x}\PY{p}{)}\PY{p}{)}
\end{Verbatim}


    \begin{Verbatim}[commandchars=\\\{\}]
[
x == 2
]

    \end{Verbatim}

    \begin{Verbatim}[commandchars=\\\{\}]
{\color{incolor}In [{\color{incolor} }]:} \PY{k}{print}\PY{p}{(}\PY{n}{solve}\PY{p}{(}\PY{l+m+mi}{2}\PY{o}{*}\PY{n}{x}\PY{o}{==}\PY{l+m+mi}{2}\PY{p}{,}\PY{n}{x}\PY{p}{)}\PY{p}{)}
\end{Verbatim}


    \begin{Verbatim}[commandchars=\\\{\}]
{\color{incolor}In [{\color{incolor} }]:} \PY{k}{print}\PY{p}{(}\PY{n}{solve}\PY{p}{(}\PY{l+m+mi}{2}\PY{o}{*}\PY{n}{x}\PY{o}{==}\PY{l+m+mi}{6}\PY{p}{,}\PY{n}{x}\PY{p}{)}\PY{p}{)}
\end{Verbatim}


    Lets try someting slightly harder

    \begin{Verbatim}[commandchars=\\\{\}]
{\color{incolor}In [{\color{incolor}14}]:} \PY{k}{print}\PY{p}{(}\PY{n}{solve}\PY{p}{(}\PY{l+m+mi}{2}\PY{o}{*}\PY{n}{x}\PY{o}{+}\PY{l+m+mi}{2}\PY{o}{==}\PY{l+m+mi}{2}\PY{p}{,}\PY{n}{x}\PY{p}{)}\PY{p}{)}
\end{Verbatim}


    \begin{Verbatim}[commandchars=\\\{\}]
[
x == 0
]

    \end{Verbatim}

    \begin{Verbatim}[commandchars=\\\{\}]
{\color{incolor}In [{\color{incolor}10}]:} \PY{n}{f1}\PY{p}{(}\PY{n}{x}\PY{p}{)}\PY{o}{=}\PY{l+m+mi}{5}\PY{o}{*}\PY{n}{x}\PY{o}{+}\PY{l+m+mi}{3}
         \PY{n}{f2}\PY{p}{(}\PY{n}{x}\PY{p}{)}\PY{o}{=}\PY{l+m+mi}{6}\PY{o}{*}\PY{n}{x}
         \PY{n}{eq}\PY{o}{=}\PY{p}{(}\PY{n}{f1}\PY{p}{(}\PY{n}{x}\PY{p}{)}\PY{o}{==}\PY{n}{f2}\PY{p}{(}\PY{n}{x}\PY{p}{)}\PY{p}{)}
         \PY{k}{print}\PY{p}{(}\PY{n}{solve}\PY{p}{(}\PY{n}{eq}\PY{p}{,}\PY{n}{x}\PY{p}{)}\PY{p}{)}
\end{Verbatim}


    \begin{Verbatim}[commandchars=\\\{\}]
[
x == 3
]

    \end{Verbatim}

    \hypertarget{exercise}{%
\subsubsection{Exercise}\label{exercise}}

Go on try the exercise !

Write down functions for the equations we have used above, then use
substitution to guess the solutions, for example

    \begin{Verbatim}[commandchars=\\\{\}]
{\color{incolor}In [{\color{incolor} }]:} \PY{n}{f1}\PY{p}{(}\PY{n}{x}\PY{p}{)}\PY{o}{=}\PY{l+m+mi}{5}\PY{o}{*}\PY{n}{x}\PY{o}{+}\PY{l+m+mi}{3}
        \PY{n}{f2}\PY{p}{(}\PY{n}{x}\PY{p}{)}\PY{o}{=}\PY{l+m+mi}{6}\PY{o}{*}\PY{n}{x}
        
        \PY{k}{print}\PY{p}{(}\PY{n}{f1}\PY{p}{(}\PY{l+m+mi}{1}\PY{p}{)}\PY{p}{)}
        \PY{k}{print}\PY{p}{(}\PY{n}{f2}\PY{p}{(}\PY{l+m+mi}{1}\PY{p}{)}\PY{p}{)}
\end{Verbatim}

Instead of guessing, a much better way of finding the solution is to rearrange the equation. Here is what we can do.
Consider the example above
$$ 5x+3=6x $$

This equation has a left hand side (LHS) and a right hand side (RHS). Now we can add numbers subtract numbers. We can also multiply and divide each side of the equation by a number.

We must obey the rule, whatever we do to the left hand side then we must also do the right hand side. We call this keeping the equation balanced.

Consider our example above
$$ f2(x)=f1(x) $$
$$ f2(x)-5x = f1(x) -5x$$

The step above is called re-arranging and if we actually work this out we'd get
$$ 5x+3 -5x = 6x - 5x$$

Can yo work this out?


    Mathematicians frequently use the term identity\ldots{} An Identity is
an equation which is two expressions on each side of the equal sign.
These are frequently used by mathematicians when they prove the truth of
the equations which are used.

    \hypertarget{further-help}{%
\subsection{Further Help}\label{further-help}}

\begin{enumerate}
\def\labelenumi{\arabic{enumi}.}
\tightlist
\item
  \href{https://www.mathplanet.com/education/algebra-1}{Math Planet -
  Algebra1}
\item
  \href{https://www.khanacademy.org/math/algebra-basics}{Khan Academy -
  Algebra Basics}
\end{enumerate}


    % Add a bibliography block to the postdoc
    
    
    
    \end{document}
